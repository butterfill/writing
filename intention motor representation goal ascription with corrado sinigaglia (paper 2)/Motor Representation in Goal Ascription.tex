%!TEX TS-program = xelatex
%!TEX encoding = UTF-8 Unicode

%\def \papersize {a4paper}
\def \papersize {letterpaper}

\documentclass[12pt,\papersize]{extarticle}
% extarticle is like article but can handle 8pt, 9pt, 10pt, 11pt, 12pt, 14pt, 17pt, and 20pt text

\def \ititle {Motor representation in goal ascription}
\def \isubtitle {}
\def \iauthor {}
\def \iauthor {C. Sinigaglia* \& S. Butterfill**
\\ 
**Dipartimento di Filosofia, Università degli Studi di Milano, Italia
\\ 
***Department of Philosophy, University of Warwick, UK}
\def \iemail{corrado.sinigaglia@unimi.it}
%\date{}

%!TEX TS-program = xelatex
%!TEX encoding = UTF-8 Unicode

\title{\ititle\\\isubtitle}
\author{\iauthor\\<{\iemail}>}

\usepackage[\papersize]{geometry} % see geometry.pdf
\geometry{twoside=false}
\geometry{headsep=2em} %keep running header away from text
\geometry{footskip=1cm} %keep page numbers away from text
\geometry{top=3cm} %increase to 3.5 if use header
\geometry{left=4cm} %increase to 3.5 if use header
\geometry{right=4cm} %increase to 3.5 if use header
\geometry{textheight=22cm}

%non-xelatex
%\usepackage[T1]{fontenc}
%\usepackage{tgpagella}

%for underline
\usepackage[normalem]{ulem}

%get the font here:
% http://scripts.sil.org/CharisSILfont

\usepackage{fontspec,xunicode}
%nb do not explicitly use package xltxtra because this introduces bugs with footnote superscripting  -- perhaps because fontspec is supposed to include it anyway.
%UPDATE:  "You need to use the no-sscript option in xltxtra: \usepackage[no-sscript]{xltxtra}, this is explained in the documentation of xltxtra.  The issue is that Sabon does not contain true superscript glyphs for every character and the no-sscript option will instead use scaled regular glyphs, which is typographically inferior, but there is no other option available when using Sabon." --- http://groups.google.com/group/comp.text.tex/browse_thread/thread/19de95be2daacade
\defaultfontfeatures{Mapping=tex-text}
%\setromanfont[Mapping=tex-text]{Charis SIL} %i.e. palatino
%\setromanfont[Mapping=tex-text]{Sabon LT Std} 
%\setromanfont[Mapping=tex-text]{Dante MT Std} 
%\setromanfont[Mapping=tex-text,Ligatures={Common}]{Hoefler Text} %comes with osx
\setromanfont[Mapping=tex-text]{Linux Libertine O} 
\setsansfont[Mapping=tex-text]{Linux Biolinum O} 
\setmonofont[Scale=MatchLowercase]{Andale Mono}


%hyperlinks and pdf metadata
%TODO avoid duplication of title & author
\usepackage{hyperref}
\hypersetup{pdfborder={0 0 0}}
\hypersetup{pdfauthor={\iauthor}}
\hypersetup{pdftitle={\ititle\isubtitle}}


%handles references to labels (e.g. sections) nicely
\usepackage{varioref}

%line spacing
\usepackage{setspace}
%\onehalfspacing
%\doublespacing
\singlespacing

\usepackage{natbib}
%\usepackage[longnamesfirst]{natbib}
\setcitestyle{aysep={}}  %philosophy style: no comma between author & year

%enable notes in right margin, defaults to ugly orange boxes TODO fix
%\usepackage[textwidth=5cm]{todonotes}

%for comments
\usepackage{verbatim}

%footnotes
\usepackage[hang]{footmisc}
\setlength{\footnotemargin}{1em}
\setlength{\footnotesep}{1em}
\footnotesep 2em

%tables
\usepackage{booktabs}
\usepackage{ctable}

%section headings
\usepackage[sf]{titlesec}
%\titlespacing*{\section}{0pt}{*3}{*0.5} %reduce vertical space after header
%large headings:
%\titleformat{\section}{\LARGE\sffamily}{\thesection.}{1em}{} 
\titlelabel{\thetitle.\quad}

%captions
\usepackage[font={small,sf}, margin=0.75cm]{caption}

%lists
\usepackage{enumitem}
\newenvironment{idescription}
{ 	
	% begin code
	\begin{description}[
		labelindent=1.5\parindent,
		leftmargin=2.5\parindent
	]
}
{ 
	%end code
	\end{description}
}


%title
\usepackage{titling}
\pretitle{
	\begin{center}
	\sffamily
	\Huge
} 
\posttitle{
	\par
	\end{center}
	\vskip 0.5em
} 
\preauthor{
	\begin{center}
	\normalsize
	\lineskip 0.5em
	\begin{tabular}[t]{c}
} 
\postauthor{
	\end{tabular}
	\par
	\end{center}
}
\predate{
	\begin{center}
	\normalsize
} 
\postdate{
	\par
	\end{center}
}


%\author{}
%\author {S. Butterfill* \& C. Sinigaglia**
%\\ 
%**Department of Philosophy, University of Warwick, UK
%\\ 
%***Dipartimento di Filosofia, Università degli Studi di Milano, Italia}
%\date{}


\begin{document}

\setlength\footnotesep{1em}

\bibliographystyle{newapa} %apalike

\maketitle
%\tableofcontents
\title{}





\begin{abstract}
\noindent
%
How could judgements about the goals of actions depend on motor representations? Many findings show that they do, but several obstacles ... Overcome obstacles by showing that motor representations support experiences of action in something like the ways in which visual representations support experiences of objects ... Implications for mindreading.
%

\ % blank line

\noindent
Key words: ***

\ % blank line

\noindent
Word count: 7 500
\end{abstract}

\tolerance=5000








\section{Introduction}

Motor representation sometimes occurs in action observation and facilitates judgements about the goals to which actions are directed (see section \vref{sec:evidence} and \citealp{rizzolatti_functional_2010} for a review). 
Suppose you observe someone's hand moving towards a pair of scissors. 
You may judge that she will shortly perform an action directed to cutting something up. 
Making this judgement may depend on representing the other's actions in motor terms \citep{boria:2009_intention, ortigue:2010_understanding}.  
This is not to say, of course, that you could not make a judgement of this type without representing the action in motor terms, only that making this particular judgement may depend on motor representation. 
How could this work? 
How could some judgements about the goals of actions ever depend on motor representations?

Our aim in this paper is to answer this question. 
The first step is to ask about the contents of motor representations.
Where judgements about the goals of observed actions depend on the observers' motor representations, what do those motor representations represent?
To understand how goal ascription judgements could depend on motor representations we need to understand how the contents of the motor representations compare with the contents of the judgements.
In section \vref{sec:content} we aim to remedy this need by identifying evidence for two claims. First, some motor representations represent not merely kinematic or dynamic features of actions but outcomes to which actions are directed. Second, although the directedness of an action to an outcome cannot be represented motorically, it can be captured by motor processes which determine how an action should unfold in order for the outcome to be realised. So the status of a particular motor representation as a representation of a goal to which an action is directed depends both on its content---on its being a representation of an outcome---and on its role in motor processes which determine how that action should unfold.
We propose that where a judgement about the goal to which an action is directed depends on motor representations, there is a motor representation of an outcome which captures the directedness of the action to the goal.

The second step in our attempt to understand how some judgements depend on motor representations concerns the apparent isolation of motor representation from thought. When judgements about the goals of observed actions depend on the observers' motor representations, which processes connect motor representations to judgements?
We shall argue (in section \vref{sec:processes} that motor representations are related to experiences.  There is a parallel between visual and motor representations. Visual representations of the particular colours of objects are related to experiences of those objects and their colours, and these experiences arguably provide reasons for judgements about the colours of objects.%
\footnote{
*Some controversy.  Campbell: yes; Davidson: no (experience plays a merely causal role).  We simply assume yes.  (Or we can duck: we're talking about appearance to the thinker of having reasons, not the fact of having reasons.)}
So the fact that visual representations enable judgements about particular colours in no way conflicts with the claim that thinkers have reasons for those judgements.  We shall argue that motor representations of outcomes are related to experiences of actions and their directedness to outcomes.  This is why thinkers can have reasons for judgements about the directedness of an action to a goal even where the judgements are enabled by motor representations---and also why it appears (veridically) to thinkers that they have such reasons.  Motor representation stands to experience of action as visual representation stands to experience of particular colours. This anyway is the view we shall defend.

This view has implications for understanding how goal ascription relates to mindreading. A familiar view in philosophy has it that goal ascription depends on the ascription of intentions, which in turn is interdependent with ascriptions of belief and desire (*refs). On this view goal ascription is possible only as part of a larger mindreading project. By contrast we shall show that in some cases the only representations involved in goal ascription are motor representations. So goal ascription can occur independently of any knowledge of mental states. It is therefore coherent to conjecture that facts about the goals of actions form part of the evidential basis for mental state ascription. Knowledge of others' minds ultimately depends on knowledge of their actions, which in turn depends on experiences made possible by motor representations.  Mindreading starts from motor experiences of actions.

***NOTE: There is not enough about the importance of the question and its wider philosophical significance.

[*rephrase (also introduction): there is evidence that is naturally interpreted as showing that motor representation facilitates judgements about the goals to which actions are directed. But there is also an obstacle to accepting this conclusion: it is unclear how this could occur.  In this paper we aim to remove the obstacle by explaining how motor representation could facilitate judgements about the goals of actions.]


\section{Does motor representation sometimes facilitate action judgements?}
\label{sec:evidence}

Before launching into the main task for this paper we shall briefly review evidence for our opening premise. Why suppose that motor representation sometimes facilitates judgements about the goals to which actions are directed?

As a preliminary, note that it is now well established that motor processes and representations are involved not only in producing actions but also in observing actions. Indeed there seem to be some striking similarities between the sorts of processes and representations usually involved in performing a particular action and those which typically occur when observing someone else perform that action \citep{rizzolatti_functional_2010,rizzolatti_mirrors_2008}. 

If motor representations occur in action observation, then observing actions might sometimes facilitate performing compatible actions and interfere with performing incompatible actions.  Both effects do indeed occur, as several studies have shown \citep{brass:2000_compatibility, craighero:2002_hand, kilner:2003_interference, costantini:2012_does}. To give just one example, \citet{kilner:2003_interference} asked subjects to move their arms horizontally while they observed another person moving their arms either horizontally (that is, congruently) or vertically (that is, incongruently). They found that subjects' movements deviated significantly more when observing incongruent actions, and much as they would deviate if the subjects had been required to move their other arm incongruently. The effect cannot be explained as a consequence of observing mere movements because the same effect was not found when subjects observed a robot moving its arm.

Of course the fact that motor representation occurs in action observation does not show, further, that motor representation actually facilitates judgements about the goals to which actions are directed.  

What would count as evidence for this further claim that motor representation can facilitate goal ascription?  One possible source of evidence involves motor expertise. If manipulating subjects' motor expertise can selectively affect their abilities to identify the goals of actions, this would be evidence that motor representation sometimes facilitates goal ascription.  Accordingly, \citet{casile:2006_nonvisual} asked subjects to make judgements about observed actions, including were some that are typically difficult to perform without training.  These actions were presented to subjects as point-light stimuli to ensure that only visual information about the actors' joint displacements was available.%
\footnote{
Readers who unfamiliar with point-light stimuli may view examples at http://www.biomotionlab.ca/Demos/BMLwalker.html.
The technique was introduced by \citet{johansson:1973_visual}.
}
They compared performance before and after subjects were trained to perform those actions themselves. They found that subjects' ability to accurately judge which action was being performed improved with the training. This is plausibly a consequence of increased motor expertise rather than any form of perceptual learning because the subjects were blindfolded during training (and the point-light stimuli on which their judgements were based are, of course, entirely visual). A related way to measure the effects of motor representation on judgements is to temporarily lesion part of the motor cortex using repetitive transcranial magnetic stimulation  \citealp{urgesi:2007_representation, moro:2008_neural}.  \citet{urgesi:2007_representation} measured how long subjects took to make judgements about the type of action they were observing. They found that a temporary lesion to the premotor cortex, but not a temporary lesion to another brain area, increased the time taken to make action judgements. And the effect of lesioning the premotor cortex was specific to action judgements: subjects' abilities to make observational judgements identifying body parts was unaffected. Putting these studies together, we have evidence that enhancing motor representation (through training) sometimes facilitates judgements about the goals of actions, while destroying motor representation even temporarily can impair such judgements.

Further evidence comes from  studies of neurological deficits. If deficits in abilities to perform certain actions specifically impair abilities to make observational judgements about actions of that type, this would also indicate that motor representation can facilitate action judgements.  \citet{serino:2009_lesions_} compared the performances of control and hemiplegic subjects who were asked to identify actions. They found that ability to perform an observed action was correlated with ability to recognise that action. In particular, hemiplegic subjects were less accurate in identifying actions performed with limbs on the hemiplegic side of their bodies than actions performed with limbs on the unaffected side. By contrast, no such pattern was found for control subjects, which included both a group of healthy subjects and also a group of brain-damaged patients with no motor deficit.  (Including the latter group makes it possible to rule out the possibility that the findings were due to a general deficit in higher-order visual processing). Other evidence for a role for motor representation in judgements of action comes from research on apraxia. In one study subjects were asked to identify actions such as cutting paper and drinking with a straw on the basis of the sounds these actions produced. Subjects with limb apraxia showed an impairment in recognising hand-related actions (such as cutting paper) whereas subjects with buccofacial apraxia were impaired in recognising mouth-related actions (such as drinking); but no subjects showed a general impairment in recognising sounds and their significance \citep{pazzaglia:2008_sound_}. These links between motor deficits and action judgements provide further evidence that motor representation sometimes facilitates action judgements.%
\footnote{
In addition to facilitating judgements about the goals of observed actions, motor representations and processes may also sometimes facilitate judgements about the possibility of performing an action \citep{grosjean:2007_fitts_law, eskenazi:2009_role}.
}

Motor representation may not always be necessary for observational judgements about the goals of actions. Indeed, as \citet{mahon:2008_action} notes, some studies suggest that apraxic subjects can recognise actions they cannot perform \citep[see also][]{hickok:2009_eight}. This is consistent with our premise that motor representation sometimes facilitates judgements about actions.  Even if this facilitation were an extremely rare phenomenon, the question of how it is possible would still arise.



\section{What can motor representations represent?}
\label{sec:content}

We have just reviewed evidence that judgements about the goals of actions  are sometimes facilitated by motor representation. Our project is to understand how this could happen.  A first step is to understand how the contents of the judgements compare with the contents of motor representations. Where judgements about the goals of actions depend on motor representations, could those motor representations represent outcomes to which the actions are directed?  In this section we argue for a positive answer on the grounds that some motor representations do represent outcomes and that there are processes which ensure that, in observing an action, the outcomes represented motorically are normally outcomes to which the observed action is directed. This will allow us to conclude that motor representations can facilitate judgements about goals in part because some motor representations represent outcomes which are among the goals of the actions.  

We take for granted that action observation sometimes involves motor representations of merely kinematic features of actions.  For example, when observing someone about to press a button, it may be useful to be able to identify with some degree of precision the trajectory her finger will follow and how long it will take to reach the button. In some cases observers represent both features of such actions motorically (*ref). But our concern is with the further claim that some motor representations represent outcomes. 

How can we distinguish a representation of an outcome from a representation of merely kinematic features of actions? This is possible because a single outcome can be realised by actions with quite different kinematic features and, conversely, actions with arbitrarily similar kinematic features can realise different outcomes in different contexts.  To illustrate, consider this outcome: the grasping of a particular pen. This could be realised by any of at least three types of action which vary kinematically: a hand action, a foot action or a tool-using action.  If we had marks we could use to identify motor representations independently of knowing their contents, and if we had evidence that these marks were constant across instances of all three types of action, then we could infer that some motor representations capture something more abstract than the kinematic features particular to the hand, foot or tool-using action. But we do have well established marks of motor representation, for motor representation can be inferred from patterns of neuronal discharge, from motor-evoked potentials, from where blood flows in motor areas of the brain, from behavioural performance profiles and in other ways besides.  And we do have evidence that such marks are constant across instances of all three types of action which realise the grasping outcome \citep{rizzolatti:1988_functional, Rizzolatti:2001ug, cattaneo:2010_state-dependent}. So we can infer that some motor representations do not represent merely kinematic features of actions. To infer, further, that some motor representations represent outcomes we need to consider what happens where, conversely, we hold kinematic features constant while varying to which outcome an action is directed. Take an action which realises the grasping of a particular pen and compare it with a second action which is as similar as possible to the first with respect to its kinematic features but differs with respect to which outcome it is directed to because the pen is manifestly absent (so the action is not plausibly directed to grasping anything) or because the pen is manifestly too large or too small to grasp.  Some marks of motor representations distinguish these actions \citep{Umilta:2001zr, villiger:2010_activity, koch:2010_resonance}.  This together with the constancy of some motor representations across variations in kinematic features is good (if not conclusive) evidence that some motor representations represent outcomes to which actions are directed.%
\footnote{
We offer only a brief argument for this conclusion here because \citet{pacherie:2008_action} and \citet{butterfill:2012_intention} also argue for it; the latter further support this conclusion by appeal to the functions of motor processes in planning and monitoring action.
}

***HERE

So far we have been arguing that some motor representations are about outcomes by contrasting this claim with the hypothesis that motor all representations concern merely kinematic features of action.  [*make more vivid!]  This is not quite enough for a convincing argument because some people have claimed that some motor representations in some way concern sensory consequences of actions.
To illustrate, grasping a small ball between your finger and thumb will have predictable tactile and visual consequences, and it is possible (at least in principle) that some motor representations involved in grasping the ball might somehow concern these sensory consequences.
Note that the outcomes to which actions are directed are often distinct from any of the action's sensory consequences.
Take grasping a ball again: although the outcome---the grasping of the ball---reliably has sensory consequences, these are neither necessary nor sufficient for the outcome to occur.  
They are not necessary because the grasping could also be performed with a foot or a tool, which could substantially alter the sensory consequences of the action; and because grasping could also in principle occur in complete darkness with numb (or even deafferented) hands. 
The sensory consequences are also not sufficient because in principle the pattern of movements of finger and thumb in relation to the ball could occur without there being any event directed to the grasping of an object---such movements might in principle occur as part of an exercise in moving one's fingers in which the ball is an unwanted distraction rather than a target.
So our claim that motor representations represent outcomes is conceptually quite distinct from the claim that some motor representations in some way concern sensory consequences of actions: in principle, both claims might be true, or one might be true while the other is false.

Suppose, at least for the sake of argument, that some motor representations do indeed somehow concern sensory consequences of actions. 
Does we still have grounds for holding that there are motor representations of outcomes?  Yes.  We noted just above that it is possible to vary the sensory consequences of an action while holding one of its outcomes constant.  
To return to the example, grasping a ball can be done with a hand, a foot or a tool.
Using different effectors will alter how the action looks (its visual consequences) but not the goal to which it is directed. 
So the fact that observing instances of all three types of action can involve a common motor representation shows that not all motor representations can concern sensory consequences and merely kinematic features only.

***The claim that some motor representations represent outcomes raises two further questions.  First, which outcomes can be represented motorically?  And, second, what (if anything) ensures that an outcome represented motorically in action observation is an outcome to which the observed action is directed?

Actions are typically directed to many outcomes.  For instance, a single action might be directed to grasping the scissors, to cutting a newspaper and to removing a particularly useful article to file later.  In reaching for the scissors the agent is already performing actions directed to all of these outcomes.  Which outcomes can be represented motorically?  Not all outcomes can be represented motorically, of course.  For instance, the difference between cutting out an article and cutting out a picture from a newspaper is not a difference that need be captured by any motor representations.  One idea about which outcomes can be represented motorically starts from the observation that outcomes can be partially ordered by the means-end relation. It might be assumed that motor representations capture only least outcomes relative to the means-end ordering.  This is wrong, however.  Differences such as that between reaching for a pair of scissors to cut something and reaching for a pair of scissors to place them somewhere (perhaps to put them away) are captured by motor representations (\citealp{Fogassi:2005nf, cattaneo:2007_impairment, bonini:2010_ventral}). This suggests that motor representations are not limited to representing least outcomes.  


  
[*So the plan is to end this section after arguing that mr don't represent directedness, and then have a new section break on 'Capturing without representing'.  The Q. for that section might be: What sort of process links the outcome represented to the observed action and ensures that the outcome represented is one to which the action is directed?]


\subsection{Directedness}
Our overall question is how motor representation could enable goal ascription judgements. The fact that motor representations can represent outcomes is a step towards answering this question. However we are not yet in a position to answer it. To see why, consider that someone might (at least in principle) observe an action and represent an outcome without yet relating the outcome to the action. Kate is observing Ludwig kick a ball. Kate has represented this outcome, the ball's being kicked, but she does not represent the outcome as in any way related to Ludwig's action. Rather, she simply finds herself somehow disposed to kick the ball. This is not because she intends to pass the ball back to Ludwig or to compete with him or because of any intention concerning Ludwig: it is just because representing the outcome has suggested to her the possibility of performing this action. So Kate represents the outcome to which Ludwig's action is directed but she does not identify that outcome as a goal to which his action is directed. (She does not relate the outcome to his action at all.)  More would be required for goal ascription. To ascribe a goal to an action involves not only representing the action and the outcome which is the goal but also relating the outcome to the action.

It is important in what follows that we sometimes represent intentions without linking them to actions.  For instance, you might know that a friend intends to break an egg and see the friend handling an egg which breaks without yet knowing whether this breaking was appropriately related to that intention.  To represent an intention and an action does not necessarily involve representing any relation between the action and the intention.

How might the directedness of actions to outcomes be represented or otherwise captured? In philosophy the notion of goal-directedness is standardly explained in terms of intention: for an action to be directed to a outcome is for the action to be appropriately related to an intention whose content specifies that outcome (or, on some views, an intention specifying an appropriately related outcome). Given this, it is perhaps natural to suppose that ascribing a particular goal to a given action might involve representing that action, representing an intention whose content specifies the goal, and representing the action as appropriately related to the intention.  If so, motor representation could have at most a limited role in goal ascription.  After all, no motor representation represents an intention.

The problem is not significantly different if we suppose that goal-directedness can be explained in terms of mental representations other than intentions.  For instance, some have argued that the directedness of an action to an outcome can be explained in terms of motor representation and not only in terms of intention \citep{butterfill:2012_intention}.  Suppose for the sake of argument that this is correct. We still face substantially the same puzzle about how motor representation in goal ascription could capture the directedness of an action to an outcome.  For motor representations do not represent any mental representations at all.  So the directedness of an action to an outcome cannot be represented motorically.

One way to avoid our argument for this conclusion would be to claim that the directedness of an action to a goal can be understood in ways that do not involve intention or indeed any representation.  For instance, an action's being directed to an outcome may consist in its having the function of bringing about that outcome. Here function could be construed teleologically. Teleological accounts of function, and of the application of this notion to understanding goal-directed action, have been extensively developed (*see further Godfrey-Smith 1996; Millikan 1984; Neander 1991; Price 2001; Wright 1976). Suppose we accept for the sake of argument  that goal-directedness can be explained in terms of function rather than intention or any other kind of mental representation.  Substantially the same puzzle remains. After all, motor representations no more represent functions than they represent intentions.

At this point it seems to us plausible that motor representations cannot represent the directedness of an action to an outcome. If we supposed that all goal ascription involved representing directedness, it would be hard to understand the role of motor representation in enabling goal ascription.  But it would be a mistake to suppose that directedness must be represented in goal ascription.  There are ways of capturing directedness which do not involve representing it.  In the rest of this section we shall argue that in goal ascription the link between actions and outcomes to which they are directed can be captured not only by representing it but alternatively by a planning-like process which involves determining how an outcome is to be achieved and whether given movements are appropriate to achieving that outcome.  

To see how the directedness of an action to an outcome can be captured without being represented, let us first step back to consider cases of goal ascription which involve representing intentions only (so motor representation is not yet in the picture). Observing Ayesha gathering some papers during a long meeting that shows no sign of ending, you conjecture that she is intending to leave discretely. You use this conjecture about Ayesha's intention to determine one or more likely sequences of behaviours, almost as if you were planning from her point of view how to act on that intention. This leaves you with expectations concerning how she will act. You might be waiting for Ayesha, who has no other friends present, to give you a nod before quietly pushing away from the table. If what you observe matches these expectations, or if there are only deviations that can be explained in ways consistent with your conjecture about Ayesha's intention, then you may come to regard the conjecture as confirmed. But if there is a mismatch---if, for example, Ayesha signals that she wants attention and looks around to see whether people are looking at her---then you might reject the conjecture about her intention to leave discretely. As this illustrates, goal ascription often (perhaps not always) involves determining how an action will or would unfold, and this process of determination may ensure that there is a reliable connection between the action's being directed to a certain goal and your ascribing that goal to the action. So the directedness of an action to a goal can be captured by a planning-like process in addition to being represented.

You might suggest that where an observer represents another agent's intention, she would only ever determine how the agent will act if she also supposes the actions will be directed to fulfilling this intention.  This may be right.  The illustration is only supposed to show that goal-directedness could in principle be captured without being represented.

Motor representations and processes may likewise capture the directedness of an action to an outcome.  In action observation it is not just that there may be motor representations of outcomes. There is evidence that a motor representation of an outcome sometimes causes a determination of which movements are likely to be performed to achieve that outcome  \citep{kilner:2004_motor, urgesi:2010_simulating}. The processes involved in determining how observed actions are likely to unfold given their outcomes are closely related, or identical, to processes involved in performing actions. This is suggested both by the involvement of regions of the brain associated with action (*refs), and by patterns of facilitation and interference which occur when simultaneously observing and performing actions \citep{brass:2000_compatibility, kilner:2003_interference, craighero:2002_hand}. It is almost as if the observer were planning from the agent's point of view how to act. These planning-like processes sometimes leave observers with expectations about how an action will unfold.  For instance, if you were to observe just the early phases of a grasping movement your eyes may jump to its likely target, ignoring nearby objects \citep{ambrosini:2011_grasping}. These proactive eye movements resemble those you would typically make if you were acting yourself \citep{Flanagan:2003lm}: in reaching for something, it is natural to look at the target rather than your hand.  Importantly, the occurrence of such proactive eye movements in action observation depends both on representing the outcome of an action (even temporary interference in your ability to represent the outcome will interfere with the eye movements,  \citealp{Costantini:2012fk}) and also on planning-like processes (having to perform actions incongruent with those you are observing also interferes with proactive eye movements, \citealp{Costantini:2012uq}).  Finally, going slightly beyond what there is currently direct evidence for, it is also plausible that whether the expectations about how the action will unfold are met or not will influence whether the motor representation of that outcome persists.  This ensures that, where someone is observing an action, there is a reliable connection between the observer representing an outcome and the action being directed to that outcome.  There is a sense, then, in which the directedness of an action to a goal can be captured by motor representations and processes: As might also happen in the case of intention, the directedness of the action to a goal is captured by planning-like processes.  

Let us stipulate that to capture the directedness of an action to an outcome is to (a) represent the action; and to (b) determine how the action is likely to unfold given that it would normally realise that outcome; where (c) this determination leaves the observer with expectations about how the action will unfold; and (d) whether the representation of the outcome persists depends on the expectations being met.  We have been arguing that there is evidence that motor representations and processes can capture the directedness of actions. 

It is perhaps worth pausing here to stress a parallel between performing and observing actions. Start with performance.  In virtue of what is an action related to the outcome or outcomes to which it is directed?  Suppose that in performing an action there is a motor representation of an outcome.  Now an action is not directed to an outcome just in virtue of there being a representation of that outcome, of course.  Rather the outcome has to play some guiding role in determining how the action will unfold.  So what makes it the case that the action is directed to the outcome is not just the content of the representation but also its role in determining how the action is to unfold.  Now switch to action observation.  How might an observer capture the directedness of an action to an outcome?  As in performing an action, there is sometimes a motor representation of an outcome and a process of working out how the action will unfold, as we have seen.  But unlike performing action, in observation the process of working out how the action will unfold does not normally directly cause of the action (of course).  So if expectations about how the action will unfold are not met, it is not normally possible to alter the action; instead it may be necessary to link the action to an outcome different from the one whose representation generated these unmet expectations.  This is how the representations and processes which ground the directedness of actions to outcomes in performing actions may also be involved in capturing the directedness in observing actions.
%
\begin{quote}
[***\textit{A previous, rough attempt at the above paragraph}:  \\
grounding /capturing: In performing an action there is a representation of an outcome.  Now an action is not directed to an outcome just in virtue of there being a representation of that outcome, of course.  Rather the outcome has to play some guiding role in working out how the action will unfold.  So what determines that the action is directed to the outcome is both the representation and the process of working out how the action is to unfold.  Now the case of action observation is similar.  There is a representation of an outcome and a process of working out how the action will unfold.  A difference is that the process of working out how the action will unfold is not (of course) normally a cause of the action.  So the representations and processes which ground the directedness of actions to outcomes in performing actions: these representations and processes also capture the directedness in observing actions.]
\end{quote}

In this section we have considered a first obstacle to understanding how judgements about the goals of actions sometimes depend on motor representations.  The obstacle was that motor representations might appear not to concern the goals of actions at all, making their influence on judgements about goals difficult to understand.  In removing this obstacle we made two points.  The first was that motor representations do not all represent merely kinematic features of action; rather, some motor representations represent outcomes to which actions are directed.  The second point was that although motor representations probably do not represent the directedness of an action to an outcome, reflection on the role of motor representations and processes in action observation shows that this directedness can be captured motorically.  Potentially, then, there is a respect in which the content of a motor representation may resemble the content of judgement about the goal to which an action is directed: they may represent a single outcome and capture the directedness of an action to that outcome.  
We therefore propose that where a judgement about the goal to which an action is directed depends on motor representations, there is a motor representation of an outcome which captures the directedness of the action to the goal.%
\footnote{
***Do we need to allow that the outcomes represented motorically and in judgement may not be identical?
}


\section{Action and Experience}
\label{sec:processes}
Our overall question is how judgements about the goals of actions could ever depend on motor representations.  The previous section concerned the possibility of a match between the contents of motor representations and judgements.  Now to say that two varieties of representations---motor representations and judgements---sometimes have nonaccidentally matching contents is not yet to explain how that match might come about.  Which processes could be responsible for the match?  When judgements about the goals of observed actions depend on the observers' motor representations, which processes connect motor representations to judgements?  

In broad terms, our will be proposal is that the dependence of observational judgements on motor representations goes via experience.  It is because motor representations can affect our experiences when we observe actions that those motor representations can facilitate judgements about these actions.  However this is consistent with two quite different hypotheses concerning the experiences associated with observing action.

One hypothesis is that in observing action we experience not only movements and sounds but also goal-directed actions as such: it is this experience which enables judgements of action, and motor impairments affect judgements of action by preventing one from experiencing actions as such. On the other hypothesis, in observing actions presented as point-light stimuli or acoustically we experience movements and sounds only. Motor representations somehow make these particular experiences of movements and sounds possible, and these experiences in turn ground judgements about actions.  On both hypotheses motor representations enable experiences which are revelatory of action in the sense that they can provide reasons for correct judgements about which action someone is performing.  The difference concerns what these revelatory experiences are experiences of: whether some are experiences of goal-directed actions as such (the first hypothesis) or whether all are experiences of movements and sounds only (the second hypothesis).  


In what follows we first consider evidence that the second hypothesis is plausible and then, in a further section, we consider arguments for the first hypothesis.  Although we shall conclude that the first hypothesis is correct---there are experience of action as such---it is helpful to approach this by way of evidence relevant to the second hypothesis.


\subsection{Motor representation can enable visual and acoustic experiences}
For convenience, in what follows we shall use the term `perceptual experiences' to mean experiences of the visual, acoustic, tactile and other things commonly associated with perception.  As a terminological stipulation (for which there are reasons, but they come later), no experiences of action as such are perceptual experiences.  Later, when we have considered arguments that some experiences are experiences of actions as such, we shall contrast `perceptual experience' with `motor experience'.

The possibility that some motor representations enable visual experiences is suggested by research showing that motor plans, such as might be involved in producing a gesture with the hand, can influence visual processes \citep{bortoletto:2011_action}. The influence of motor representations on visual processes varies depending on both the gesture to be performed and the visual stimulus presented. Importantly, these influences can be detected before any action occurs, showing that it is the motor representations rather than any consequent movements which are influencing visual processes. [*Some other evidence; stress that the effect can be found using different measures.]  As it may be possible to influence visual processes in ways that make no difference to experience, we do not take these findings alone to establish the claim that motor representations could enable visual experiences.  But these findings do at least show that the claim is consistent with the neuropsychological facts.% 
\footnote{
For a recent review of some of the evidence concerning motor influences on perceptual processing, see \citet{halasz:2012_unconscious}. (*refs)
}  

What would establish that some motor representations enable perceptual experiences? For some very special pairs of tones, the first is sometimes perceived as lower in pitch than the second and sometimes as higher in pitch.  Repp and Knoblich (\citeyear{repp:2007_action}) had a group of expert pianists and a control group of non-experts press two keys in sequence, where the first key was sometimes left and sometimes right of the second key.  They found that, for the expert pianists, the direction of the key presses modulated the experience: it influenced the perceived direction of the change in pitch. Could what modulates experience in this case be not a motor representation but the occurrence of a movement, or perhaps even the perception of a movement of the subject's own fingers?  Against this, note that the effect was not observed in non-expert pianists: for them the direction of action did not measurably influence the experienced pitch direction. Since both groups moved in the same way, if the modulation were due to movement only we would expect it to occur irrespective of expertise. Instead it seems likely that differences in expertise between the two groups of subjects affected how the movements they performed were represented motorically, and these differences in motor representation are in turn what explains their experience of the change in pitch.  This study, and others like it,%
\footnote{
See also \citet{zwickel:2010_interference} who investigate effects of action on visual experience of motion.  *more
} 
provide relatively direct evidence that motor representations can enable acoustic experiences.

How do these findings bear on the second hypothesis above, according to which motor representation in action observation enables experiences of sounds and movements that somehow facilitate the identification of actions?  Perhaps the motor representations enable certain experiences of kinematic features which somehow enable subjects to better identify actions.  We do not claim that this is entirely plausible, only that the second hypothesis cannot be rejected out of hand.  This is enough to show that we cannot infer that there are experiences of action just from the fact that motor representation enables experiences which are revelatory of action.  Stronger evidence for the first hypothesis is needed.

\subsection{Experiences of action}
Our focus is experiences involved in observing actions, but let us first consider experiences involved in performing actions.  We shall argue that acting sometimes involves experiences which can ground judgements about what one has done even though these are not experiences of movements or sounds.  In effect, the idea will be that you can strip away perceptual experiences and still be left with experiences revelatory of action.  These can only be experiences of action, or so we claim.  Two phenomena are involved in making this argument.  The first is anosognosia for hemiplegia, which shows that there can be experiences revelatory of action even in the absence of any actual movements.  The second is deafference, which confirms that these experiences are unlikely to be illusory experiences of movement. Together these phenomena support the view that, in performing actions, we have experiences of actions and not only of movements, sounds and the like.  

The argument sketched so far concerns experiences associated with performing actions, not observing them.  Note, however, that the basis of these experiences of action is motor representation and planning.  Since these are the very things which, in action observation, underlie the experiences enabling one to identify actions, it is reasonable to conclude that there is a phenomenal element common to performing an action and observing an action performed, namely the action itself.  Roughly, what you experience in performing an action is what you would experience if you were observing it.  But this is all to come; let us start by introducing some background. 

Anosognosia for hemiplegia is a clinical condition in which patients deny, and appear largely unaware of, a severe paralysis of one or more limbs on one side of their body.  It typically follows a stroke involving damage to motor areas in the right hemisphere of the brain (*refs).  To illustrate the effects of this condition, consider a subject with anosognosia for hemiplegia who was asked to brush her hair holding a brush in her contralesional hand.  Although she was unable to move the hand, she proceeded to move her head as if her hair were being brushed, and then reported having successfully brushed her hair \citep{berti:2008_motor}. This form of anosognosia cannot be explained as entirely due to neglect or complete ignorance of part of one's body \citep[p.\ 165]{berti:2008_motor}.  

Could anosognosia for hemiplegia be explained as mere  confabulation or a top-down effect of judgement on experience?  This is incompatible with the performance of anosognosics on bimanual tasks \citep{berti:2008_motor, garbarini:2012_moving}.  To illustrate, consider an anosognosic who has been asked to wash with both hands.  Although she can only actually move one hand, she may move this hand in much the way that she would do if she were actually washing with both hands (*ref: personal communication?).  This is hard for ordinary subjects to reproduce (try it). For a more carefully controlled illustration, consider anosognosic subjects who were asked to draw simultaneously with both hands, one hand was supposed to continuously draw a vertical line and the other hand to continuously draws a circle.  Anosognosic subjects can of course only actually move one hand, but they show interference as if both hands were actually being moved.  That is, if you were to perform this task, your vertical line would become somewhat oval; this interference is an effect of bimanual coordination.  The striking finding is that anosognosic subjects show the same pattern of interference although not actually drawing a circle.  By contrast, subjects with motor neglect and subjects with hemiplegia but no anosognosia do not show this pattern of interference \citep{garbarini:2012_moving}.  This suggests that anosognosia involves relatively detailed action-related experiences and cannot be entirely explained as confabulation.  

On the leading, best supported explanation, anosognosia for hemiplegia arises from defects in monitoring action.  When a subject with anosognosia for hemiplegia is asked to perform an action involving a hemiparetic limb, motor planning occurs as it might do in ordinary subjects. Now in ordinary subjects, and in hemiplegic subjects without anosognosia, the predicted effects of a planned action are compared with motor representations of action outcomes.  Consequently if such subjects tried and failed to initiate an action, they would normally be aware of their failure even without seeing or touching their limbs. However, in anosognosic subjects the comparator is damaged and no signal of failure is generated.  This is why it seems to them that they have acted despite failing even to initiate the action.  Of course, anosognosic subjects can perceive that their objectives have not been met: after apparently opening a bottle, for instance, they can see that the bottle remains shut; they would not normally proceed to drink from it.  And these subjects do not necessarily have visual, proprioceptive or tactile sensory impairments.  But the experience associated with motor planning dominates, so that it seems to them as if they have acted.  From their point of view, it seems that the bottle is peculiarly difficult to open rather than that they are unable to move a limb \citep[pp.\ 173-4]{berti:2008_motor}.  

How is anosognosia for hemiplegia relevant to the issue of whether there are experiences of action as such?  It seems that anosognosic subjects have some experiences which are revelatory of action---revelatory in the sense that they provide reasons for judgements about actions---despite not actually moving a limb.  What could these experiences be experiences of?  Consider the claim that these are experiences of movement only.  While it is arguably not generally correct to infer an absence of experience of movement from the absence of actual movement, in this particular case it does seems unlikely that there could be experience of movement.  After all, some anosognosic subjects have intact perceptual systems: if they experience movement at all, they should experience an absence of movement.  So any experiences of movement will not be the sort of experiences that would provide reasons for judgements about successful action.  Given that (as argued above) the experiences on which anosognosic subjects' judgements of action are based are also not entirely a product of confabulation, it is plausible to accept that they are experiences of action as such.

This view is further supported by research on deafferented subjects.  These subjects lack tactile and proprioceptive sensory feedback from their actions but, unlike subjects with anosognosia for hemiplegia, can actually move their limbs.  Such subjects may continue to have some experiences characteristic of action; for instance, when asked to flex an index finger they can report having done so even in the absence of visual information \citep{kristeva:2006}.  These subjects' sensory deficits make it unlikely that their experiences could be of movements or the like: it is more plausible that, concerning acting, their experiences are like those of subjects with anosognosia for hemiplegia.  None experience movement: what they experience is action.

Anosognosia and deafference comprise a useful pair of cases: some anosognosics we have preserved perceptual senses and an absence of actual bodily movement, whereas individuals suffering deafference have damaged senses and actual bodily movement.  But in both cases there are experiences revelatory of action.  The existence of these disorders indicates that motor representation is sufficient for experiences capable of grounding judgements about actions even in the absence of any perceptual experiences of movements, sounds or the like.  It is plausible to infer that these experiences are experiences of action as such.

***HERE: todo.  Two things.  First, explain the timing experiment---it's not all about subjects with brain damage, there is converging evidence from ordinary subjects.  Second, explain the comparitor model.  

***Anosognosia for hemiplegia : experience of action without acting (cannot be an experience of movement, \&c).  Or could this be an illusory experience of one's body moving?  \citep{garbarini:2012_moving}%
\footnote{
See also \citet{gallese:2010_bodily}: "Brugger, Kollias, Muri, Crelier, and Hepp-Reymond (2000) described one of these cases. fMRI imaging of their patient during phantom limb sensations of hand movements showed activation of premotor and posterior parietal cortex. Aplasic phantoms can therefore be explained as the phenomenal correlate of planning/monitoring action of an absent limb."}
and hand-washing: \citep{berti:2008_motor}

***parallel with speech (short)



\section{Conclusion}
Motor representation enables judgements about the goals of actions by making possible experiences of action as such.  

\subsection{Reasons}
In making judgements about the goals of actions, it appears (perhaps veridically) to the thinker that she has reasons for those judgements, much as it appears to her that she has when making visual judgements about the particular colours of objects.  Or, if we are sceptical about thinkers having reasons for visual judgements, at least we can say that it does not seem to the thinker that she is simply landed with the judgement.  Contrast someone who `knew at a glance' that a vase is genuine Ding ware, or someone on a blind date who, on first catching sight of her partner for the night, `just knows' that this is the one she wants to spend the rest of her life with.  Action judgements often seem different from these: rather than simply being landed with them, it may seem to a thinker that they bear a straightforward relation to her experience. 

Our view is consistent with the claim that it appears to a thinker that she has reasons for making judgements about the goals of actions even when  those judgements depend on motor representations.
  Motor representation stands to experience of action as visual representation stands to experience of particular colours. 


\subsection{Goal ascription without metarepresentation}
Having noted above that goal ascription often or always involves representing the directedness of an action to an outcome, we argued that this is something no motor representation can represent. Equally, however, we noted that the directedness of actions to outcomes can be captured by planning-like processes, and we suggested that motor representations can feature in such processes.  One possibility, then, is that there might be a kind of goal ascription---or something resembling goal ascription---where the only representations involved are motor representations.  This enables us to see how there could be goal ascription, or something very like it (we are not insisting that the term 'goal ascription' be used) could occur without metarepresentation. This gives us a first role for motor representation in goal ascription which distinguishes motor representation from representations of intentions or other mental states.  Where goal ascription involves capturing the directedness of actions to goals by representations of intentions it thereby involves metarepresentation; but where the only representations involved in goal ascription are motor representations, there is no metarepresentation at all.











\section{* PLAN *}
[Corrado's Challenge: Why is this not just about implementation details?]
\begin{enumerate}
\item What roles, if any, might motor representation play in goal ascription?
\item Some motor representations occur in action observation and sometimes facilitate goal ascription (Serino; basketball; \& others?) (otherwise the question would be bizarre).  This is what makes the question pressing (timely).
\item The Question: How is this possible?  How could motor representation facilitate goal ascription?
\item Philosophical motivation: A familiar view in philosophy has it that goal ascription depends on the ascription of intentions, which in turn is interdependent with ascriptions of belief and desire (*refs). On this view goal ascription is possible only as part of a larger mindreading project. By contrast, developmental psychologists have assumed that goal ascription is possible even without knowing anything of an agent's mental states (*refs).  Our concern is with this pure form of goal ascription, goal ascription that occurs independently of knowledge of mental states.
\item There are two problems.  (a) adequacy of motor representation; (b) link between motor representation and judgement about the action and its goal
\item Problem (a).  Motor representations may represent outcomes but not the directedness of actions to outcomes?
\item Solution to problem (a): Capturing vs. representing the directedness of actions.
\item Problem (b): If we thought of motor representation as cut off from experience, it would seem impossible that differences in motor representation could make a difference to the reasons available from the observer's point of view.  
\item However, there is evidence that motor representation influences experience (e.g. acoustic experience of pitch direction).
\item One possibility: motor representation influences visual and acoustic experiences of movements and sounds, and these somehow facilitate goal ascription
\item This does not seem to be a fully adequate explanation because ?
\item Furthermore, motor representation enables experience which are (i) revelatory of action and (ii) distinct from visual or auditory experiences (Berti \&c).
\item So maybe: motor representation enables experiences of action as such
\item Conclusion: motor representation enables goal ascription by making possible experiences of action as such.  The role of motor representation in goal ascription is like the role of visual representation in judgements of motion.
\end{enumerate}

MR----> | acquisition| Action Judgment (action concept) <----> B \& D Judgment (b\&d concept)
tracking goal -----> || ascribing goal <----------------- > ascribing beliefs and desires

MR<----> Action Judgment (action concept) <----> B \& D Judgment (b\&d concept)








\bibliography{$HOME/endnote/phd_biblio}

\end{document}