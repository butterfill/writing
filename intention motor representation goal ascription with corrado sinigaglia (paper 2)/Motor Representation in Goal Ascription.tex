%!TEX TS-program = xelatex
%!TEX encoding = UTF-8 Unicode

%\def \papersize {a4paper}
\def \papersize {letterpaper}

\documentclass[12pt,\papersize]{extarticle}
% extarticle is like article but can handle 8pt, 9pt, 10pt, 11pt, 12pt, 14pt, 17pt, and 20pt text

\def \ititle {Motor representation in goal ascription}
\def \isubtitle {}
\def \iauthor {}
\def \iauthor {C. Sinigaglia* \& S. Butterfill**
\\ 
**Dipartimento di Filosofia, Università degli Studi di Milano, Italia
\\ 
***Department of Philosophy, University of Warwick, UK}
\def \iemail{corrado.sinigaglia@unimi.it}
%\date{}

%!TEX TS-program = xelatex
%!TEX encoding = UTF-8 Unicode

\title{\ititle\\\isubtitle}
\author{\iauthor\\<{\iemail}>}

\usepackage[\papersize]{geometry} % see geometry.pdf
\geometry{twoside=false}
\geometry{headsep=2em} %keep running header away from text
\geometry{footskip=1cm} %keep page numbers away from text
\geometry{top=3cm} %increase to 3.5 if use header
\geometry{left=4cm} %increase to 3.5 if use header
\geometry{right=4cm} %increase to 3.5 if use header
\geometry{textheight=22cm}

%non-xelatex
%\usepackage[T1]{fontenc}
%\usepackage{tgpagella}

%for underline
\usepackage[normalem]{ulem}

%get the font here:
% http://scripts.sil.org/CharisSILfont

\usepackage{fontspec,xunicode}
%nb do not explicitly use package xltxtra because this introduces bugs with footnote superscripting  -- perhaps because fontspec is supposed to include it anyway.
%UPDATE:  "You need to use the no-sscript option in xltxtra: \usepackage[no-sscript]{xltxtra}, this is explained in the documentation of xltxtra.  The issue is that Sabon does not contain true superscript glyphs for every character and the no-sscript option will instead use scaled regular glyphs, which is typographically inferior, but there is no other option available when using Sabon." --- http://groups.google.com/group/comp.text.tex/browse_thread/thread/19de95be2daacade
\defaultfontfeatures{Mapping=tex-text}
%\setromanfont[Mapping=tex-text]{Charis SIL} %i.e. palatino
%\setromanfont[Mapping=tex-text]{Sabon LT Std} 
%\setromanfont[Mapping=tex-text]{Dante MT Std} 
%\setromanfont[Mapping=tex-text,Ligatures={Common}]{Hoefler Text} %comes with osx
\setromanfont[Mapping=tex-text]{Linux Libertine O} 
\setsansfont[Mapping=tex-text]{Linux Biolinum O} 
\setmonofont[Scale=MatchLowercase]{Andale Mono}


%hyperlinks and pdf metadata
%TODO avoid duplication of title & author
\usepackage{hyperref}
\hypersetup{pdfborder={0 0 0}}
\hypersetup{pdfauthor={\iauthor}}
\hypersetup{pdftitle={\ititle\isubtitle}}


%handles references to labels (e.g. sections) nicely
\usepackage{varioref}

%line spacing
\usepackage{setspace}
%\onehalfspacing
%\doublespacing
\singlespacing

\usepackage{natbib}
%\usepackage[longnamesfirst]{natbib}
\setcitestyle{aysep={}}  %philosophy style: no comma between author & year

%enable notes in right margin, defaults to ugly orange boxes TODO fix
%\usepackage[textwidth=5cm]{todonotes}

%for comments
\usepackage{verbatim}

%footnotes
\usepackage[hang]{footmisc}
\setlength{\footnotemargin}{1em}
\setlength{\footnotesep}{1em}
\footnotesep 2em

%tables
\usepackage{booktabs}
\usepackage{ctable}

%section headings
\usepackage[sf]{titlesec}
%\titlespacing*{\section}{0pt}{*3}{*0.5} %reduce vertical space after header
%large headings:
%\titleformat{\section}{\LARGE\sffamily}{\thesection.}{1em}{} 
\titlelabel{\thetitle.\quad}

%captions
\usepackage[font={small,sf}, margin=0.75cm]{caption}

%lists
\usepackage{enumitem}
\newenvironment{idescription}
{ 	
	% begin code
	\begin{description}[
		labelindent=1.5\parindent,
		leftmargin=2.5\parindent
	]
}
{ 
	%end code
	\end{description}
}


%title
\usepackage{titling}
\pretitle{
	\begin{center}
	\sffamily
	\Huge
} 
\posttitle{
	\par
	\end{center}
	\vskip 0.5em
} 
\preauthor{
	\begin{center}
	\normalsize
	\lineskip 0.5em
	\begin{tabular}[t]{c}
} 
\postauthor{
	\end{tabular}
	\par
	\end{center}
}
\predate{
	\begin{center}
	\normalsize
} 
\postdate{
	\par
	\end{center}
}


%\author{}
%\author {S. Butterfill* \& C. Sinigaglia**
%\\ 
%**Department of Philosophy, University of Warwick, UK
%\\ 
%***Dipartimento di Filosofia, Università degli Studi di Milano, Italia}
%\date{}


\begin{document}

\setlength\footnotesep{1em}

\bibliographystyle{newapa} %apalike

\maketitle
%\tableofcontents
\title{}





\begin{abstract}
\noindent
%
How could judgements about the goals of actions depend on motor representations? Many findings show that they do, but several obstacles ... Overcome obstacles by showing that motor representations support experiences of action in something like the ways in which visual representations support experiences of objects ... Implications for mindreading.
%
\end{abstract}

\tolerance=5000









\section{Introduction}

Motor representation sometimes occurs in action observation and enables judgements about the goals to which actions are directed (see \citealp{rizzolatti_functional_2010} for a review). For instance, suppose you observe someone's hand moving towards a pair of scissors. You may judge that she will shortly perform an action directed to cutting something up. Making this judgement may depend on representing the other's actions in motor terms \citep{boria:2009_intention, ortigue:2010_understanding}.  This is not to say, of course, that you could not make a judgement of this type without representing the action in motor terms, only that making this particular judgement may depend on motor representation. How is this possible? How could any judgements about the goals of actions ever depend on motor representations?

Two obstacles stand in the way of answering this question. The first is an apparent mismatch between what goal ascription requires and what motor representations can capture. Goal ascription requires capturing not only the outcome to which an action is directed but also the directedness of the action to that outcome. Where goal ascription involves representing intentions, it is clear enough how both can be captured. But it is not obvious how both can be captured with motor representation alone. To understand how goal ascription judgements could depend on motor representations we need to understand how the contents of the motor representations are related to the contents of the judgements.

The second obstacle is the apparent isolation of motor representation from thought. It is often assumed that motor processes and representations are entirely cut off from experience and so could not have any rational effect on judgement. But ascribing goals does not seem from the inside to be like gazing into a crystal ball and simply being struck with an idea. From a thinker's point of view, experience reveals not only movements of objects but also the goals of actions. To understand how goal ascription judgements could depend on motor representations we need to understand how the role of motor representations in enabling judgements is compatible with it appearing (perhaps veridically) to the thinker that she has reasons for those judgements.

In this paper we investigate ways of overcoming both obstacles. The first obstacle concerns how the contents of the motor representations are related to the contents of the judgements. On this we shall identify evidence for two claims. First, some motor representations represent not merely kinematic or dynamic features of actions but outcomes to which actions are directed. Second, although the directedness of an action to an outcome cannot be represented motorically, it can be captured by motor processes which determine how an action should unfold in order for the outcome to be realised. So the status of a particular motor representation as a representation of a goal to which an action is directed depends both on its content---on its being a representation of an outcome---and on its role in motor processes which determine how that action should unfold. This removes one obstacle to understanding how motor representations could enable judgements about the goals of actions.  But in removing this first obstacle we run straight into a second.  This second obstacle concerns how it could appear to the thinker that she has reasons for judgements when the judgements are enabled by motor representations. To remove this obstacle we shall argue that motor representations are related to experiences.  There is a parallel between visual and motor representations. Visual representations of the particular colours of objects are related to experiences of those objects and their colours, and these experiences arguably provide reasons for judgements about the colours of objects.%
\footnote{
*Some controversy.  Campbell: yes; Davidson: no (experience plays a merely causal role).  We simply assume yes.  (Or we can duck: we're talking about appearance to the thinker of having reasons, not the fact of having reasons.)}
So the fact that visual representations enable judgements about particular colours in no way conflicts with the claim that thinkers have reasons for those judgements.  We shall argue that motor representations of outcomes are related to experiences of actions and their directedness to outcomes.  This is why thinkers can have reasons for judgements about the directedness of an action to a goal even where the judgements are enabled by motor representations---and also why it appears (veridically) to thinkers that they have such reasons.  Motor representation stands to experience of action as visual representation stands to experience of particular colours. This anyway is the view we shall defend.

This view has implications for understanding how goal ascription relates to mindreading. A familiar view in philosophy has it that goal ascription depends on the ascription of intentions, which in turn is interdependent with ascriptions of belief and desire (*refs). On this view goal ascription is possible only as part of a larger mindreading project. By contrast we shall show that in some cases the only representations involved in goal ascription are motor representations. So goal ascription can occur independently of any knowledge of mental states. It is therefore coherent to conjecture that facts about the goals of actions form part of the evidential basis for mental state ascription. Knowledge of others' minds ultimately depends on knowledge of their actions, which in turn depends on experiences made possible by motor representations.  Mindreading is grounded in motor representation.


\section{How to capture the directedness of actions to outcomes}

Our project is to understand how judgements about the goals of actions  could depend on motor representations.  One obstacle to understanding this is an apparent mismatch between what goal ascription requires and what motor representations can capture. To illustrate, let us return to our opening example. Observing someone's hand moving towards a pair of scissors you judge that she will shortly perform an action directed to cutting something.  In making this judgement you are representing an outcome and linking it to an action. How could motor representation enable you to make such a judgement?

If all motor representations were about merely kinematic or dynamic features of action only, their role in facilitating judgements about outcomes would be mysterious. However some motor representations represent outcomes to which actions are directed, as we shall now argue.  For instance, observing someone's hand moving towards a pair of scissors can involve a motor representation of an outcome such as the cutting of something and not only motor representations of individual joint displacements that might bring about this outcome.  Motor representations can facilitate goal ascriptions in part because some motor representations represent the outcomes which are the goals to be ascribed.  

But how can we distinguish a representation of an outcome from a representation of merely kinematic features of actions?  This is possible because a single outcome can be realised by actions with quite different kinematic features and, conversely, actions with arbitrarily similar kinematic features can realise different outcomes in different contexts.  To illustrate, consider this outcome: the grasping of a particular pen. This could be realised by any of three types of action which vary kinematically: a hand action, a foot action or a tool-using action.  If we had marks that we could use to identify motor representations without already knowing their contents, and if we had evidence that these marks were constant across all three types of action, then we could infer that some motor representations capture something more abstract than the kinematic features particular to the hand, foot or tool-using action. But we do have well established marks of motor representation, for motor representation can be inferred from patterns of neuronal discharge, from motor-evoked potentials, from where blood flows in the brain, from behavioural performance profiles and in other ways.  And we do have evidence that such marks are constant across all three actions which realise the grasping outcome \citep{rizzolatti:1988_functional, Rizzolatti:2001ug, cattaneo:2010_state-dependent}. So we can infer that some motor representations do not represent merely kinematic features of actions. To infer, further, that some motor representations represent outcomes we need to consider what happens where, conversely, we hold kinematic features constant while varying to which outcome an action is directed. Take an action which realises the grasping of a particular pen and compare it with a second action which is as similar as possible to the first with respect to its kinematic features but differs with respect to which outcome it is directed to because the pen is manifestly absent (so the action is not plausibly directed to grasping that pen) or manifestly too large or too small to grasp.  Some marks of motor representations distinguish these actions \citep{Umilta:2001zr, villiger:2010_activity, koch:2010_resonance}.  This together with the constancy of some motor representations across variations in kinematic features is good (if not conclusive) evidence that some motor representations represent outcomes to which actions are directed.%
\footnote{
\citet{pacherie:2008_action} and \citet{butterfill:2012_intention} also argue for this conclusion; the latter further supported this claim by appeal to the functions of motor processes in planning and monitoring action.
}

Actions are typically directed to many outcomes.  For instance, an action might be directed to grasping the scissors, to cutting a newspaper and to removing a particularly useful article to file later.  In reaching for the scissors the agent is already performing actions directed to all of these outcomes.  Which outcomes can be represented motorically?  Outcomes can be partially ordered by the means-end relation. It might be assumed that motor representations capture only least outcomes relative to the means-end ordering. It is certainly true that not all outcomes can be represented motorically.  For instance, the difference between cutting out an article and cutting out a picture from newspapers is not as such a difference that need be captured by any motor representations.  But we should not infer from this that motor representations are limited to representing least (in the means-end ordering) outcomes. In fact, differences such as that between reaching for a pair of scissors to cut something and reaching for a pair of scissors to place them somewhere (perhaps to put them away) are captured by motor representations (\citealp{Fogassi:2005nf, cattaneo:2007_impairment, bonini:2010_ventral, }). This suggests that motor representations are not limited to representing least outcomes.

Our overall question is how motor representation could enable goal ascription judgements. The fact that motor representations can represent outcomes is a step towards answering this question. However we are not yet in a position to answer the question. To see why, consider that someone might (at least in principle) observe an action and represent an outcome without yet relating the outcome to the action. Kate is observing Ludwig kick a ball. Kate has represented this outcome, the ball's being kicked, but she does not represent the outcome as in any way related to Ludwig's action. Rather, she simply finds herself somehow disposed to kick the ball. This is not because she intends to pass the ball back to Ludwig or to compete with him or because of any other intention: it is just because representing the outcome has suggested to her the possibility of performing this action. So Kate represents the outcome to which Ludwig's action is directed but she does not identify that outcome as the goal of his action. More would be required for goal ascription. To ascribe a goal to an action is to represent the action and the goal and also to relate the goal to the action.

But how might the directedness of actions to outcomes be represented or otherwise captured? In philosophy the notion of goal-directedness is standardly explained in terms of intention: for an action to be directed to a goal is for the action to be appropriately related to an intention whose content specifies that goal (or, on some views, an intention specifying an appropriately related goal). Given this, it is perhaps natural to suppose that ascribing a particular goal to a given action might involve representing that action, representing an intention whose content specifies the goal, and representing the action as appropriately related to the intention.  If so, motor representation could have at most a limited role in goal ascription.  After all, no motor representation represents an intention.


How can you get from motor representations to the directedness of actions to outcomes?

If motor representations were representations of kinematic or dynamic features only, the claim that they enable goal ascription would be mysterious.  In fact, however, motor representations represent outcomes: or so we shall argue.  This is a first step towards 

How could motor representation represent the outcome to which an action is directed?

To understand how motor representation could enable goal ascription we need to understand the relation between motor representations and outcomes and the relation between






it is perhaps natural to suppose that ascribing a particular goal to a given action might involve representing the action, representing an intention whose content specifies the goal, and representing the action as appropriately related to the intention.






\section{* PLAN *}
[Corrado's Challenge: Why is this not just about implementation details?]
\begin{enumerate}
\item What roles, if any, might motor representation play in goal ascription?
\item Some motor representations occur in action observation and sometimes facilitate goal ascription (Serino; basketball; \& others?) (otherwise the question would be bizarre).  This is what makes the question pressing (timely).
\item The Question: How is this possible?  How could motor representation facilitate goal ascription?
\item Philosophical motivation: A familiar view in philosophy has it that goal ascription depends on the ascription of intentions, which in turn is interdependent with ascriptions of belief and desire (*refs). On this view goal ascription is possible only as part of a larger mindreading project. By contrast, developmental psychologists have assumed that goal ascription is possible even without knowing anything of an agent's mental states (*refs).  Our concern is with this pure form of goal ascription, goal ascription that occurs independently of knowledge of mental states.
\item There are two problems.  (a) adequacy of motor representation; (b) link between motor representation and judgement about the action and its goal
\item Problem (a).  Motor representations may represent outcomes but not the directedness of actions to outcomes?
\item Solution to problem (a): Capturing vs. representing the directedness of actions.
\item Problem (b): If we thought of motor representation as cut off from experience, it would seem impossible that differences in motor representation could make a difference to the reasons available from the observer's point of view.  
\item However, there is evidence that motor representation influences experience (e.g. acoustic experience of pitch direction).
\item One possibility: motor representation influences visual and acoustic experiences of movements and sounds, and these somehow facilitate goal ascription
\item This does not seem to be a fully adequate explanation because ?
\item Furthermore, motor representation enables experience which are (i) revelatory of action and (ii) distinct from visual or auditory experiences (Berti \&c).
\item So maybe: motor representation enables experiences of action as such
\item Conclusion: motor representation enables goal ascription by making possible experiences of action as such.  The role of motor representation in goal ascription is like the role of visual representation in judgements of motion.
\end{enumerate}

MR----> | acquisition| Action Judgment (action concept) <----> B \& D Judgment (b\&d concept)
tracking goal -----> || ascribing goal <----------------- > ascribing beliefs and desires

MR<----> Action Judgment (action concept) <----> B \& D Judgment (b\&d concept)




\bibliography{$HOME/endnote/phd_biblio}

\end{document}