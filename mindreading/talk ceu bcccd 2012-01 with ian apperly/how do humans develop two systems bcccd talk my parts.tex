%!TEX TS-program = xelatex
%!TEX encoding = UTF-8 Unicode

\def \papersize {a4paper}
\documentclass[14pt,\papersize]{extarticle}
% extarticle is like article but can handle 8pt, 9pt, 10pt, 11pt, 12pt, 14pt, 17pt, and 20pt text

\def \ititle {How do humans develop two systems for mindreading?}
\def \isubtitle {}
\def \iauthor {Ian Apperly \& Stephen A. Butterfill}
\def \iemail{s.butterfill@warwick.ac.uk}
%\def \iauthor {}
%\def \iemail{}
\date{}


%!TEX TS-program = xelatex
%!TEX encoding = UTF-8 Unicode

\title{\ititle\\\isubtitle}
\author{\iauthor\\<{\iemail}>}

\usepackage[\papersize]{geometry} % see geometry.pdf
\geometry{twoside=false}
\geometry{headsep=2em} %keep running header away from text
\geometry{footskip=1cm} %keep page numbers away from text
\geometry{top=3cm} %increase to 3.5 if use header
\geometry{left=4cm} %increase to 3.5 if use header
\geometry{right=4cm} %increase to 3.5 if use header
\geometry{textheight=22cm}

%non-xelatex
%\usepackage[T1]{fontenc}
%\usepackage{tgpagella}

%for underline
\usepackage[normalem]{ulem}

%get the font here:
% http://scripts.sil.org/CharisSILfont

\usepackage{fontspec,xunicode}
%nb do not explicitly use package xltxtra because this introduces bugs with footnote superscripting  -- perhaps because fontspec is supposed to include it anyway.
%UPDATE:  "You need to use the no-sscript option in xltxtra: \usepackage[no-sscript]{xltxtra}, this is explained in the documentation of xltxtra.  The issue is that Sabon does not contain true superscript glyphs for every character and the no-sscript option will instead use scaled regular glyphs, which is typographically inferior, but there is no other option available when using Sabon." --- http://groups.google.com/group/comp.text.tex/browse_thread/thread/19de95be2daacade
\defaultfontfeatures{Mapping=tex-text}
%\setromanfont[Mapping=tex-text]{Charis SIL} %i.e. palatino
%\setromanfont[Mapping=tex-text]{Sabon LT Std} 
%\setromanfont[Mapping=tex-text]{Dante MT Std} 
%\setromanfont[Mapping=tex-text,Ligatures={Common}]{Hoefler Text} %comes with osx
\setromanfont[Mapping=tex-text]{Linux Libertine O} 
\setsansfont[Mapping=tex-text]{Linux Biolinum O} 
\setmonofont[Scale=MatchLowercase]{Andale Mono}


%hyperlinks and pdf metadata
%TODO avoid duplication of title & author
\usepackage{hyperref}
\hypersetup{pdfborder={0 0 0}}
\hypersetup{pdfauthor={\iauthor}}
\hypersetup{pdftitle={\ititle\isubtitle}}


%handles references to labels (e.g. sections) nicely
\usepackage{varioref}

%line spacing
\usepackage{setspace}
%\onehalfspacing
%\doublespacing
\singlespacing

\usepackage{natbib}
%\usepackage[longnamesfirst]{natbib}
\setcitestyle{aysep={}}  %philosophy style: no comma between author & year

%enable notes in right margin, defaults to ugly orange boxes TODO fix
%\usepackage[textwidth=5cm]{todonotes}

%for comments
\usepackage{verbatim}

%footnotes
\usepackage[hang]{footmisc}
\setlength{\footnotemargin}{1em}
\setlength{\footnotesep}{1em}
\footnotesep 2em

%tables
\usepackage{booktabs}
\usepackage{ctable}

%section headings
\usepackage[sf]{titlesec}
%\titlespacing*{\section}{0pt}{*3}{*0.5} %reduce vertical space after header
%large headings:
%\titleformat{\section}{\LARGE\sffamily}{\thesection.}{1em}{} 
\titlelabel{\thetitle.\quad}

%captions
\usepackage[font={small,sf}, margin=0.75cm]{caption}

%lists
\usepackage{enumitem}
\newenvironment{idescription}
{ 	
	% begin code
	\begin{description}[
		labelindent=1.5\parindent,
		leftmargin=2.5\parindent
	]
}
{ 
	%end code
	\end{description}
}


%title
\usepackage{titling}
\pretitle{
	\begin{center}
	\sffamily
	\Huge
} 
\posttitle{
	\par
	\end{center}
	\vskip 0.5em
} 
\preauthor{
	\begin{center}
	\normalsize
	\lineskip 0.5em
	\begin{tabular}[t]{c}
} 
\postauthor{
	\end{tabular}
	\par
	\end{center}
}
\predate{
	\begin{center}
	\normalsize
} 
\postdate{
	\par
	\end{center}
}


%\author{}

%\setromanfont[Mapping=tex-text]{Sabon LT Std} 

\begin{document}

\setlength\footnotesep{1em}

\bibliographystyle{newapa} %apalike

\maketitle
%\tableofcontents

\begin{abstract}
Experimental evidence, and reflection on everyday examples suggest that adults’ mindreading is both flexible and sophisticated (as when judging someone’s guilt in a court of law) and fast and efficient (as when taking part in a competitive sport). However, conceptual analysis, and evidence from analogous cognitive domains, suggests that a single cognitive process for “theory of mind” is unlikely to be able to meet these diverse demands. This leads to the proposition that adults have “two systems” for mindreading that make different processing trade-offs between flexibility and efficiency (Apperly \& Butterfill, 2009).

How are adults’ two systems related to the mindreading abilities observed in infants? One possibility is that the infant system grows up: although simple and cognitively efficient at the outset the infant system becomes increasingly sophisticated as children gain conceptual, linguistic and executive resources. On this view, the cognitively efficient abilities of adults must have some other origin, perhaps in automatisation of abilities that were previously effortful. A second possibility is that the infant system persists: the system that explains efficient mindreading in infants also explains efficient mindreading in adults. On this view the flexible and sophisticated abilities of adults must have some other origin, perhaps in the protracted developments charted in traditional studies of children’s “theory of mind”. We will draw inspiration from Susan Carey’s work on “signature limits” in other aspects of core cognition to identify evidence that might distinguish between these accounts.


\end{abstract}


\section{Introduction}
How do humans develop two systems for mindreading?
Our question presupposes that human adults (at least) do have two systems for mindreading.
Actually this is something we want to argue for rather than assume. 
So let me start by making a distinction and an observation ...

\section{slide}
The distinction is between processes which are automatic 
and those which are not.

Roughly speaking,
a process is *automatic* if it occurs irrespective of its relevance to the particulars of an agent's motives and aims.  

\section{slide}
The puzzle starts with a question.
Is adult mindreading automatic or not?

\section{slide}
Some evidence suggests that it is.
For example,
Kovacs and colleagues (2010) asked adults to identify the location of a ball.
They found that adults were significantly slower to identify the ball's location when an onlooker had a false belief about the location of the ball,
even though the onlooker's belief was not relevant to the task at all.
This and related findings seem to suggest that, at least in adults, mindreading is automatic.

\section{slide}
By contrast,
Back \& Apperly (2010) found that subjects are significantly slower to answer an unexpected question about another's belief when that belief is false compared to when it is true.
This suggests that, at least in adults, mindreading is not automatic.

So my question is whether adult mindreading is automatic.
And this question is puzzling because there seems to be evidence for both positive and negative answers.

\section{Ian's first bit}
***

\section{Steve's second bit}
So the question is how we develop two systems for mindreading, one automatic the other not. Two hypotheses are:

a. automaticity through automatization
 
and 

b. original automaticity.

(Define 'originally automatic' mindreading as mindreading that is automatic and not a consequence of automatization.)

Our suggestion is that 
(i) automatic mindreading systems are subject to arbitrary limits (because automaticity involves trading flexibility for efficiency),
and 
(ii) we can distinguish the two hypotheses by finding arbitrary limits on automatic mindreading in infants and adults.
If the limits match, this is evidence for original automaticity.
If the limits differ, this is evidence for automaticity through automatization.

What sorts of limits might we expect to find on originally automatic mindreading?

Full-blown, explicit mindreading involves representing beliefs, desires and other propositional attitudes.
Propositional attitudes---including beliefs---are states which form complex causal structures, have arbitrarily nest-able contents, and are individuated by their causal and normative roles in explaining thoughts and actions.  
If anything should make a cognitive process effortful, it is surely representing states with that combination of properties

So there are strong theoretical reasons to suppose that originally automatic mindreading could not involve representing propositional attitudes as such.

\section{slide}
Elsewhere Ian and I have suggested, in effect, that originally automatic mindreading involves representing relational attitudes ...

\section{slide}
Paradigmatic examples of relational attitudes are things like 
being excited by an event,
encountering an object,
or wanting some type of thing.



\section{slide}
whereas propositional attitudes have arbitrarily nestable contents,
relational attitudes have no contents

whereas propositional attitudes have uncodifiably complex effects on action, 
relational attitudes set parameters for action

and whereas representing relational attitudes would permit a mindreader to track false beliefs involving mistakes about 
   appearances,
   identity
 and 
    existence,
representing relational attitudes would allow one to track only a limited range of false beliefs about properties such as location, colour and shape.

So consider this hypothesis:
whereas non-automatic mindreading involves representing propositional attitudes,
automatic mindreading involves representing relational attitudes only.

This hypothesis, if true, would explain why automatic mindreading is less demanding than full-blown mindreading.


\section{slide}

This hypothesis also has the virtue of indicating what sort of limits could distinguish automatic from full-blown mindreading ...


\section{slide}
for the hypothesis suggests that 

- automatic mindreading should be good for level-1 perspective taking tasks,
but not for level-2 perspective taking tasks.


Some work on the level-1/level-2 distinction has already been published,
and this does suggest that for both children and adults,
automatic mindreading may be sufficient for success on level-1 but not level-2 perspective taking tasks.

But since this has already been published I don't want to cover that here.


\section{slide}

The hypothesis about relational attitudes also suggests that 

- automatic mindreading should be good for tracking false beliefs with a simple predicate-object structure, but not for tracking false beliefs which essentially involve the non-existence of something.

\section{slide}

and, finally, the hypothesis suggests that

-automatic mindreading should be good for tracking false beliefs about the location of something,
but not for tracking false beliefs which essentially involve mistakes about identity.

So one nice feature of this hypothesis is that it makes several independently testable predictions about the limits of automatic mindreading.

If we want to find out whether   adult automatic mindreading and infant mindreading are subject to the same limits,
it makes sense to start by looking at these limits.

I want to finish by focusing on the issue of false beliefs about location vs. false beliefs about identity.

What do we mean by a mistake about identity?

Here's an illustration.
Suppose that a single person has two roles that don't often go together,
say she is both a school teacher  by day and an assassin by night.
If you shot the assassin, you might be surprised that the school teacher doesn't turn up to work.
They are one and the same person but you didn't know that.
This would be a mistake about identity: in mistakenly expecting the school teacher to turn up, you have failed to realise that she is the assassin.
Your mistake could be predicted by subjects who can represent propositional attitudes as such.
But it cannot be predicted by subjects who represent relational attitudes only.


\section{slide}
Here's how this idea might work in a format suitable for infants and adults (without the shooting).

On the screen you can see the back of the subject's head.
He is sitting opposite the protagonist.
On the table there is a screen.
Behind the screen there is a puppet.
The puppet will serve as the school teacher-assassin.
The protagonist can't see behind the barrier.
She can't see the puppet right now.
But the subject can see everything.


\section{slide}
This is the same thing, just smaller so that I can show you how the story unfolds.

I should say that the puppet has a face and costume with two aspects.
Seen from one side she looks like an assassin.
Seen from the other she looks like a school teacher.


\section{slide}
This is the same thing, just smaller so that I can show you how the story unfolds.

I should say that the puppet has a face and costume with two aspects.
Seen from one side she looks like an assassin.
Seen from the other she looks like a school teacher.


\section{slide}
The puppet now rotates behind the barrier.
The subject can see this but the protagonist can't.


\section{slide}
Now the puppet goes out from behind the barrier.
This time she looks to the protagonist as if she is an assassin.
The protagonist shoots her and she falls off the table.
So to the protagonist it looks like the assassin has been shot.
But the subject knows that the assassin is the school teacher.


\section{slide}
In the final phase,
the protagonist reaches around the barrier.

If the subject can represent propositional attitudes as such this makes perfect sense:
the protagonist wants to  retrieve the school teacher.

But if the subject is representing relational attitudes only,
this reaching doesn't make any sense.
For the assassin is the school teacher,
and so one's disappearance is the other's disappearance.

So the protagonist's reach should be expected to subjects who represent propositional attitudes but unexpected to subjects who represent relational attitudes only.

[Don't mention control conditions]


\section{slide}
So this was our idea.
We want to know whether adults' automatic mindreading is (a) automatic through automatization
or
(b) originally automatic.

Our suggestion is that if adult automatic mindreading is originally automatic, then it should be subject to the same arbitrary limits as infant mindreading.

And there are theoretical reasons to suppose that originally automatic mindreading would involve representing relational attitudes only, not propositional attitudes.

Someone who represents relational attitudes only
should have no problem tracking false beliefs about location
but should fail to track false beliefs about identity.

And, as I've just tried to illustrate, it would be possible to test for abilities to track false beliefs about identity using designs very similar to those used to track false beliefs about location.

So here is a way to test for original automaticity.
If we find that infant mindreading is like adult automatic mindreading in that neither enables them to track  false beliefs about identity,
then we have evidence for original automaticity.
But if it turns out that infants and adults differ in this respect,
then we have evidence for automaticity by automatization.


\section{Ian's conclusion}
***

\bibliography{$HOME/endnote/phd_biblio}

\end{document}