%!TEX TS-program = xelatex
%!TEX encoding = UTF-8 Unicode

\def \papersize {a4paper}
\documentclass[12pt,\papersize]{extarticle}
% extarticle is like article but can handle 8pt, 9pt, 10pt, 11pt, 12pt, 14pt, 17pt, and 20pt text

\def \ititle {How to Construct a Minimal Theory of Mind}
\def \isubtitle {}
\def \iauthor {Stephen A. Butterfill \& Ian A. Apperly}
\def \iemail{s.butterfill@warwick.ac.uk}
%for anonymous submisison
%\def \iauthor {}
%\def \iemail{}
%\date{}

%!TEX TS-program = xelatex
%!TEX encoding = UTF-8 Unicode

\title{\ititle\\\isubtitle}
\author{\iauthor\\<{\iemail}>}

\usepackage[\papersize]{geometry} % see geometry.pdf
\geometry{twoside=false}
\geometry{headsep=2em} %keep running header away from text
\geometry{footskip=1cm} %keep page numbers away from text
\geometry{top=3cm} %increase to 3.5 if use header
\geometry{left=4cm} %increase to 3.5 if use header
\geometry{right=4cm} %increase to 3.5 if use header
\geometry{textheight=22cm}

%non-xelatex
%\usepackage[T1]{fontenc}
%\usepackage{tgpagella}

%for underline
\usepackage[normalem]{ulem}

%get the font here:
% http://scripts.sil.org/CharisSILfont

\usepackage{fontspec,xunicode}
%nb do not explicitly use package xltxtra because this introduces bugs with footnote superscripting  -- perhaps because fontspec is supposed to include it anyway.
%UPDATE:  "You need to use the no-sscript option in xltxtra: \usepackage[no-sscript]{xltxtra}, this is explained in the documentation of xltxtra.  The issue is that Sabon does not contain true superscript glyphs for every character and the no-sscript option will instead use scaled regular glyphs, which is typographically inferior, but there is no other option available when using Sabon." --- http://groups.google.com/group/comp.text.tex/browse_thread/thread/19de95be2daacade
\defaultfontfeatures{Mapping=tex-text}
%\setromanfont[Mapping=tex-text]{Charis SIL} %i.e. palatino
%\setromanfont[Mapping=tex-text]{Sabon LT Std} 
%\setromanfont[Mapping=tex-text]{Dante MT Std} 
%\setromanfont[Mapping=tex-text,Ligatures={Common}]{Hoefler Text} %comes with osx
\setromanfont[Mapping=tex-text]{Linux Libertine O} 
\setsansfont[Mapping=tex-text]{Linux Biolinum O} 
\setmonofont[Scale=MatchLowercase]{Andale Mono}


%hyperlinks and pdf metadata
%TODO avoid duplication of title & author
\usepackage{hyperref}
\hypersetup{pdfborder={0 0 0}}
\hypersetup{pdfauthor={\iauthor}}
\hypersetup{pdftitle={\ititle\isubtitle}}


%handles references to labels (e.g. sections) nicely
\usepackage{varioref}

%line spacing
\usepackage{setspace}
%\onehalfspacing
%\doublespacing
\singlespacing

\usepackage{natbib}
%\usepackage[longnamesfirst]{natbib}
\setcitestyle{aysep={}}  %philosophy style: no comma between author & year

%enable notes in right margin, defaults to ugly orange boxes TODO fix
%\usepackage[textwidth=5cm]{todonotes}

%for comments
\usepackage{verbatim}

%footnotes
\usepackage[hang]{footmisc}
\setlength{\footnotemargin}{1em}
\setlength{\footnotesep}{1em}
\footnotesep 2em

%tables
\usepackage{booktabs}
\usepackage{ctable}

%section headings
\usepackage[sf]{titlesec}
%\titlespacing*{\section}{0pt}{*3}{*0.5} %reduce vertical space after header
%large headings:
%\titleformat{\section}{\LARGE\sffamily}{\thesection.}{1em}{} 
\titlelabel{\thetitle.\quad}

%captions
\usepackage[font={small,sf}, margin=0.75cm]{caption}

%lists
\usepackage{enumitem}
\newenvironment{idescription}
{ 	
	% begin code
	\begin{description}[
		labelindent=1.5\parindent,
		leftmargin=2.5\parindent
	]
}
{ 
	%end code
	\end{description}
}


%title
\usepackage{titling}
\pretitle{
	\begin{center}
	\sffamily
	\Huge
} 
\posttitle{
	\par
	\end{center}
	\vskip 0.5em
} 
\preauthor{
	\begin{center}
	\normalsize
	\lineskip 0.5em
	\begin{tabular}[t]{c}
} 
\postauthor{
	\end{tabular}
	\par
	\end{center}
}
\predate{
	\begin{center}
	\normalsize
} 
\postdate{
	\par
	\end{center}
}



\begin{document}

\setlength\footnotesep{1em}

\bibliographystyle{newapa} %apalike

%these two lines are for anonymous submission --- they remove author and date
%but don't forget to remove defs above as well --- otherwise it will be in the metadata
%\author{}
%\date{}


\maketitle
%\tableofcontents

\begin{abstract}
\noindent
What could someone represent that would enable her to track, at least within limits, perceptions, beliefs including false beliefs, and other propositional attitudes? 
An obvious possibility is that she might represent these very propositional attitudes as such.
It is sometimes tacitly or explicitly assumed that this is the only possible answer.
However we argue that several recent discoveries in developmental, cognitive, and comparative psychology indicate the need for other, less obvious possibilities.
Our aim is to meet this need by describing the construction of a minimal theory of mind.  
Minimal theory of mind is rich enough to explain systematic success on tasks held to be acid tests for theory of mind cognition including many false belief tasks.  
It is also extensible in ways that can explain a wide range of findings from non-human animals and human infants that are sometimes presented as evidence for full-blown theory of mind cognition.  
Yet minimal theory of mind does not require representing propositional attitudes, or any other kind of representation, as such.
Minimal theory of mind may be what enables those with limited cognitive resources or little conceptual sophistication, such as infants, chimpanzees, scrub-jays and human adults under load, able to track, within limits, facts about perceptions and beliefs.

\ 

\noindent
\textbf{Keywords:}
Theory of Mind, False Belief, belief, perception, development, comparative
\end{abstract}



\section{Introduction}
What could someone represent that would enable her to track, at least within limits, perceptions, beliefs including false beliefs, and other propositional attitudes? 
One answer is obvious: she might track these things by virtue of representing them as such, that is, by representing perceptions, beliefs, and other propositional attitudes as such.
Our aim in what follows is to identify another, less obvious answer.
There is a form of cognition---minimal theory of mind---which does not involve representing propositional attitudes as such but does involve representing simpler, relational mental states which could, within limits, enable one to track propositional attitudes such as beliefs.
Minimal theory of mind 
is  rich enough to enable systematic success on tasks held to be acid tests for theory of mind cognition including many false belief tasks.
As we will explain, this has consequences 
for interpreting a range of findings concerning infants', adults' and nonhumans' performances on theory of mind tasks.
It may help us to understand what enables those with limited cognitive resources or little conceptual sophistication, such as infants, chimpanzees, scrub-jays and human adults under load, to track, within limits, facts about perceptions and beliefs.

In this section we defend explain our question; in the next sections we introduce the findings which motivate facing it before starting to answer it in the fourth section.

Some may find our question initially incomprehensible.
Could abilities to track false beliefs (say) really involve anything other than representing false beliefs?
To see the possibility of a positive answer it may help to consider a non-mental analogy.
What could someone represent that would enable her to track, at least within limits, the toxicity of potential food items?
Here the most straightforward answer (she could represent their toxicity) is clearly not the only one.
After all, someone might track toxicity by representing odours or by representing visual features associated with putrefaction, say.
Suppose Sin\'ead has no conception of toxins but represents the odours of food items and 
treats those with foul odours as dangerous to eat,
so that she would not normally offer them to friends or family
nor conceal them from competitors.
This brings nutritional and competitive benefits obtaining which depends on facts about toxicity.
If Sin\'ead tends to behave in this way because of these benefits, 
representing odours enables her to track, in a limited but useful range of cases,  toxicity.
Our question, put very roughly, is whether   belief has something like an odour.

To make the question more precise it is useful to distinguish 
theory of mind abilities from theory of mind cognition.  
A  \textit{theory of mind ability} is an ability that exists in part because exercising it brings benefits obtaining which depends on exploiting or influencing facts about others’ mental states.
To illustrate,
suppose that Hannah is able to discern whether another's eyes are in view,
that Hannah exercises this ability to escape detection while stealing from others,
that Hannah's ability exists in part because it benefits her in this way,
and 
that Hannah's escaping detection depends on exploiting a fact about other's mental states (namely that they usually cannot  see Hannah's acts of theft when Hannah doesn't have their eyes in view).
Then Hannah has a theory of mind ability.
(This is not supposed to be a plausible, real-world example but only to illustrate what the definition requires.)
An ability to \textit{track} perceptions or beliefs (say) is a theory of mind ability which involves exploiting or influencing facts about these states.
By contrast, \textit{theory of mind cognition} involves representing mental states or processes as such.
And \textit{full-blown} theory of mind cognition involves  representing propositional attitudes such as beliefs, desires and intentions to construct reason-giving, causal explanations of action.  
The distinction between theory of mind abilities and theory of mind cognition matters because the facts about other minds which theory of mind abilities exploit are not necessarily the facts which are represented in theory of mind cognition.  
To return to the illustration, Hannah is able, within limits, to exploit facts about what others perceive without representing perceptions as such. 
She has a theory of mind ability while possibly lacking any theory of mind cognition.

It should be uncontroversial that some theory of mind abilities do not necessarily involve any theory of mind cognition at all. 
Our question concerns abilities to track what others perceive and believe, including their false beliefs; these have been central in psychological research.
Can anything less than full-blown theory of mind cognition  explain systematic success on a range of false belief tasks?
We do not aim to argue that someone could track beliefs, true and false, without any theory of mind cognition at all.
Our concern is rather with the construction of a minimal form of theory of mind cognition.
As we shall explain, minimal theory of mind does involve representing  belief-like  states, but it does not involve representing beliefs or other propositional attitudes as such.

The notion that some abilities to track perceptions or beliefs involve only on theory of mind cognition which does not involve representing perceptions or beliefs as such is not entirely novel.
To mention only those we draw most directly on, Gomez (\citeyear[][p.\ 730]{en_1259}) has emphasized primitive intentional relations to objects established by gaze, O’Neil and Doherty have separately discussed a notion of engagement with objects (\citealp{en_1159, en_1140}), Call and Tomasello (\citeyear[][p.\ 58]{en_1669}) have suggested that chimpanzees track the `likely target' of others’ visual access and understand something about its effects on behaviour, and Whiten (\citeyear[]{en_1415, en_1416}) uses the notion of an `intervening variable' to explain primitive theory of mind notions.  These are illuminating ideas and what follows can be seen as an elaboration and partial synthesis of them.  
The result---minimal theory of mind---is unlike its precursors in that it is rich enough to explain systematic success on a range of false belief tasks.  Our approach is novel in this and others respects to which we return (in the Conclusion) after presenting the substance of our account.



%The notion that theory of mind cognition could involve representing non-propositional counterparts of belief 
%According to Davidson, 
%\begin{quote}
%`We are stuck with our two main ways of describing and explaining things, one which treats objects and events as mindless, and the other which treats objects and events as having propositional attitudes. I see no way of bridging the gap by introducing an intermediate vocabulary.' \citep[p.\ 697]{Davidson:2003bw}
%\end{quote}
%It may be tempting to dismiss this assertion on the grounds that we can readily describe an infant as excited by a clapping game, or as preferring one toy to another. 
%On the surface at least, these descriptions seem neither to involve propositional attitudes nor to involve treating the infant as mindless; it seems there are non-propositional forms of excitement and preference which are nevertheless mental states.
%
%Even so, Davidson is right that there is a genuine difficulty when it comes to understanding non-propositional counterparts of attitudes like belief and perception.  We cannot rely entirely on commonsense here because our commonsense concepts of perception, belief, intention and action exhibit a form of holism: we grasp them only if we understand their interdependent roles in reason explanations (Davidson 1995b, 1995a). 
%




\section{Motivation}
Our question is theoretical: it concerns 
not what anyone does represent
but what someone could represent that would enable her, at least within limits, to track perceptions, beliefs and other propositional attitudes.
The motivation for facing up to this question is, of course, partly empirical.

Consider ordinary adult humans.
Since they can represent beliefs and other propositional attitudes as such, 
it is natural to assume that such representations underpin their abilities to track perceptions and beliefs.
But is this natural assumption correct?

To see that it might not be, consider a further question.
Is tracking others' perceptions and beliefs automatic?
Roughly speaking,
a process is \emph{automatic} if whether it occurs is to a significant degree independent of its relevance to the particulars of the subject's motives and aims.
(Note that a process may occur spontaneously without thereby being automatic.)  
Some evidence suggests that, for ordinary adult humans, belief tracking is automatic.
For example,
\citet{kovacs_social_2010} asked adults to identify the location of a ball.
They found that adults were significantly slower to identify the ball's location when an onlooker had a false belief about the location of the ball,
even though the onlooker's belief was not relevant to the task at all.
Relatedly, \citet{Samson:2010jm} provide evidence that identifying what another perceives is automatic;  this finding is indirectly supported by  evidence that tracking others' perceptions is not disrupted by a secondary executive task \citep{qureshi:2010_executive}.
Taken together, these findings suggest that, at least in adults, tracking others' perceptions and beliefs is sometimes automatic.

But there is also a body of evidence supporting a different conclusion.
\citet{back:2010_apperly} found that subjects are significantly slower to answer an unexpected question about another's belief when that belief is false compared to when it is true \citep[see also][]{apperly:2006_belief}.
This suggests that, at least in adults, belief tracking is not automatic.
There is also evidence that, even in relatively simple situations, 
using facts about others' beliefs is not automatic \citep{Keysar:2003xu,apperly:2010_limits}.
The case for nonautomaticity is indirectly supported by evidence that tracking perceptions and beliefs---and even merely holding in mind what another believes, where no inference is required---involves a measurable processing cost  \citep{apperly:2008_back,apperly:2010_limits}, consumes attention and working memory in fully competent adults (\citealp{en_1698, lin:2010_reflexively, en_1547} experiments 4-5), may require inhibition \citep{bull:2008_role} and makes demands on executive function \citep{apperly:2004_frontal,samson:2005_seeing}.
These findings, taken together, suggest that tracking others' perceptions and beliefs is sometimes not automatic.

The question was whether, in adult humans,  tracking perception and belief is automatic.  
If we assume, further, that either all such processes are automatic or else none are, then the evidence creates a conflict.
This conflict  cannot easily be explained away by appeal to simple methodological factors or extraneous task demands.
For instance, it may be tempting to suppose that the conflict can be explained by distinguishing between linguistic and nonlinguistic tasks.  
But belief ascription may fail to be automatic even on some nonlinguistic tasks \citep[e.g.][]{apperly:2004_frontal}, and we know of no reason to assume that belief ascription could not be automatic on some linguistic theory of mind tasks (such as those where spontaneous tracking is already established, e.g.\ \citet{ferguson_listeners_2012}).

If the conflict is not a methodological artefact, how should we interpret the evidence?
Perhaps it should be taken at face value.
This means we must reject the assumption that tracking others' perceptions and beliefs is either always automatic or else always nonautomatic.
In other cases, such as number and causation, it is already quite widely accepted that, in adult humans, some abilities to track these things are automatic whereas others are not.\footnote{
On number: \citet{trick:1994_small};
on causation: \citet{Michotte:1946nz}, \citet{Scholl:2004dx}.
}
The evidence suggests that the same may be true for perception and belief.
In adult humans, some theory of mind abilities involve automatic processes whereas others depend on nonautomatic processes. 

A closely related view has already been elaborated and defended in more detail by
\citet[]{Apperly:2009ju},
although their argument complements ours by drawing  primarily on developmental and comparative research. 
According to them,
adults may enjoy efficient but inflexible forms of theory of mind cognition in addition to the full-blown form which involves representing beliefs and other propositional attitudes as such.
While aspects of this conjecture have already been tested \citep[]{Samson:2010jm, en_2397, surtees_direct_2011}, it raises two complementary questions (as Apperly \& Butterfill themselves note).  

First, why isn't tracking belief and perception always automatic?
Consider what is involved in representing beliefs and other propositional attitudes.
On any standard view, propositional attitudes form complex causal structures, have arbitrarily nestable contents, interact with each other in uncodifiably complex ways and are individuated by their causal and normative roles in explaining thoughts and actions \citep[]{en_809, en_249}.  
If anything should consume working memory and other scarce cognitive resources, it is surely representing states with this combination of properties.
So even without knowing in any detail how theory of mind cognition is implemented, it is plausible that some feature, or combination of features, of the propositional attitudes  makes full-blown theory of mind cognition demanding.%
\footnote{
Several hypotheses about which feature of the propositional attitudes explains why full-blown theory of mind cognition is cognitively and conceptually demanding have been defended \citep[e.g.][]{en_1263, en_634, en_1269, en_78, en_81, en_404, 	%en_687, %bibtex error authors and year too similar
	en_643, en_1130}.  
More than one feature may contribute. 
We are agnostic about which feature or features are to blame.
}  
%
A possible explanation, then, is this.
Tracking perception or belief is not always automatic because 
it sometimes involves representing propositional attitudes as such,
which typically or always places demands on working memory, attention and executive function  that are incompatible with automaticity.


Second, how could  tracking perceptions or beliefs ever be automatic?
If we assumed that such tracking always involved propositional attitudes as such, this question would present a puzzle.
For, as we saw, representing propositional attitudes as such generally places demands on working memory, attention and executive function that are incompatible with automaticity.
In some cases these demands might be overcome through automatization in something like the way that initially effortful numerical operations can through practice become automatic.\footnote{
On the automatization of simple sums, see \citet{lefevre:1988_cognitive}.
For the suggestion that something similar might happen concerning mental states, see \citet{Suddendorf:2003co}.
%which is also discussed by \citet[p.\ 961]{Apperly:2009ju}.
}
However, almost nothing is known about to what extent, if any, automatization occurs in theory of mind. 
And in any case automatization can only explain the automaticity of routine inferences.
So it is possible that automatization, although perhaps important, does not fully explain the automaticity of some of adult humans' perception- and belief-tracking abilities.
A full explanation may  depend on showing that tracking perceptions and beliefs can be done without  representing beliefs or other propositional attitudes as such.

This is a source motivation for our question about what someone could represent that would enable her to track perceptions and beliefs.
The existence of both automatic and nonautomatic tracking of perceptions and beliefs in human adults 
suggests (without decisively showing, of course), 
contrary to a natural assumption mentioned above,
that not all of their abilities to track perceptions and beliefs involve representing propositional attitudes as such.

\section{More motivation}
Further motivation for our question comes from evidence for theory of mind abilities in young children and infants.
Children in their second year use pointing to provide information to others \citep[]{en_1093} in ways that reflect a partner’s ignorance or knowledge \citep[]{en_1699}, as well as providing more information to ignorant than knowledgeable partners when making requests \citep[]{en_1140}.  One-year-old children also predict actions of agents with false beliefs about the locations of objects \citep[]{en_1092, en_1208} and choose different ways of interacting with others depending on whether their beliefs are true or false \citep[]{en_1783,Knudsen:2011fk,southgate:2010fb}.  And in much the way that irrelevant facts about the contents of others’ beliefs modulate adult subjects’ response times, such facts also affect how long 7-month-old infants look at some stimuli \citep[]{kovacs_social_2010}.

What do these infants and young children represent that enables them, within limits, to track others’ beliefs and other propositional attitudes?   
The most straightforward answer would be to suppose that they represent perceptions, beliefs and other propositional attitudes as such \citep[e.g.][]{en_1138, en_1691}.  
But this answer faces several objections.  A body of evidence  suggests that representing beliefs requires conceptual sophistication, for it has a protracted developmental course stretching over several years \citep[]{en_87, en_89} and its acquisition is tied to the development of executive function \citep[]{en_410, en_1130} and language \citep[]{en_1209}.  Infants and young children are deficient in these.  
Development of reasoning about beliefs in humans may also be facilitated by explicit training \citep[]{en_85} and environmental factors such as siblings \citep[]{en_507, en_1299}.  
This is evidence that representations of belief in humans typically emerge from extensive participation in social interactions (as \citealp{en_1300} suggest).  
If any of this is right, we must reject the hypothesis that infants are representing beliefs or other propositional attitudes as such.

In principle an alternative would be to suppose that infants' and young children's abilities to track perceptions and beliefs 
do not involve any theory of mind cognition at all
but are instead based on 
representations of nonintentional behaviour only.
It is arguably possible in principle to explain some belief-tracking abilities by appeal to hypothetical behaviour reading capacities  \citep{perner:1988_developing,en_1168, en_1169}.
But there are several objections to the claim that the full range of even infants' abilities to track perceptions and beliefs could be explained in this way \citep{en_1691,Apperly:2009ju}.  
And what is is currently known about humans' actual behaviour reading capacities suggests that they are unlikely to explain systematic success on false belief tasks.%
\footnote{
Key studies include
	\citet{Newtson:1976ni}, 
	\citet{Byrne:1999jk},
	\citet{Baldwin:2001rs},
	\citet{Saylor:2007pj} and
	\citet{Baldwin:2008mw}.
} 
 
Here, then, is a second source of motivation for our question about what someone could represent that would enable her, within limits, to track perceptions and beliefs.
As we have seen, there are significant if not decisive objections to the two best developed conjectures about infant theory of mind abilities, the conjecture that these involve representing beliefs and other propositional attitudes as such and the conjecture that these involve representing nonintentional behaviour only.  
These objections, while not decisive, justify exploring alternatives.

Theory of mind abilities are not only found in humans.
For instance,
scrub-jays can selectively re-cache their food in ways that deprive competitors of knowledge of its location \citep{Clayton:2007fh}, and  chimpanzees can both select routes to approach food which conceal them from a competitor’s view \citep[]{en_1546} and also retrieve food using strategies that optimize their return given what a dominant competitor has seen \citep[]{en_1545}.  
There is debate about the cognitive underpinnings of these abilities (e.g.\ 
	\citealp{povinelli:2004vonk, 
			Penn:2007ey,
			Tomasello:2005ce,
			Call:2008di}). 
If it is not yet known precisely what explains these abilities 
and if the available evidence does not already tightly constrain the space of admissible conjectures,
then the construction of a minimal theory of mind may be relevant to these debates.
For the conjecture that minimal theory of mind explains chimpanzees' or scrub-jays' abilities to track perceptions or beliefs can be empirically distinguished from conjectures about representations of nonintentional behaviour only and from conjectures about representations of perceptions, beliefs and other propositional attitudes as such (as we explain in section *** below).


%The construction of minimal theory of mind shows that there is a form of theory of mind cognition which does not involve representing beliefs or other propositional attitudes as such but is capable of explaining, within limits, systematic success on a range of false belief tasks.
%As we shall show,
%the conjecture that minimal theory of mind cognition underpins some abilities to track perceptions and belief in human adults and infants generates testable predictions that distinguish it from the alternatives.




\section{The need for a constructive approach}
Davidson ...

%This is reflected in the dichotomy between behaviour reading (P&Vonk) \& full-blown ToM (Baillargeon review).

Davidson might be right about ordinary, commonsense psychology.

So a constructive approach is needed. ***





\section{Minimal theory of mind}

In this section we identify concepts and principles necessary for constructing a form of theory of mind cognition.  We begin with someone, call her Lucky, capable  of representing nonintentional behaviour only and ask what more is needed for minimal theory of mind cognition.  We describe Lucky’s progress with a series of five principles. The principles are constructed in such a way that it would be coherent to suppose that Lucky has the abilities codified by the first \textit{n} principles only. They are not intended to represent a developmental or evolutionary progression.  The principles can also be extended to explain more sophisticated theory of mind abilities than those considered here.  We restrict ourselves to these five principles because they are sufficient to explain success on some well-known false belief tasks.%
\footnote{
In standard false belief tasks, “[t]he subject is aware that he/she and another person [call him Maxi] witness a certain state of affairs x.  Then, in the absence of the other person the subject witnesses an unexpected change in the state of affairs from x to y” \citep[][p.\ 106]{en_89}.  The test concerns whether the subject realises that Maxi will falsely believe x to obtain.  In many cases the states of affairs, x and y, differ only with respect to the location of an object \citep[e.g.][]{en_1092, en_1208, en_1824}. 
As we go on to discuss, our proposal for a minimal theory of mind could easily be extended to cover a range of other cases;
but importantly there are also false belief tasks success on which cannot be explained by minimal theory of mind cognition (see section ***).
}


	We aim to provide the core elements of a computational theory in Marr’s sense (\citeyear[][pp.\ 15-29]{en_917}) where our computational theory, unlike the standard full-blown theory of mind which hinges on beliefs, desires and other propositional attitudes, is one that could be realised in a cognitively efficient manner without requiring conceptual sophistication.\footnote{ 	To provide a computation theory in Marr’s sense is not to imply that its implementation will involve computational processes.}  There are multiple ways in which this computational theory might be implemented.  We shall not discuss how the theory might be implemented here other than to note that it seems unlikely that the principles formulated below are represented explicitly.  It is valuable to articulate the computational theory in some detail before formulating and testing conjectures about implementation.  



\subsection{First principle}

***HERE

Until now Lucky has been aware only of object-directed actions.  A further basic unit of action in theory of mind cognition is goal-directed action. To illustrate, a paradigm goal-directed action would be drinking from a bottle, which typically involves multiple object-directed actions such as first grasping a bottle and then lifting a bottle.  Many predictions do not directly concern object-directed actions but rather goal-directed actions.  Awareness of goal-directed actions is therefore an important further step in acquiring minimal theory of mind cognition. 

The term ‘goal-directed action’ can be used to mean several things.  One is intentional action.  This notion is no use for constructing a minimal theory of mind.  To represent intentional actions as such you also have to represent intentions and related propositional attitudes such as beliefs \citep[]{en_18}.  Constructing a minimal theory of mind requires a basic notion of goal-directed action.  We take the units of goal-directed action to be collections of one or more object-directed behaviours.  We stipulate that for an outcome, \textit{g}, to be the goal of a collection of object-directed behaviours is for it to occur in order to bring about \textit{g}; that is, \textit{g} is the function of this collection.  Here ‘function’ should be understood teleologically.  On the simplest teleological construal of function, for an action to have the function of bringing about \textit{g} would be for actions of this type to have brought about \textit{g} in the past and for this action to occur in part because of this fact \citep[see further][]{en_144, en_141, en_162, en_139, en_161}.  Lucky needs some ability to track the functions of things (in this sense of ‘function’) so that she can distinguish from among all the outcomes of an action those which are its goals.%
\footnote{
Note that the requirement is not that Lucky understands the theoretical account of functions, only that she can distinguish between things which have different functions in this theoretical sense of ‘function’.  A wide variety of research supports the claim that young children, non-human primates and corvids track the functions of things (including 
	\citealp{en_1086},
	\citealp{en_1318},
	\citealp{en_1431},
	\citealp{en_1447},
	\citealp{en_1325} and
	\citealp{en_1708}%
).
}

This is not an account of full-blown goal-directed action.  It is what someone who has only a minimal grasp of goal-directed action might understand.  Representing goals in this sense does not require representing representations because such goals are non-representational.  Representing goals only involves representing outcomes as functions of behaviours. 

To illustrate the difference between object-directed and goal-directed actions, consider a scrub-jay repeatedly caching and re-caching a worm (we take our inspiration from studies such as Emery \& Clayton, 2001).  The outcome of this collection of object-directed caching behaviours is (let’s say) that an observer is deprived of information about the location of the worm.  This is not an outcome of any of the constituent object-directed behaviours taken individually.  It is only an outcome of the collection considered as a whole.  The significance of the whole sequence can only by appreciated by Lucky once she is equipped to represent goals.  

The first principle, then, is that object-directed actions form larger units of action which are directed to goals.  This second principle is sufficient to explain some cases of imitative learning, which can be defined as attempting to reproduce the actions necessary to achieve a goal \citep[]{en_1317}.  


































\small
\bibliography{$HOME/endnote/phd_biblio_en_record_num_keys}

\end{document}