%!TEX TS-program = xelatex
%!TEX encoding = UTF-8 Unicode

%\def \papersize {a4paper}
\def \papersize {letterpaper}

\documentclass[12pt,\papersize]{extarticle}
% extarticle is like article but can handle 8pt, 9pt, 10pt, 11pt, 12pt, 14pt, 17pt, and 20pt text

\def \ititle {Joint Action and the Emergence of Mindreading}
\def \isubtitle {}
\def \iauthor {}
\def \iauthor {S. Butterfill*
\\ 
**Department of Philosophy, University of Warwick, UK
}
\def \iemail{s.butterfill@warwick.ac.uk}
%\date{}

\input{$HOME/Documents/submissions/preamble_steve_paper2}
%\author{}
%\author {S. Butterfill* \& C. Sinigaglia**
%\\ 
%**Department of Philosophy, University of Warwick, UK
%\\ 
%***Dipartimento di Filosofia, Università degli Studi di Milano, Italia}
\date{}


\begin{document}

\setlength\footnotesep{1em}

\bibliographystyle{newapa} %apalike

\maketitle
%\tableofcontents
\title{}

\begin{abstract}
\noindent
***
\end{abstract}

\tolerance=5000



\section{A Challange and a Conjecture}

I want to start with a challenge.  
The challenge is to explain the emergence, in evolution or development (or both), of sophisticated forms of mindreading. 
By `mindreading' I mean the ascription to others of mental states and the use of such ascriptions in explaining and predicting their thoughts or actions. 
%%compare:
%`In saying that an individual has a theory of mind, we mean that the individual imputes mental states to himself and to others ... [T]he system can be used to make predictions, specifically about ... behavior'
%(Premack & Woodruff 1978: 515)
And by  `sophisticated forms of mindreading', I mean those which involve representing perceptions, knowledge states, intentions, beliefs and other propositional attitudes. 
%In other words, the sort of mindreading  measured by tasks such as the famous false belief task \citep{Wimmer:1983dz}.


Several researchers have conjectured that social interaction partially explains how sophisticated forms of cognition, including mindreading, emerge in development or evolution.
According to Moll and Tomasello' Vygotskian Intelligence Hypothesis, 
%
\begin{quote} 
`the unique aspects of human cognition ... were driven by, or even constituted by, social co-operation'
\citep[p.\ 1]{Moll:2007gu}.
\end{quote}
%
Independently, Knoblich and Sebanz hypothesise that: 
\begin{quote} 
`perception, action, and cognition are grounded in social interaction
 … functions traditionally considered hallmarks of individual cognition originated through the need to interact with others' \citep[p.\ 103]{Knoblich:2006bn}.
\end{quote}
%
Although these hypotheses concern social interaction or social co-operation generally, both sets of researchers focus on a particular kind of social interaction, 
I also want to focus on joint action.
The conjecture I am interested in links mindreading and joint action:
%
\begin{quote}
The existence of abilities to engage in joint action partially explain how sophisticated forms of mindreading emerge in evolution or development (or both).%
\end{quote}
%
I am not certain that this conjecture would be endorsed by Moll and Tomasello, or by Knoblich and Sebanz.
But it is inspired by their ideas and findings.
In what follows I present a series of objections to the conjecture and explain how they can be overcome.
I do not hope to show that the conjecture is correct. 
My aim is only to show that, so far, there are no good reasons to reject it.
Doing this will involve making clearer what the conjecture amounts to and relating it to some testable predictions.  
This may not sound like much of an achievement. 
Shouldn't I roll my sleeves up and actually test the conjecture before writing about it? 
Perhaps.
But note that there is no good alternative to this conjecture, and that we currently lack suitable conceptual tools even to properly understand, let alone test, it. 
In particular the notions of mindreading and joint action are  in need of some empirically motivated philosophical attention.
Anyway, we have to start somewhere if we are going to explain the emergence of mindreading: I proposal we start with the conjecture.

I have been using the term `joint action' without explanation. 
It is not easy to say what joint action is; this (along with how to distinguish kinds of mindreading) is a central issue in what follows. 
Researchers normally introduce it just by giving examples.  Paradigm cases in philosophy include two people 
	painting a house together \citep{Bratman:1992mi},
	lifting a heavy sofa together  \citep{Velleman:1997oo},
	preparing a hollandaise sauce together \citep{Searle:1990em},
	going to Chicago together \citep{Kutz:2000si},
	and walking together \citep{gilbert_walking_1990}.
In developmental psychology paradigm cases of joint action include  two people 
	tidying up the toys together \citep{Behne:2005qh},
	cooperatively pulling handles in sequence to make a dog-puppet sing \citep{Brownell:2006gu},
	and bouncing a block on a large trampoline together \citep{Tomasello:2007gl}.
Other paradigm cases from research in cognitive psychology include two people
	lifting a two-handled basket  \citep{Knoblich:2008hy},
	putting a stick through a ring \citep{ramenzoni_joint_2011},
	and swinging their legs in phase \citep%[p. 284]
	{schmidt_richardons:_2008}.
We need to be careful because it is not obvious that there is a single phenomenon of which all these are paradigm cases. 

A better way to introduce joint action is by using contrast cases.
Contrast cases are pairs of events which are similar in terms of the behaviour and coordination they involve but where one is a joint action while the other is not.  
Thus \citet{gilbert_walking_1990} contrasts two people walking together with two people individually walking side by side.  
The two pairs' movements may be the same and similarly coordinated (to avoid collision), but walking together is a joint action whereas merely walking side by side is not. 
Relatedly,  \citet{Searle:1990em}  contrasts a case in which several park visitors simultaneously run to a central shelter in order to perform a dance with another case in which the park visitors run to the central shelter in order to escape a storm.  The first is a case of joint action, the second is not; but the same movements occur in both.  
These contrast cases invite the question, 
How do joint actions differ from individual but parallel actions? 
Gilbert's example shows that the difference can’t just be a matter of coordination, because people who are merely walking alongside each other also need to coordinate their actions in order to avoid colliding.  
And Searle's example shows that the difference between joint action and parallel individual action can’t just be that the actions have a common effect because merely parallel actions can have common effects too. 
Contrast cases at least enable us agree on what joint action isn't.


My plan is simple. 
The first objection is that sophisticated forms of mindreading emerge (in evolution or development) before joint action. 
In that case, the emergence of mindreading could hardly involve joint action. 
The reply to this objection focuses on kinds of mindreading.  The second (of two) objections is that engaging in joint action requires sophisticated mindreading.
If this were right, any appeal to joint action in an account of how sophisticated forms of mindreading emerge  would presuppose what was supposed to be explained.   
In considering this objection, we will see that some kinds of joint action do require sophisticated mindreading. 
A good reply therefore depends on showing that there is more than one kind of joint action, and that not all joint action requires sophisticated kinds of mindreading.
At this point we will have removed some objections to the conjecture.
It is then natural to ask how it could to work.
How could joint action be involved in explaining the emergence of sophisticated forms of mindreading?
We will see how materials from the replies to the two objections can be assembled to answer this question.



\section{First Objection}





\bibliography{$HOME/endnote/phd_biblio}

\end{document}