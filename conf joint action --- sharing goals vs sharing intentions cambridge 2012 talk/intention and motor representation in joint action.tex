%!TEX TS-program = xelatex
%!TEX encoding = UTF-8 Unicode

\def \papersize {a4paper}

\documentclass[12pt,\papersize]{extarticle}
% extarticle is like article but can handle 8pt, 9pt, 10pt, 11pt, 12pt, 14pt, 17pt, and 20pt text

\def \ititle {Intention and Motor Representation in Joint Action}
\def \isubtitle {}
\def \iauthor {S. Butterfill* \& C. Sinigaglia**
\\ 
**Department of Philosophy, University of Warwick
\\ 
***Dipartimento di Filosofia, Università degli Studi di Milano}
\def \iemail{s.butterfill@warwick.ac.uk}
%\date{}

%!TEX TS-program = xelatex
%!TEX encoding = UTF-8 Unicode

\title{\ititle\\\isubtitle}
\author{\iauthor\\<{\iemail}>}

\usepackage[\papersize]{geometry} % see geometry.pdf
\geometry{twoside=false}
\geometry{headsep=2em} %keep running header away from text
\geometry{footskip=1cm} %keep page numbers away from text
\geometry{top=3cm} %increase to 3.5 if use header
\geometry{left=4cm} %increase to 3.5 if use header
\geometry{right=4cm} %increase to 3.5 if use header
\geometry{textheight=22cm}

%non-xelatex
%\usepackage[T1]{fontenc}
%\usepackage{tgpagella}

%for underline
\usepackage[normalem]{ulem}

%get the font here:
% http://scripts.sil.org/CharisSILfont

\usepackage{fontspec,xunicode}
%nb do not explicitly use package xltxtra because this introduces bugs with footnote superscripting  -- perhaps because fontspec is supposed to include it anyway.
%UPDATE:  "You need to use the no-sscript option in xltxtra: \usepackage[no-sscript]{xltxtra}, this is explained in the documentation of xltxtra.  The issue is that Sabon does not contain true superscript glyphs for every character and the no-sscript option will instead use scaled regular glyphs, which is typographically inferior, but there is no other option available when using Sabon." --- http://groups.google.com/group/comp.text.tex/browse_thread/thread/19de95be2daacade
\defaultfontfeatures{Mapping=tex-text}
%\setromanfont[Mapping=tex-text]{Charis SIL} %i.e. palatino
%\setromanfont[Mapping=tex-text]{Sabon LT Std} 
%\setromanfont[Mapping=tex-text]{Dante MT Std} 
%\setromanfont[Mapping=tex-text,Ligatures={Common}]{Hoefler Text} %comes with osx
\setromanfont[Mapping=tex-text]{Linux Libertine O} 
\setsansfont[Mapping=tex-text]{Linux Biolinum O} 
\setmonofont[Scale=MatchLowercase]{Andale Mono}


%hyperlinks and pdf metadata
%TODO avoid duplication of title & author
\usepackage{hyperref}
\hypersetup{pdfborder={0 0 0}}
\hypersetup{pdfauthor={\iauthor}}
\hypersetup{pdftitle={\ititle\isubtitle}}


%handles references to labels (e.g. sections) nicely
\usepackage{varioref}

%line spacing
\usepackage{setspace}
%\onehalfspacing
%\doublespacing
\singlespacing

\usepackage{natbib}
%\usepackage[longnamesfirst]{natbib}
\setcitestyle{aysep={}}  %philosophy style: no comma between author & year

%enable notes in right margin, defaults to ugly orange boxes TODO fix
%\usepackage[textwidth=5cm]{todonotes}

%for comments
\usepackage{verbatim}

%footnotes
\usepackage[hang]{footmisc}
\setlength{\footnotemargin}{1em}
\setlength{\footnotesep}{1em}
\footnotesep 2em

%tables
\usepackage{booktabs}
\usepackage{ctable}

%section headings
\usepackage[sf]{titlesec}
%\titlespacing*{\section}{0pt}{*3}{*0.5} %reduce vertical space after header
%large headings:
%\titleformat{\section}{\LARGE\sffamily}{\thesection.}{1em}{} 
\titlelabel{\thetitle.\quad}

%captions
\usepackage[font={small,sf}, margin=0.75cm]{caption}

%lists
\usepackage{enumitem}
\newenvironment{idescription}
{ 	
	% begin code
	\begin{description}[
		labelindent=1.5\parindent,
		leftmargin=2.5\parindent
	]
}
{ 
	%end code
	\end{description}
}


%title
\usepackage{titling}
\pretitle{
	\begin{center}
	\sffamily
	\Huge
} 
\posttitle{
	\par
	\end{center}
	\vskip 0.5em
} 
\preauthor{
	\begin{center}
	\normalsize
	\lineskip 0.5em
	\begin{tabular}[t]{c}
} 
\postauthor{
	\end{tabular}
	\par
	\end{center}
}
\predate{
	\begin{center}
	\normalsize
} 
\postdate{
	\par
	\end{center}
}


%\author{}
%\date{}

%\setromanfont[Mapping=tex-text]{Sabon LT Std} 

\begin{document}

\setlength\footnotesep{1em}

\bibliographystyle{newapa} %apalike

\maketitle
%\tableofcontents
\title{}

\begin{abstract}
On the assumption that social motor representation plays a role in explaining how effective joint action is possible, do we also need motor representation to explain what joint action is?  Philosophers tend to assume that motor representation is only an enabling condition for joint action and of no direct interest to narrowly philosophical theories of joint action and shared intention.  In this talk I shall argue that social motor representation and shared intention  have distinctive roles in explaining the purposiveness of joint action.  This gives rise to a challenge.  On the one hand, effective joint action---imagine two people erecting a tent in a gale together---sometimes requires both shared intentions and social motor representations plus a certain kind of harmony between the two.  On the other hand, recognizing their distinctive roles precludes the existence of direct inferential links between shared intentions and social motor representations.  The challenge is to explain how these two kinds of representation could sometimes harmoniously contribute to effective joint action despite the lack of inferential integration.
\end{abstract}

\section{Slide}
[hello]

\section{Slide}
As a philosopher I’m going to start from an empirically controversial premise.
I don’t think this premise has been established, but I do think it’s a reasonable bet.


\section{Slide}
A motor representation is \emph{agent-neutral} if it concerns an action which is not one’s own or, in the case of joint action, not entirely one’s own.


\section{Slide}
Two or more motor representations are \emph{reciprocal} just if there is a single outcome which each motor representation represents.


\section{Slide}
Sometimes I’ll using the term \emph{social} motor representation for reciprocal agent-neutral motor representations.
I have a slightly bad conscience about this because I’m not sure they’re actually social in any ordinary sense.

It is almost uncontroversial that agent-neutral motor representations exist, and I see no reason to doubt that they could be reciprocal.
The controversial part of the premise is the claim that such representations can enable joint action.

I should say, by the way, that I’m not suggesting that reciprocal agent-neutral motor representation is involved in every joint action.
Perhaps some joint actions do not involve social motor representation.
The premise is just that social motor representation is among the factors which enable some joint actions.



\section{Slide}
I’m not going to defend the premise, although I will try to explain the premise in more detail in a moment.  But first let me outline where I’m going with this.



\section{Slide}
The premise leads to a question and, relatedly, to a challenge.  



\section{Slide}
The question is whether social motor representation is not only an enabling condition for joint action but also plays a role in explaining what joint action is. I want to argue that it does.  In fact I want to argue that it plays a role similar and complementary to that of shared intention.



\section{Slide}
After this, in the last part of my talk, I also want to point to a challenge raised by the existence of social motor representation.  
The challenge is roughly this: planning for joint action involves representations of target outcomes in two different formats, a motor format and a propositional format.  
Given that representations in these different formats cannot straightforwardly be inferentially integrated in practical reasoning, what could ever insure that there is sometimes non-accidental harmony between motor representations and shared intentions?





\section{Slide}
Let me start by explaining the premise: Reciprocal agent-neutral motor representation enables joint action.
Why think this?
A direct way to test the premise would be to selectively intervene on social motor representation and see how it affects agents’ performance of joint actions.
As far as I know no one has done this.
But there are some indirect findings.
I shall mention just one, a finding by Dimitris Kourtis and colleagues.


\section{Slide}
They found that `the social relation between individuals modulates action simulation' \citep[p.\ 1]{kourtis:2010_favoritism}.  


\section{Slide}
In slightly more detail:
\begin{quote}
`motor activation during action anticipation depends on the ... relation between the actor and the observer ... Simulation of another person’s action, as reflected in the activation of motor cortices, gets stronger the more the other is perceived as an interaction partner.’  \citep[p.\ 4]{kourtis:2010_favoritism}
\end{quote}
I realise this doesn’t conclusively support the premise.  But I think it is evidence in its favour.



\section{Slide}
Here’s the experiment they did.
The task was simple: two people sat opposite each other.  Sometimes they acted alone, picking up and replacing an object.  And sometimes had to act together, passing an object between them.  Also present was the ‘loner’ who always acted alone.



\section{Slide}
EEG measurements of motor activation were recorded.



\section{Slide}
[image shrinks]


\section{Slide}
The researchers found that patterns of activation for self and joint action partner were similar, and different from patterns of activation for the actions of the loner which were similar to patterns in a ‘no go’ condition where no one was to act.



\section{Slide}
So here’s what I take from this paper.
If you are engaged in a joint action with someone, one which involves moving an object by passing it between you, then each of you has motor representations of the other’s actions and these motor representations are functionally equivalent to motor representations of your own actions in the sense that they are just the sorts of representation that might have caused you to do what the other is doing (if you were in her position).


\section{Slide}
Suppose that Kourtis et al are right that social motor representation is more likely to occur in joint action than when one is merely observing.
How could this facilitate joint action?



\section{Slide}
Let’s step back and consider an individual action.
An agent moves a mug from one place to another, passing in from her left hand to her right hand half way [*demonstrate].
It’s a familiar idea that motor representations for planning and monitoring action involve an hierarchical structure,
where there is a relatively abstract representation of an outcome that is progressively filled in.




\section{Slide}
I don’t suppose that this attempt to depict the hierarchy of motor representations involved in moving a mug with two hands is accurate.
I’m only trying to illustrate two familiar ideas.
One is that motor planning involves starting with relatively abstract representations of outcomes and filling in details.




\section{Slide}
The other is that there is a need, even for a single agent, to synchronize the exchange between the two hands.

How is this relevant to the case of joint action?



\section{Slide}
In joint action the agents have the same goal, to move the object from there to here.



\section{Slide}
They also face a similar coordination problem, requiring a precisely timed swap



\section{Slide}
And Koutis et al’s findings (and others’ findings) suggest that the same planning in involved in the joint action case, almost up to the actual muscle contractions.
That is, in the joint action situation each agent plans both agents’ actions as if they were the actions of a single agent.
This may be what enables them to coordinate so well : each is able to plan her own actions in a way that meshes with the other agent’s actions because each agent is planning (and monitoring) both their actions almost as if a single agent were going to execute the whole action.
And of course this is exactly what we want for small-scale joint action---we want two or more agents to act as one.



\section{Slide}
So what is the difference between the individual and the joint case?  From the point of view of motor representation, the primary difference may be that in joint action there is a need to inhibit execution of the parts of the action which are not one’s own.

Here then is the basic idea I take to be guiding Kourtis and others.
The idea is that coordination is sometimes achieved by having each agent’s motor system plan all of their actions; 
given some assumptions, this could be a way of making it likely that each will execute their part in the joint action in a way that meshes with the way the other agents execute their parts.

Now so far I’ve only been considering a possible role for social motor representation in enabling joint action.  
But I think the details already give us grounds for holding that motor representation has a role to play in explaining what joint action is.
To see why,
let’s go back to individual action for a moment again.



\section{Slide}
A basic question about ordinary, individual action is:
What is the relation between a purposive action and the outcome or outcomes to which it is directed?

Many ordinary purposive actions have many different outcomes.
Grabbing little Isabel by the hands I swing her around, causing her to laugh and, simultaneously, breaking a vase.
In fact the outcome to which this purposive action was directed might not be among its actual outcomes; after all, actions can fail.



\section{Slide}
So among all the actual and possible outcomes of my action, one or some are singled out as specially related to this action.
One aspect of the question concerns what singles out the outcome or outcomes, actual or merely possible, to which a particular purposive is directed.
But there is also a second aspect ...



\section{Slide}
Ordinary purposive actions are sometimes composed of more than one motor action.  My swinging Isabel around starts with my reaching for her wrists, grasping them and then spinning us around ... and my action doesn’t include other things which I might be doing simultaneously, like refusing a cup of tea with my eyes or  trying to determine whether that smell is coming from Isabel’s sister Hannah’s nappy.



\section{Slide}
So another aspect of our question is what determines which activities comprise the purposive action and which do not.





\section{Slide}
The standard answer to this question involves intention.
An intention (1) represents an outcome, (2) coordinates the one or several activities which comprise the action; and (3) coordinate these activities in a way that would normally facilitate the outcome’s occurrence.

What binds particular component actions together into larger purposive actions?  It is the fact that these actions are all parts of plans involving a single intention.
What singles out an actual or possible outcome as one to which the component actions are collectively directed?  It is the fact that this outcome is represented by the intention.

So the intention is what binds component actions together into purposive actions and links the action taken as a whole to the outcomes to which they are directed.



\section{Slide}
Now as Elisabeth Pacherie has argued \citep[pp.\ 189-90]{pacherie:2008_action} (and I’ve had a go at arguing this in joint work with Corrado Sinigaglia recently too),
motor representations are relevantly similar to intentions.

Of course motor representations differ from intentions in some important ways (as Pacherie also notes).

But they are similar in the respects that matter for explaining the purposiveness of action.
(1) Like intentions, some motor representations represent outcomes (and not merely patters of joint displacements, say).
(2) Like intentions, some motor representations play a role in coordinating multiple  component activities by virtue of their role as elements in hierarchically structured plans.
(3) And, like intentions, some motor representations coordinate these activities in a way that would normally facilitate the outcome’s occurrence.

So in the individual case, it seems to me quite straightforward that there is a role for motor representation to play in explaining the purposiveness of action [*explaining the possibility of purposive action?].

The claim is not that \emph{all} purposive actions are linked to outcomes by motor representations, just that some are.
In some cases, the purposiveness of an action is grounded in a motor representation of an outcome; in other cases it is grounded in an intention.
And of course in many cases it may be that both intention and motor representation are involved.



\section{Slide}
Now let’s turn to joint action.



\section{Slide}
The same question we asked about ordinary, individual action also arises for joint action.
What is the relation between a purposive joint action and the outcome or outcomes to which it is directed?




\section{Slide}
And again the question has two aspects.
What singles out the outcome or outcomes to which a purposive joint action is directed?



\section{Slide}
And what binds together the various activities that make up the joint action?
The difference in the case of joint action is, of course, that these activities are not necessarily activities of a single agent.




\section{Slide}
The answer to this question for the case of joint action is also superficially similar in the answer we gave in the case of ordinary, individual action.

A shared intention is what relates purposive joint actions to the outcomes to which they are directed.
For the shared intention
(1) involves a representation, on the part of each agent, of an outcome
(2) coordinates the several agents’ activities
and 
(3) coordinates the several agents’ activities in such a that would normally facilitate the occurrence of the represented outcome.

It is in this sense that a shared intention can ground the purposiveness of a joint action.




\section{Slide}
But what we saw earlier, what the research by Kourtis et al and others indicates, is that social motor representation can play a  role similar to that of shared intention.

Return to the example of two agents moving an object in a way that involves passing it between them.
Suppose that their passing involves reciprocal agent-neutral motor representations of the outcome, which is the movement of the object. 
These motor representations
(1) represent an outcome to which the joint action is directed,
(2) coordinate the several agents’ activities
and 
(3) coordinate the several agents’ activities in a way that would normally facilitate the occurrence of the represented outcome.
Because each agent represents the whole movement and plans all of its implementation irrespective of which parts she will actually perform, each agent plans the action in a way that should coordinate with the other agent’s plans providing they use similar planning procedures

[*What I’m saying here, in effect, is that both shared intention and social motor representation can yield a COLLECTIVE GOAL]



\section{Slide}
What I’m suggesting is very simple.

If you think that in ordinary, individual action, the purposiveness of actions can be grounded by motor representations
(and you should think this because it’s true),
then you should also think the same about actions involving two or more agents---the purposiveness of a joint action can be grounded in motor representations as well as in shared intentions.



\section{Slide}
Let me try another way of presenting the same idea.
Here are Michael Bratman’s sufficient conditions for shared intention.
I want to suggest that social motor representation provides a parallel



\section{Slide}
There is a direct parallel with the first condition: in the case of motor representation, each agent represents the outcome (e.g. the movement of the object).
The key claim here is that some motor representations (i) represent outcomes, and (ii) represent the outcomes of actions not all of whose components will be executed by the agent whose motor representation it is.



\section{Slide}
Here there is clearly no \emph{direct} parallel. 
I don’t think motor representations can represent motor representations in the way that intentions can represent intentions.
But I do think there is a parallel of sorts.
Each agent’s having a motor representation of the distributed goal of their action does ensure meshing of subplans.
What ensures this meshing is not the fact that each agent represents the other's plans \emph{as the other's plans}.
Rather in the case where joint action is grounded in social motor representation, what ensures meshing of subplans is two facts (i) each agent plans all of the agents’ actions, and (ii) the agents rely on similar planning strategies (planning strategies that are sufficiently similar to ensure meshing subplans).


\section{Slide}
I’m less sure about a parallel to the common knowledge condition.
As I see things, the justification for supposing that shared intention involves common knowledge concerns a normative link between intention and reasons.
In acting on intentions, one should be acting for reasons.
And a consideration can only be among your reasons if you know that consideration.
So I think the need for common knowledge arises from the need to explain how another person’s intentions could be among your reasons for acting.
I don’t think this need arises in the case of motor representation because it seems to me that the sort of planning of which motor representation is an element does not involve acting for reasons in the same sense.  
(In motor action, there are reasons why we do things (of course!) but these are not  reasons for which we act.)
What motor joint action requires is not that your motor plans provide reasons for mine.
There just has to be a good chance that this is true relative to the costs and benefits of joint action and the alternatives to joint action.
So \textbf{I think that instead of common \emph{knowledge}, in the case of social motor representation there is a common \emph{background} of dispositions, habits and expectations.}%
\footnote{
[***CUT but one thing that might do the work of common knowledge is a custom or habit 
that would allow the agents, in their particular social context, to rely on each other’s cooperation.
In some countries this sort of thing works on public transport; 
it is reasonable to take for granted that, if you are obviously struggling with a pram or suitcase, then someone nearby will help.]
}



\section{Slide}
If this is right, if social motor representations play a role analogous to the structure of intentions and knowledge which Bratman identifies as sufficient for shared intention, then this is another reason to think that motor representations can ground the purposiveness of joint action.



\section{Slide}
Recap: the question was: Does social motor representation  play a role in explaining what joint is?
I have just been arguing for a positive answer.
My thesis is this:
\textbf{Reciprocal agent-neutral motor representations coordinate multiple agents’ actions around an outcome in part by virtue of representing that outcome.}
That is, reciprocal social motor representations can ground the purposiveness of joint action.
This is why I think that fully understanding what joint action is requires understanding the coordinating role of social motor representation and not only understanding shared intention.

I don’t mean to suggest that all joint actions involve social motor representation.  Surely some joint actions do not. 

But, equally, \emph{there could be purposive joint actions which do not involve shared intentions},
just as there can be purposive actions which do not involve intentions at all.
Let me go slowly in explaining why I think there could be joint action without shared intention and start by returning to the case of ordinary, individual action.




\section{Slide}
So what are intentions for?

I’m going to assume that intentions are something over and above basic beliefs and desires; that an intention is not, for instance, merely a strongest desire or, as Donald Davidson held at an early stage of his thinking, merely a belief-desire pair.
Intention involves more than this.

There is a temptation to assume that intention is involved in every case of purposive action.
But it’s hard to see what the argument for this assumption could be.
In many cases it seems that beliefs, desires and motor representations are all that is needed to explain purposive action. 
You offer me a biscuit.  I want one, and I believe I can get one by reaching out for it.  So I do reach for it.  As far as I can see, there’s no need to suppose that, in addition to the belief and desire, it must be the case that I also intend to take a biscuit.  
(At least not unless we take ‘intention’ to mean ‘strongest desire’, which it does not.)  
Maybe I do intend this.  
But it’s possible for an agent to take and eat a biscuit, and to do so purposively, without having any intentions at all.  
Beliefs, desires and motor representations are sufficient.

So if we don't need intentions merely to perform a purposive action, what are intentions for?



\section{Slide}
This question becomes more pressing if you consider that motor representations enable quite sophisticated planning over short periods of time and sequences of action; for example, how you grasp a pointer will depend on what you are about to do with it \citep{zhang:2007_planning}.

This sort of planning does not need intentions at all.  So (again) what are intentions for?



\section{Slide}
Michael Bratman suggests that 
\emph{Intentions are for planning multiple separate actions over longer periods of time; and for planning multiple separate actions whose execution is mutually constraining where the outcomes cannot be represented motorically.}

This is a case where intentions are really needed [see figure in slide] --- here one can’t act on strongest desire (for the big reward) if want to maximise rewards by collecting the small and the large reward.
And one can’t rely on motor representation because the motor system doesn’t care about things that cannot be represented in motor terms.

I don't think, of course, that intentions are only involved in actions which require planning of this sort.
But I do think it's only in such actions that absolutely require intentions.

Not all purposive actions involve any planning of this sort. 

Now you could imagine a two-person version of this task where we are rewarded for what we collectively achieve.  In this case it’s optimal if one of us goes for the small reward and the other goes for the large reward.  I think it’s this kind of planning that shared intention is really for.



\section{Slide}
By contrast, in many ordinary cases of joint action there is no need for planning of this sort and so no need for shared intention.  Actions such as these \emph{might} involve shared intention but they do not \emph{necessarily} involve shared intention.

I’m suggesting that some joint actions---like the one two people move an object in a way that involves passing it between them---don’t require this kind of planning and so don’t necessarily involve shared intentions.
In some cases, social motor representation alone is sufficient for purposive joint action.



\section{Slide}
This goes against a widely shared view in the literature



\section{Slide}
To focus on just one case, Alonso says ‘the key property of joint action lies in its internal component [...] in the participants’ having a “collective” or “shared” intention.’ \citep[pp.\ 444-5]{alonso_shared_2009}

I want to spend a bit of time on this claim because it matters for my second theme, which is a problem concerning how motor representation and shared intention interface in joint action.

As far as I know, this claim is not explicitly argued for.  From conversation with philosophers I think the central argument hinges on  contrast cases.
Contrast cases are pairs of actions which are similar in terms of behaviour and coordination but where one is joint but the other isn’t....



\section{Slide}
[nothing]


\section{Slide}
[Gilbert's Jack and Sue walking together case]


\section{Slide}
[Searle's park vistor contrast case]

These sorts of contrast case invite the question, 
How do joint actions differ from individual actions which may occur in parallel? 
What is the difference between Jack and Sue walking together and their walking side-by-side?  
Gilbert’s example shows that the difference can’t just be a matter of coordination, because people merely walking alongside each other also need to coordinate their actions in order to avoid colliding with each other.  
And Searle’s example shows that the difference between joint action and parallel individual action can’t just be that the actions have a common effect because merely parallel actions can have common effects too. 

How might this lead someone to think that all joint action involves shared intention?
I think the idea is supposed to be this.
One difference between the genuinely joint actions and their merely parallel counterparts is that the genuinely joint actions involve shared intentions.
And there is no further difference that enables one systematically to distinguish the two cases.

I think this argument is mistaken for two reasons.


\section{Slide}
First because there are contrast cases in which the joint action does not involve shared intention.
For instance, contrast our lifting a sofa together with us each individually raising either end of the sofa, coincidentally at the same time.
The former is a joint action whereas the latter is arguably not.
But I don't think it's right to assume that lifting a sofa together necessarily involves shared intention (although our \emph{buying} a sofa together might).
So it seems to me that the contrast cases give us no reason at all to suppose that all joint actions involve shared intention.

In fact the existence of cases like the sofa indicate that appeal to shared intention does not provide a sufficiently general way of explaining the difference illustrated by the contrast cases.

This brings me to a second consideration.
There is a more general notion appeal to which enables us to distinguish the contrast cases.
Genuine joint action differs from merely parallel action because the former involves each agent’s representing an outcome to which all of their actions are directed where these outcome representations coordinate their actions in a way that would normally facilitate their collective success in bring about this outcome.



\section{Slide}
(Actually I don’t think this is quite general enough because I think some joint actions involve non-representational coordinative structures only; I am also doubtful that there is a sharp distinction between merely parallel and genuinely joint action and I think it is possible to see the difference as a matter of degree. But I don’t want to get into that here.  It’s sufficient that we have moved away from the bare shared intention account. [*WHAT WE REALLY NEED IS A COLLECTIVE OR SHARED GOAL, and the possibility of gradual construction shows that there’s no magic moment separating joint from parallel action.)

To sum up so far, I reject the claim that reflection on the contrast cases provides any support for the idea that all joint action involves shared intention.



\section{Slide}
So far I have been assuming that there is an inconsistency between two claims:
(1) my claim that reciprocal  agent-neutral motor representation can ground purposive joint action
(2) many philosophers’ claim that all joint actions involve shared intention.

But maybe it is a mistake to think that these are inconsistent.
Why not suppose that some social motor representations are  shared intentions?




\section{Slide}
If some reciprocal agent-neutral motor representations are shared intentions, then there is no inconsistency.
And what I’ve just been arguing is that agent-neutral motor representations resemble shared intentions in that both play a role in coordinating agents’ actions by virtue of representing outcomes.  Isn’t that enough to justify identifying them as shared intentions?

This issue might easily seem narrowly conceptual or terminological.  
At the end of the day it doesn’t much matter if we want to call some motor representations ‘shared intentions’.  
After all, on most accounts shared intentions are neither shared nor intentions so we would hardly be doing more violence to the term than is already being done.  
So insofar as labeling some social motor representations shared intentions might help to avoid unnecessarily philosophical discussion, I’m all in favour.
However, there is an important difference between the states normally regarded as  shared intentions and any motor representations.
And this difference matters for understanding the interface between shared intention and motor representation.


[*PLAN for what follows:*]

Difference in format.
Leads to the interface problem.  
Shared intentions can be inferentially integrated with other shared intentions; but not they cannot be inferentially integrated with social motor intentions (two disjoint planning processes).  
Elisabeth Pacherie’s proposal: shared intentions set outcomes to be achieved by social motor representations.  
I borrow this idea from her, but it raises a further problem.
The problem is how the one sets outcomes for the other given the difference in representational format.


\section{Slide}
In what respects do reciprocal agent-neutral motor representations differ from shared intentions?


\section{Slide}
As background we first need a generic distinction between content and format. Imagine you are in an unfamiliar city and are trying to get to the central station. A stranger offers you two routes. Each route could be represented by a distinct line on a paper map. The difference between the two lines is a difference in content. 


\section{Slide}
Each of the routes could alternatively have been represented by a distinct series of instructions written on the same piece of paper; these cartographic and propositional representations differ in format. 

Format matters because only where two representations have the same format can they be straightforwardly inferentially integrated.



\section{Slide}
To illustrate, let’s stay with representations of routes.  
Suppose you are given some verbal instructions describing a route. You are then shown a representation of a route on a map and asked whether this is the same route that was verbally described. You are not allowed to find out by following the routes or by imagining following them. 
Special cases aside, answering the question will involve a process of translation because two distinct representational formats are involved, propositional and cartographic. It is not be enough that you could follow either representation of the route. You will also need to be able to translate from at least one representational format into at least one other format. 



\section{Slide}
This brings me to the argument ...

\begin{enumerate}
\item Only representations with a common format can be inferentially integrated.

\item Any two intentions can be inferentially integrated in practical reasoning.

\item My intention that I visit Paris on Friday is a propositional attitude.
\end{enumerate}


\section{Slide}
It follows that 
All intentions are propositional attitudes



\section{Slide}
But it is also widely agreed that:
No motor representations are propositional attitudes.



\section{Slide}
So we can conclude that:
 No motor representations are intentions



\section{Slide}
This step---Any two intentions can be inferentially integrated in practical reasoning---is questionable.  I don’t have an argument for this and I’m not sure it isn’t terminological.  
What I care about is that we distinguish attitudes according to the processes in which they feature.
So if you like we could distinguish two kinds of intention, one propositional the other motor.
As long as we distinguish representations of different formats I don’t see that it matters too much whether we call them all intentions or whether we use that term for only some of them.




\section{Slide}
So where does this leave us?
The question was whether reciprocal agent-neutral motor intentions could count as shared intentions.




\section{Slide}
In answer to that question, I think this:
(a)
IF you agree motor representations are not intentions,
THEN reciprocal agent-neutral motor representations are not shared intentions.

And
(b)
IF motor representations are a non-propositional variety of intention,
THEN reciprocal agent-neutral motor representations are a non-standard variety of shared intentions.

The key thing is that, either way, reciprocal agent-neutral motor representations cannot be inferentially integrated with shared intentions in practical reasoning.
This leads to what I’ll call ‘The Interface Problem’


\section{Slide}
It will take me a moment to explain what the problem is.


\section{Slide}
The first step is to note that:
Some joint actions involve both shared  intention and reciprocal agent-neutral motor representation.
Imagine two people setting up a tent in a gale together, for example.
It seems success will often require many ingredients including shared intention for the large-scale planning and reciprocal agent-neutral motor representation for passing objects and bending the poles together.



\section{Slide}
Earlier I argued that reciprocal agent-neutral motor representations:

\begin{itemize}
\item represent outcomes;

\item ground the purposiveness of some 
joint actions; and
\end{itemize}


\section{Slide}
This means, I think, that in at least some cases of effective joint action there will be a certain kind of harmony between the contents of 
 reciprocal agent-neutral motor representations
 and the contents of 
 shared intentions.
Further, I think it's probable that, in some cases, the harmony is non-accidental.
 
To be clearer about what `harmony' means, we need the notion of matching:
\begin{quote}
Two  outcomes, A and B, \emph{match} in a particular context just if, in that context, either the occurrence of A would normally constitute or cause, at least partially, the occurrence of B or vice versa. 
\end{quote}
In some cases of joint action, the outcomes each kind of representation specifies non-accidentally match.



\section{Slide}
This leads to a question: how are non-accidental matches between the outcomes specified by shared intentions and by reciprocal agent-neutral motor representations possible?

A natural suggestion is that matching is achieved through practical reasoning.
It is in this way that the contents of shared intentions could partially determine the contents of reciprocal agent-neutral motor representations.
But there is \textbf{an obstacle} to this idea.


\section{Slide}
For as we saw earlier, reciprocal agent-neutral motor representations
\begin{itemize}
\item differ in format from (the constituent attitudes of) shared intentions.
\end{itemize}

Given that only representations with a common format can be inferentially integrated in practical reasoning,
we cannot appeal to reasoning to explain how the outcomes specified by shared intentions match the outcomes specified by reciprocal agent-neutral motor representations.
This is why the interface problem is a problem.


\section{Slide}
In conclusion,
I have suggested that 
reciprocal agent-neutral motor representations
can 
ground purposive joint action
for they
can coordinate two or more agents' actions in virtue of representing a single outcome to which each agent's actions are directed.

I also suggested, further, that
reciprocal agent-neutral motor representation
and
shared intention  
have distinctive roles in explaining the purposiveness of joint action.

And this suggestion has two consequences.
One is that we cannot properly understand what joint action is if we focus only on shared intention.
The other is that we face a challenge.
The challenge is to explain how these two kinds of representation could sometimes harmoniously contribute to effective joint action despite not being inferentially integrated in practical reasoning.


\bibliography{$HOME/endnote/phd_biblio}

\end{document}