%!TEX TS-program = xelatex
%!TEX encoding = UTF-8 Unicode

\def \papersize {a4paper}

\documentclass[12pt,\papersize]{extarticle}
% extarticle is like article but can handle 8pt, 9pt, 10pt, 11pt, 12pt, 14pt, 17pt, and 20pt text

\def \ititle {Intention and motor representation:}
\def \isubtitle {how to resist temptation}
\def \iauthor {S. Butterfill* \& C. Sinigaglia**
\\ 
**Department of Philosophy, University of Warwick
\\ 
***Dipartimento di Filosofia, Università degli Studi di Milano}
\def \iemail{s.butterfill@warwick.ac.uk}
%\date{}

\input{$HOME/Documents/submissions/preamble_steve_paper2}
%\author{}
%\date{}

%\setromanfont[Mapping=tex-text]{Sabon LT Std} 

\begin{document}

\setlength\footnotesep{1em}

\bibliographystyle{newapa} %apalike

\maketitle
%\tableofcontents
\title{}

\tolerance=5000


\section{A motor format for representation}

The second temptation is to suppose that motor representation, or some species of it, is a variety of intention, where intention is understood in the standard way as a propositional attitude with a characteristic role in practical reasoning \citep{Bratman:1987xw}. In this section we explain why this temptation should also be resisted. 

As background we first need a generic distinction between content and format. Imagine you are in an unfamiliar city and are trying to get to the central station. A stranger offers you two routes. Each route could be represented by a distinct line on a paper map. The difference between the two lines is a difference in content. Each of the routes could alternatively have been represented by a distinct series of instructions written on the same piece of paper; these cartographic and propositional representations differ in format. The format of a representation constrains its possible contents. For example, a representation with a cartographic format cannot represent what is represented by sentences such as `There could not be a mountain whose summit is inaccessible.'\footnote{Note that the distinction between content and format is orthogonal to issues about representational medium. The maps in our illustration may be paper map or electronic maps, and the instructions may be spoken, signed or written. This difference is one of medium.} The distinction between content and format is necessary because, as our illustration shows, each can be varied independently of the other.

Our aim in this section is to show that motor representations differ from intentions with respect to their format. This is why the second temptation must be resisted. That motor representations differ in format from intentions shows that they are genuinely distinct phenomena.\footnote{ Readers already convinced that motor representation differs from intention in being non-conceptual will not need the following argument in order to conclude that they are distinct phenomena. However the following considerations also indicate that motor and perceptual representations differ in format, which will be relevant later when we consider how motor representations and intentions jointly lead to action.}

How in general can we identify or distinguish representational formats? Because a representational format is typically associated with characteristic processing profiles, it is sometimes possible to infer similarities and differences in representational format from similarities and differences in the processes in which representations feature. This works both for artefactual and mental representations.  To illustrate in terms of our earlier example, suppose that two people have representations of the same route but for one person the route is represented by a line on a map (so in a cartographic format) whereas the other person has a propositional representation of the route. Some transformations are likely to be easier for the person with the cartographic route representation (depending on the projection used, of course); examples include reversing the route, determining how many times a certain river is crossed and transforming the route into a sequence of compass bearings. Other transformations, such as turning the route into a list of salient landmarks, may be easier for the person with the propositional route description. So some patterns of difference in the two people's performances may be explained by the difference in the formats of their representations. If we did not already know that the two people's route representations involved different formats, we might infer this from the facility with which each performed various transformations of the route. 

Cognitive neuroscience frequently depends on inferences of just this type. To illustrate, compare imagining seeing an object moving and with actually seeing it move. For this comparison we need to distinguish two ways of imagining seeing. There is a way of imagining seeing which phenomenologically is something like seeing except that it does not necessarily involve being receptive to stimuli. This way of imagining seeing, sometimes called 'sensory imagining', is commonly distinguished from more cognitive ways of imagining seeing which might for example involve thinking about seeing \citep[§2.1]{Gendler:2011_imagination}.%
\footnote{Note that we define sensory imagining in terms of phenomenology and make no stipulations about the processes and representations involved.  This is essential for our purposes, since we wish to consider how evidence bears on a conjecture about the format of representations involved in sensory imagining.}
It is sensorily imagining seeing an object move that we wish to compare with actually seeing an object move.  These two share some processing characteristics.  For instance, whether an object can be seen all at once depends on its size and distance from the perceiver; strikingly, when subjects imagine seeing an object, whether they can imagine seeing it all at once depends in the same way on size and distance (\citealp{kosslyn:1978_measuring}; \citealp[p.\ 99ff]{kosslyn:1994_image}.  Also, how long it takes to imagine looking over an object depends on the object's subjective size in the same way that how long it would take to actually look over that object would depend on its subjective size \citep{kosslyn:1978_visual}.%
\footnote{These and further examples are discussed by \citet[p.\ 165]{currie:1997_mental}.} Further, imagining seeing something (for example, imagining seeing a visual mask) can modulate and interfere with actually seeing in much the way that actually seeing the thing imagined would \citep{pearson:2008_functional,ishai:1995_common}. The similarities in processing profile and the particular patterns of interference are good (if non-decisive) reasons to conjecture that imagining seeing and actually seeing involve representations with a common format.  This conjecture is indirectly supported by evidence that sensorily imagining seeing and actually seeing not only have a common neural basis but also involve similar patterns of cortical activation \citep[e.g.][]{page:2011_erp}. 

Let us turn to motor representation. Compare imagining moving a ball with actually moving a ball.  To fully specify the comparison we intend, it is again necessary to distinguish two ways of imagining. One way of imagining action is phenomenologically something like acting except that such imaginings are not necessarily responsive to the features of actual objects and do not necessarily result in bodily movements. To illustrate suppose you are about to dive into a pool and, standing at the edge, mentally pantomime launching yourself from the bank. Some phenomenal characteristics of this imaginative exercise may be barely distinguishable from some of those associated with actually launching yourself into the pool. This way of imagining action can be distinguished from more cognitive ways of imagining action which might involve thinking about an action.% 
\footnote{
On distinguishing these two ways of imagining action, see \citet[p.\ 161]{currie:1997_mental}, \citet[p.\ 727]{jeannerod:1995_mental}, and \citet[p.\ 638-9]{kosslyn:2001_neural}. The former, phenomenologically action-like imagining is sometimes labelled 'motor' or 'internal' and occasionally identified by its links to motor processes or by features of the format or content of the representations involved (\citealp[p.\ 1400]{annett:1995_motor}). We avoid these labels because we introduce the distinction by appeal to phenomenology only and do not stipulate that motor representations are involved. It is essential for what follows that the involvement of motor representations in the phenomenologically action-like way of imagining action is a discovery rather than a terminological stipulation.}
%
The comparison we intend is between imagining moving a ball in the former, phenomenologically-action like way and actually moving a ball. There is evidence that the way imagining performing an action unfolds in time is similar in some respects to the way actually performing an action of the same type would unfold.  For instance, how long it takes to imagine moving an object is closely related to how long it would take to actually move that object \citep{decety:1989_timing, decety:1996_imagined, Jeannerod:1994oz}. In addition, for object-related actions such as grasping the handle of a cup, manipulating the object in ways that make the action harder (such as orienting the handle to make it less convenient for you to grasp) make a corresponding difference to the effort involved in imagining performing the action \citep{parsons:1994_temporal, frak:2001_orientation}. Further, imagining performing an action can selectively interfere with performance of a related action. For example, suppose you are faced with an array of objects one of which---the target---you will shortly be required to grasp.  Subjects who imagine grasping an object other than the target object tend to be slower in subsequently grasping the target object than subjects who do not imagine acting or subjects who imagine grasping the target object \citep{ramsey:2010_incongruent_}. Just as the similarities between imagining seeing and actually seeing are evidence for the hypothesis that the representations involved in imagining seeing and actually seeing have a common format, so also the similarities in processing profile between imagining acting and actually acting, together with the particular patterns of interference between the two, suggest that imagining acting and actually acting involve a common representational format. And also much as in the case of seeing and imagining seeing, acting and imagining acting involve many of the same processes up to the actual muscle contractions \citep{jeannerod:1995_mental, jeannerod:2003_mechanism}. 

The fact that motor representation is involved in one way of imagining acting means that we can use imagining acting to investigate the format of motor representations. This enables us to examine characteristics of processes involving motor representation which are are not a direct consequence of limits on agents' abilities to act.

Differences in processing profile provide reasons to distinguish the formats of visual and motor representations. To see why, contrast the way of imagining seeing that is phenomenologically something like seeing with the way of imagining acting that is phenomenologically something like acting. To make things concrete, contrast imagining seeing a ball rotating with imagining rotating a ball.%
\footnote{It may be difficult in practice to imagine acting without also imagining seeing, and conversely; it may also sometimes be difficult to distinguish imagine acting from imagining seeing \citep[as][p.\ 170 note]{currie:1997_mental}. However ordinary subjects can separate the two well enough to confirm predictions about their differences \citep[see, e.g.,][]{kosslyn:2001_imagining}.
}
We know that the way imagining acting unfolds in time is quite different from the way imagining seeing unfolds because, as already mentioned, how quickly the former can be done is a function of how long it would take the agent to rotate the ball, whereas how quickly the latter can be done depends on how rapidly the ball can rotate and still be perceived as rotating. Similarly, we mentioned that factors making actually acting more effortful also make imagining acting more effortful: such factors are unlikely to affect the effort involved in imagining seeing. The explanation for these differences in characteristic processing profile between imagining acting and imagining seeing plausibly involves a difference in the formats of the representations involved. That the two involve different representational formats is consistent with  evidence that these two imaginings also involve different processes \citep{kosslyn:2001_imagining}. For instance, each can be selectively impaired \citep{sirigu:2011_motor}; and factors such as limb amputation or hand posture can selectively interfere with imagining performing an action \citep{nico:2004_left, vargas:2004s_influence,  fourkas:2006_influence}. 

But are we too hasty in drawing this conclusion? It might be supposed that imagining seeing a ball rotating differs from imagining moving a ball with respect to the contents involved---after all, one involves an action, the other does not. In principle someone might attempt to explain differences in characteristic processing profile between imagining seeing a ball move and imagining moving the ball by appeal to differences in content rather than in format. We do not dispute that \emph{some} differences might be explained in this way. But could any differences in content fully explain differences in characteristic processing profile? To see why not, consider a further case. Ordinary subjects who are asked to judged the laterality of a hand rotated to various degrees are slower and less accurate when the hand's position is maximally biomechanically awkward (rotated 180 degrees, middle finger pointing down). By contrast, no such effect occurs for comparable tasks involving letters rather than hands. This difference in performance has been explained by appeal to the hypothesis that the hand task involves the phenomenologically action-like form of imagination whereas the letter task involves sensory visual imagination (**ref).  But could this performance difference instead be explained by a difference in content (one task involves representing hands, the other letters)? Subjects with Amyotrophic Lateral Sclerosis (ALS) which impairs motor representation retain the ability to judge the laterality of hands (and letters) but their performance is unaffected by whether the hand is presented in a biomechanically awkward position \citep{Fiori:2012fk}. The difference between these subjects and ordinary, control subjects is not plausibly a matter of content since both are representing rotated hands; rather, the difference in the processing profile of their laterality judgements is plausibly a reflection of a difference in the formats in which they represent hands. This is why we hold that some differences between imagining seeing and imagining acting indicate differences in representational format. 

-----------------------------

Recall that our aim in this section is to show that motor representations differ from intentions with respect to their format.  
\bibliography{$HOME/endnote/phd_biblio}

\end{document}