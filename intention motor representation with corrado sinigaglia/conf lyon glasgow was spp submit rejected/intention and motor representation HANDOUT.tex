%!TEX TS-program = xelatex
%!TEX encoding = UTF-8 Unicode

\documentclass[11pt]{extarticle}
% extarticle is like article but can handle 8pt, 9pt, 10pt, 11pt, 12pt, 14pt, 17pt, and 20pt text

\def \ititle {Intention and Motor Representation in Action Explanation}
\def \isubtitle {}
\def \iauthor {Stephen A. Butterfill}
\def \iemail{s.butterfill@warwick.ac.uk}
\date{}

%for strikethrough
\usepackage[normalem]{ulem}

\input{$HOME/Documents/submissions/preamble_steve_handout}

\bibpunct{}{}{,}{s}{}{,}  %use superscript TICS style bib
%remove hanging indent for TICS style bib
%TODO doesnt work
\setlength{\bibhang}{0em}
%\setlength{\bibsep}{0.5em}


%itemize bullet should be dash
\renewcommand{\labelitemi}{$-$}

\begin{document}

\begin{multicols}{3}

\setlength\footnotesep{1em}


\bibliographystyle{newapa} %apalike

%\maketitle
%\tableofcontents



\

\begin{center}
{\Large
\textbf{Intention and Motor Representation in Action Explanation}
}


Stephen A. Butterfill \& Corrado Sinigaglia

<s.butterfill@warwick.ac.uk>

\end{center}



%\textbf{Abstract}
%Are there distinctive roles for intention and motor representation in explaining the purposiveness of action? Standard accounts of action assign a role to intention but are silent on motor representation. The temptation is to suppose that nothing need be said here because motor representation is either only an enabling condition for purposive action or else merely a variety of intention. This paper provides reasons for resisting temptation. Some motor representations, like intentions, coordinate actions in virtue of representing outcomes; but, unlike intentions, motor representations cannot feature as premises or conclusions in practical reasoning.  This implies that motor representation has a distinctive role in explaining the purposiveness of action. It also gives rise to a problem: were the roles of intention and motor representation entirely independent, this would impair effective action. Resisting temptation therefore requires explaining how intentions interlock with motor representations. The solution, we argue, is to recognise that the contents of intention can be partially determined by the contents of motor representations. Understanding this content-determining relation enables better understanding how intentions are related to actions.

\section{The Possibility of Purposive Action}
What is the relation between a purposive joint action and the outcome or outcomes to which it is directed?


\section{In outline}
[A] Like intentions, motor representations (i) represent outcomes, (ii) coordinate actions and (iii) do so in ways that would normally facilitate the occurrence of the represented outcomes.

[B] Unlike intentions, motor representations do not have a propositional format. 

[C] Motor representation has a distinctive role in explaining the purposiveness of action (from [A] and [B]).

[D] There is a problem about how intention and motor representation can be harmoniously integrated (`The Interface Problem').

[E] The solution to this problem involves constituents of intentions which refer to actions by deferring to motor representations.

[F] Maybe where an intention properly and reliably produces bodily movement, either acting on that intention involves a further intention or else the intention involves concepts which refer to actions by deferring to motor representations.

[G] Intentionally acting in the world may ultimately depend on motor experience.



\section{Motor representation can ground purposive action (§2.A)}

Motor representations carry information about outcomes.

Action planning depends on information about outcomes.

Therefore, motor representations represent outcomes.


\section{Motor representations aren't intentions (§2.B)}
\begin{enumerate}
\item Only representations with a common format can be inferentially integrated.

\item Any two intentions can be inferentially integrated in practical reasoning.

\item My intention that I visit Paris on Friday is a propositional attitude.

\item All intentions are propositional attitudes (from 1--3).

\item No motor representations are propositional attitudes.

\item No motor representations are intentions (from 4, 5).
\end{enumerate}





\section{The Interface Problem (§2.D)}
Two  outcomes, A and B, \emph{match} in a particular context just if, in that context, either the occurrence of A would normally constitute or cause, at least partially, the occurrence of B or vice versa. 

Two representations of outcomes are \emph{in harmony} in a particular context if the outcomes they represent match in that context.

Some actions involve both  intention and motor representation.

In some cases, an intention and a motor representation are non-accidentally in harmony.

How is non-accidental harmony ever possible?

A natural way to answer this question would be by appeal to a process of planning or practical reasoning.
But intention and motor representation cannot be inferentially integrated (because they differ in format).  


\section{Demonstrative action concepts (§2.E)}
There are demonstrative concepts of actions.

A demonstrative concept can refer to an action by deferring to a pantomime of an action.

A demonstrative concept can refer to an action by deferring to a \emph{mental} pantomime of an action.

Experiences involved in actually acting, like those involved in mentally pantomiming an action, can ground the possibility of demonstrative reference to action by deference to motor representation.

Some concepts are constituents of intentions and refer to actions by deferring to motor representations.



\end{multicols}

\end{document}