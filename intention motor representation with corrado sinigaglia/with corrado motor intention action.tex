%!TEX TS-program = xelatex
%!TEX encoding = UTF-8 Unicode

\documentclass[12pt,a4paper]{extarticle}
% extarticle is like article but can handle 8pt, 9pt, 10pt, 11pt, 12pt, 14pt, 17pt, and 20pt text

\def \ititle {Grasping a cup of coffee}
\def \isubtitle { }
\def \iauthor {}
\def \iemail{}
%\date{}

\input{$HOME/Documents/submissions/preamble_steve_paper}

\begin{document}

\setlength\footnotesep{1em}

\bibliographystyle{newapa} %apalike

\maketitle
%\tableofcontents


\begin{quote}
`our primitive actions, the ones we do not by doing something else, mere movements of the  body---these are all the actions there are.
We never do more than move our bodies.'
\citep[p.\ 59]{Davidson:1971fz}
\end{quote}

\section{Claims and terms}
\begin{enumerate}
\item A \emph{motor goal} is an action outcome that can be represented by the motor system as such.  For example, a motor goal might be to grasp the object there by its handle.

\item A \emph{motor-like act} is a unit of action corresponding to an act described by a motor goal.  For example, grasping the object there by its handle may be a motor-like act.

\item \emph{Motor-like acts are the basic elements in practical reasoning}

(i) correspondence---Motor-like acts are the basic units of action for practical reasoning in this sense: where acting on an intention involves bodily movement, acting on the intention requires agents to intend to perform motor-like acts which will realise the intention. (ii) representation---humans and other primates typically represent these basic units of action not as raw bodily movements but as activities with a goal.

Note: How to understand the relation between a description of an action and a description of an action in terms of motor goals?
What does one need to be able to represent the action in non-motor terms given that the agent is able to represent the action in motor terms?
Can a monkey plan in Bratman's sense of planning?  

\item \emph{Motor-like acts are the basic elements in perceiving action}

(i) correspondence---Motor-like acts are the basic units of action for perceiving and understanding action in approximately the sense that physical objects are the basic units for perceiving.  (ii) representation---these basic units of action in terms of which actions are perceived and understood are not represented as raw bodily movements; they are represented as directed to objects and as having a goal.

(This conflicts with Csibra \& Gergeley on teleological stance as essence of action representation (although not that efficiency is entirely irrelevant, just that it's not basic).)


\item Relation to knowledge of own action (see O'Brien)


\end{enumerate}



\section{Problems}
\begin{enumerate}

\item encoding intentions for the motor system: how could practical reasoning result in bodily movements?  
Since agents often lack detailed knowledge of the bodily movements required for an action (such as the sequence of finger movements needed to grasp and post a letter), their intentions cannot directly specify the bodily movements.  
Equally, since motor cognition does not involve representations of actions under descriptions which are significant for practical reasoning (paying a bill, walking home, impressing Leslie), these cannot be transformed into bodily movements by the motor system.

\item parameters: how is it possible to rapidly identify perceived actions (e.g.\ a hand grasping a cup behind a screen) given that, construed as the movement of an articulated object, there are so many parameters to track?

\item imagination: How is it possible to imagine an action such as a hand grasping a cup behind a screen given that when construed as the movement of an articulated object this is analogous to imagining the individual movements of bees in a swarm?

\item segmentation: How is it possible to segment  actions into intention-relevant units, and to recover some of the structure of intentional action?  How are observers able to distinguish which movements are part of an action and which are extraneous to it?

\end{enumerate}



\section{*TODO}

Currently phrased in terms of motor goal is least specific representation.
Important that motor system represents actions at different levels of specificity.
From the point of view of the perceiver or agent, the representation of the action might be a composite of these many levels.
The number of parameters in terms of which an action can be described depends on the concurrent multi-level description of the action.  (E.g. grasp ON; left hand grasp OFF; right hand grasp ON; gives a pattern.)
NOT just a description at the least specific level.

READ `Cognition in action. A new look at the cortical motor system'

\section{Action from the point of view of practical reasoning}

Intentions are basic elements in practical reasoning which individuate actions.
Sam intends to get a sandwich.
His knowledge that the only nearby sandwich shop will shortly close together with his intention gives him a reason to extract himself from a conversation and walk to the shop.
At the shop he orders, pays for and collects a sandwich.
The whole episode, starting with Sam's extracting himself from the conversation and ending with Sam holding the sandwich, constitutes a single intentional action.
The episode constitutes a single intentional action because a single intention is responsible for the whole episode.  
The same pattern of movements might have occurred due to the execution of two distinct intentions (if, for example, Sam had first intended just to leave the building and only later decided to get a sandwich).
The intentional action can naturally be broken into five component episodes, the extraction, the walk, the order, the payment and the collection.  
These component episodes may overlap with each other (the walk might start before the extraction has ended).  
Each component episode is likely to be an intentional action in its own right; this is at least part of what makes it natural to break the whole into these five component episodes, and what explains why it would not be natural to break the action into sequences of bodily movement each lasting for exactly three seconds (say).
Some or all of these episodes might have further components which are themselves intentional actions.
Of course, the hierarchy of intentions can only have finite depth.  So given any intentional action, there will be a set of intentional actions where none of them has proper components which are themselves intentional actions and which jointly constitute the whole intentional action.
This set of intentional actions represents the most specific description of action that is directly involved in practical reasoning.

Let a \emph{minimally intentional action} be an intention action with no proper components which are themselves intentional actions.


\section{Action from the point of view of the motor system}

One of the motor system's roles is to control the activation of muscles which produce the bodily movements that are the ultimate physical realisation of many intentional actions.
Achieving this involves representing actions at different levels of specificity.
In the course of paying for his sandwich, Sam has to grasp one of several coins resting in his left hand and pass it to the vendor.
The grasping episode is represented at several levels of specificity.
A relatively specific representation specifies that the right hand will grasp the coin with a precision grip.
A less specific representation specifies that the coin will be grasped with a precision grip but not which effector will be used.  
If Sam were using the other hand, another effector with which he can execute precision grips or a pair of pliers, the less specific representation of the grasping episode would be the same.
The least specific representation involved in motor cognition specifies that the coin will be grasped but not how it will be grasped.
Simplifying, we can think of these least specific representations as individuated by three parameters: a type of action such as tearing, grasping and pressing; zero or more targets; and features of each target (if any) relevant to executing the action such as its size, rigidity and mass.  

[This picture is complicated by representations of grasping-to-eat and grasping-to-place, which are less specific that representations of precision-grip-grasping with respect to the motor act type but more specific with respect to the motor action type.  For this reason we need to avoid talking about `\emph{the} least specific representations']

A \emph{motor goal} is an action outcome that can be represented by the motor system as such. 
To describe actions in terms of motor goals is to describe actions as the motor system might represent them.
A \emph{motor intention} is a representation of a motor goal by the motor system.
A \emph{motor act} is an action resulting from the execution of motor intention.

Two points are vital for what follows.
First, the motor system represents actions as graspings, tearings and pullings.  These representations are effector agnostic even to the extent of failing to specify whether the action will be performed with a hand, mouth or tool.  
They are highly abstract relative to both patterns of bodily movement which realise these actions and to the patterns of muscle activation responsible for producing them.
Second, the motor system does not represent actions in terms of many of the properties that are significant to agents.  An agent's intention may be to get to her office, or to his home; the motor system cares about places but it doesn't care whose office a place is.
Apparently, then, it is possible for there to be a gap between what is specified by the most specific intentions involved in practical reasoning and what is specified by the least specific motor intentions.


\section{From intentions to motor intentions}
For many intentions, their execution will involve motor actions (possible exceptions include intending to abstain from eating).  
This means that the intentions formed through processes of practical reasoning will have to cause the execution of appropriate motor intentions.
It is easy to understand how this could happen where intentions are very closely related to possible motor intentions.  
For example, an intention to grasp a coin at a particular perceived location already contains most of the information that would be needed to specify the content a motor intention: it contains the action type (grasp) and an identifier for the target (its location).
By contrast, much more is required if we compare an intention that is less closely related to any possible motor intention.
For example, there is no simple way to transform an intention to order a sandwich into motor intentions whose execution could realise the intention.

So how are intentions translated into motor intentions?  
Two possibilities:
\begin{enumerate}
\item Practical reasoning continues to generate means until it reaches intentions whose contents can be directly mapped to motor intentions.  So minimal intentional actions involve intentions whose contents are, or can be simply mapped onto, motor goals.
\item There is a process distinct from both practical reasoning and from motor cognition which normally transforms intentions into motor intentions.
\end{enumerate}
%
Against the first alternative: on the face of it, it seems that we can intend simply to pay for a sandwich without having more fine-grained intentions.

In favour of the first alternative: (a) novice actors do have to intend each motor act; (b) when conditions are not normal it also becomes necessary to intend each motor act; (c) the existence of chained motor representations (as involved in grasping-to-eat) means that experienced actors can intend whole sequences of motor acts; (d) describing an action in terms of episodes corresponding to motor acts comes naturally.  
This suggests that the intentions concerning motor acts may be present in many cases even though they are not particularly salient.

Compare the idea that in uttering a sentence it is necessary to have an intention concerning each word.
This is arguably not how it always seems when speaking, but it sometimes seems that we utter each word intentionally and this is largely how novice speakers begin (familiar phrases and songs aside).
But aside from the use of well-worn phrases, it is plausible that each word is intended.
This is why malaprop and related phenomena are so striking.
They are failures to act in accordance with an intention


Conjecture:
\begin{quote}
There is a normative requirement on intention: an intention to do something commits one to intending a course of motor-like actions which realise the intention.  (The intuitive idea was: Motor goals are the interface between deliberation and motor cognition.  This is an attempt to get part of that idea.)
\end{quote}
%


\section{Next}

Analogy between motor goals and words; the contents of intentions and sentences

Motor goals are the atoms in terms of which we perceive others' actions

Alt.: perceiving actions in terms of bodily movements.  How to characterise these?  Abstract view of body.  Just take hand and fingers.  23 parameters to track.  25 dimensional space to track movement of hand.  Description in terms of motor actions vastly reduces number of parameters that need to be tracked.  Plus there is a connection between the affordance of the object and the shape of the hand.
I.e. it's possible to represent an action very economically by identifying a small number of motor parameters.


Key point is that there is not just intentional action and continuously bodily movement.
Rather there is an (several) intermediate description.



\section{Notes}

Mele: How deeply does the action representation need to go into the motor details?  
This is just an empirical question. 
Corrado: It's not just an empirical question.  
The question has to do with the fact that we can individuate components of action which allow us to represent a goal as such.  
Non-motor goals can be described in terms of movements only if (a) we describe movement, which is practically impossible; or (b) we describe them as composites of motor goals.

Imagining acting: it's impossible to imagine grasping an object, but possible to imagine grasping it with a hand.   
Imagine that you are grasping my cup of coffee.
Motor imagery immediately characterises the action.
It's immediately obvious both that precision grip and whole-hand grip have something in common.  
E.g. You say grasp the cup.
I grasp it with a precision grip.
You say, Not in this way.
I grasp it with a whole-hand grip.
The pattern of possible changes is immediately obvious to me.
How can that be?
Correspondence between category structure implicit in motor representations of action and category structure implicit in ordinary thinking about action.
I.e. motor system representations of action have hierarchical structure in the sense that a precision grip is represented as a grasp.

Violation of expectations.


Distinguish motor repertoire from being in a position to act or interact. (...)

\bibliography{$HOME/endnote/phd_biblio}

\end{document}