%!TEX TS-program = xelatex
%!TEX encoding = UTF-8 Unicode

%a4paper or letterpaper (also used in preamble_steve_paper2
\def \papersize {letterpaper}

\documentclass[12pt,\papersize]{extarticle}
% extarticle is like article but can handle 8pt, 9pt, 10pt, 11pt, 12pt, 14pt, 17pt, and 20pt text

\def \ititle {Intention and motor representation:}
\def \isubtitle {how to resist temptation}
\def \iauthor {
%S. Butterfill* \& C. Sinigaglia**
%\\ 
%**Department of Philosophy, University of Warwick
%\\ 
%***Dipartimento di Filosofia, Università degli Studi di Milano
}
\def \iemail{
%	s.butterfill@warwick.ac.uk
}
%\date{}

%!TEX TS-program = xelatex
%!TEX encoding = UTF-8 Unicode

\title{\ititle\\\isubtitle}
\author{\iauthor\\<{\iemail}>}

\usepackage[\papersize]{geometry} % see geometry.pdf
\geometry{twoside=false}
\geometry{headsep=2em} %keep running header away from text
\geometry{footskip=1cm} %keep page numbers away from text
\geometry{top=3cm} %increase to 3.5 if use header
\geometry{left=4cm} %increase to 3.5 if use header
\geometry{right=4cm} %increase to 3.5 if use header
\geometry{textheight=22cm}

%non-xelatex
%\usepackage[T1]{fontenc}
%\usepackage{tgpagella}

%for underline
\usepackage[normalem]{ulem}

%get the font here:
% http://scripts.sil.org/CharisSILfont

\usepackage{fontspec,xunicode}
%nb do not explicitly use package xltxtra because this introduces bugs with footnote superscripting  -- perhaps because fontspec is supposed to include it anyway.
%UPDATE:  "You need to use the no-sscript option in xltxtra: \usepackage[no-sscript]{xltxtra}, this is explained in the documentation of xltxtra.  The issue is that Sabon does not contain true superscript glyphs for every character and the no-sscript option will instead use scaled regular glyphs, which is typographically inferior, but there is no other option available when using Sabon." --- http://groups.google.com/group/comp.text.tex/browse_thread/thread/19de95be2daacade
\defaultfontfeatures{Mapping=tex-text}
%\setromanfont[Mapping=tex-text]{Charis SIL} %i.e. palatino
%\setromanfont[Mapping=tex-text]{Sabon LT Std} 
%\setromanfont[Mapping=tex-text]{Dante MT Std} 
%\setromanfont[Mapping=tex-text,Ligatures={Common}]{Hoefler Text} %comes with osx
\setromanfont[Mapping=tex-text]{Linux Libertine O} 
\setsansfont[Mapping=tex-text]{Linux Biolinum O} 
\setmonofont[Scale=MatchLowercase]{Andale Mono}


%hyperlinks and pdf metadata
%TODO avoid duplication of title & author
\usepackage{hyperref}
\hypersetup{pdfborder={0 0 0}}
\hypersetup{pdfauthor={\iauthor}}
\hypersetup{pdftitle={\ititle\isubtitle}}


%handles references to labels (e.g. sections) nicely
\usepackage{varioref}

%line spacing
\usepackage{setspace}
%\onehalfspacing
%\doublespacing
\singlespacing

\usepackage{natbib}
%\usepackage[longnamesfirst]{natbib}
\setcitestyle{aysep={}}  %philosophy style: no comma between author & year

%enable notes in right margin, defaults to ugly orange boxes TODO fix
%\usepackage[textwidth=5cm]{todonotes}

%for comments
\usepackage{verbatim}

%footnotes
\usepackage[hang]{footmisc}
\setlength{\footnotemargin}{1em}
\setlength{\footnotesep}{1em}
\footnotesep 2em

%tables
\usepackage{booktabs}
\usepackage{ctable}

%section headings
\usepackage[sf]{titlesec}
%\titlespacing*{\section}{0pt}{*3}{*0.5} %reduce vertical space after header
%large headings:
%\titleformat{\section}{\LARGE\sffamily}{\thesection.}{1em}{} 
\titlelabel{\thetitle.\quad}

%captions
\usepackage[font={small,sf}, margin=0.75cm]{caption}

%lists
\usepackage{enumitem}
\newenvironment{idescription}
{ 	
	% begin code
	\begin{description}[
		labelindent=1.5\parindent,
		leftmargin=2.5\parindent
	]
}
{ 
	%end code
	\end{description}
}


%title
\usepackage{titling}
\pretitle{
	\begin{center}
	\sffamily
	\Huge
} 
\posttitle{
	\par
	\end{center}
	\vskip 0.5em
} 
\preauthor{
	\begin{center}
	\normalsize
	\lineskip 0.5em
	\begin{tabular}[t]{c}
} 
\postauthor{
	\end{tabular}
	\par
	\end{center}
}
\predate{
	\begin{center}
	\normalsize
} 
\postdate{
	\par
	\end{center}
}


\author{}
\date{}

%\setromanfont[Mapping=tex-text]{Sabon LT Std} 

\begin{document}

\setlength\footnotesep{1em}

\bibliographystyle{newapa} %apalike

\maketitle
%\tableofcontents
\title{}

\begin{abstract}
\noindent
Are there distinctive roles for intention and motor representation in explaining the purposiveness of action? Standard philosophical accounts of action assign a role to intention but are silent on motor representation. The temptation is to suppose that nothing need be said here because motor representation is either only an enabling condition for purposive action or else merely a variety of intention. Drawing on cognitive and neuroscientific research, this talk identifies reasons for resisting temptation. Some motor representations, like intentions, coordinate actions in virtue of representing outcomes; but, unlike intentions, motor representations cannot feature as premises or conclusions in practical reasoning.  This implies that motor representation has a distinctive role in explaining the purposiveness of action. It also gives rise to a problem: were the roles of intention and motor representation entirely independent, this would impair effective action. Resisting temptation therefore requires explaining how intentions interlock with motor representations. The solution, we argue, is to recognise that the contents of intention can be partially determined by the contents of motor representations. Understanding this content-determining relation enables better understanding how intention is related to action.

\

\noindent
Word count: 3731 including notes, excluding references

\end{abstract}

\tolerance=5000

\section{Introduction}

What is the relation between a purposive action and the outcome or outcomes to which it is directed? The standard way of answering this question appeals to intention, a propositional attitude which plays a characteristic role in planning and coordinating action, is linked to practical reasoning and is subject to characteristic norms (\citealp{Bratman:1987xw}).  On the standard view, an action is directed to an outcome in virtue of the action's being appropriately related to an intention which represents this outcome or some related outcome. As this view is usually expounded, the relation between actions and outcomes to which they are directed is treated as largely independent of the motor processes and representations underpinning action execution. Motor representations are usually considered either as philosophically irrelevant enabling conditions, or else as merely filling in additional details of the basic schema provided by the standard story. Our aim is to show that this is a mistake:
the standard view must be  refined and extended  to accommodate distinctive roles for intention and motor representation in explaining the purposiveness of action.




\section{Motor representations link actions to outcomes}

Twin temptations stand in our way.
The first  temptation is to suppose that motor representation has no bearing at all on our question about the relation between actions and the outcomes to which they are directed. Just as it would be an error to suppose that details of musculoskeletal structure are relevant to this question, so equally---so the temptation---it would be an error to suppose that facts about motor representation are relevant here. 

Surrendering to this temptation might be reasonable if all motor representations represented only kinematic or dynamic features of actions, such as mere joint displacements or muscle contractions. However, 
 some motor representations represent action outcomes such as grasping, tearing or throwing.\footnote{ On the notion of action outcome in motor representation, see \citet{jeannerod_motor_2006} and \citet{rizzolatti_mirrors_2008}.}
 Furthermore, as we shall go on to argue, some such representations ground purposive actions. This in outline is why the first temptation should be resisted.

Why accept that there are motor representations of action outcomes? The first step is to consider evidence that motor processes carry information about action outcomes. For any given marker of motor processing (such as a pattern of neuronal discharge or motor-evoked potentials), how can we test whether that marker carries information about action outcomes? The basic principle is straightforward: vary kinematic and dynamic features while holding constant an action outcome; and, conversely, vary action outcomes while holding kinematic and dynamic features constant. 
In order to vary kinematic and dynamic features while holding action outcomes constant, in some studies a single action outcome is achieved using different effectors, hand, mouth or foot, say \citep{rizzolatti:1988_functional,Rizzolatti:2001ug,cattaneo:2010_state-dependent}. A variation on this approach is to contrast performing a grasping action with different tools, so that the same action outcome might require closing or opening the hand depending on the tool used \citep{umilta:2008pliers,cattaneo:2009_representation,rochat:2010_responses}. In order to vary action outcome while holding kinematic and dynamic features constant, researchers have contrasted grasping movements with different distal outcomes such as eating and placing \citep{Fogassi:2005nf,bonini:2010_ventral,cattaneo:2007_impairment}. Another approach is to contrast the same grasping movements performed in the presence or manifest absence of a target object \citep{Umilta:2001zr,villiger:2010_activity}. 
%A related alternative is to contrast the same grasping movements in the presence of objects which could, or manifestly could not, be grasped by means of such movements \citep{koch:2010_resonance}. 
In each of these cases some markers of motor processing are correlated with action outcomes rather than narrowly kinematic or dynamic features of action.\footnote{ Of course some researchers have raised doubts  (e.g.\ \citealp{cavallo_grasping_2011,borroni_mirroring_2011}).  On balance, however, the evidence supports this conclusion.}


So far we have not explicitly distinguished two aspects of action outcomes.  
Action outcomes often specify both a way of acting---whether to grasp or release, say---and also what to act on---on the mug or the pen, say.  
We have been focusing on ways of acting. 
 However a range of behavioural and neurophysiological evidence shows that motor processes also carry information about objects on which actions might be performed.\footnote{
See \citet{buccino:2009_broken,costantini:2010_where,cardellicchio:2011_space,Tucker:1998,tucker:2001_potentiation}.
For a review see \citet{Gallese:2011uq}; for discussion see \citet[pp.\ 410-3]{pacherie:2000_content}. 
}



To say that motor processes involve \emph{information} about action outcomes is not, of course, to say that there are motor \emph{representations} of action outcomes.  To make the step from information to representation we have to show that information about action outcomes guides processing \citep[compare][]{Dretske:1988sq}. 

To this end  consider how information about action outcomes is relevant in motor planning, a functional role of motor representation.  Motor planning involves satisfying a variety of requirements. For example, in grasping a mug it is necessary for the hand to prefigure the shape of the mug, to move towards it avoiding potential obstacles and to reach it at a velocity that is both compatible with achieving the type of grip to be used and also suitable given features of the mug such as its fragility and weight \citep{Jeannerod:1995bb,jeannerod:1998nbo}. The need to plan sequences of actions, which may overlap, imposes further requirements. How one should reach for and then grasp a heavy frying pan (say) depends on what one will then do with it. One way of grasping it might be ideal for safely transporting its contents, another for emptying it. 
Requirements on motor planning could not normally be met by explicit practical reasoning, especially given the rapid and fluid transitions involved in many action sequences. 
Rather they require motor processes and representations.
Since 
many of the requirements to be satisfied by motor planning
are partially dependent on action outcomes and not only on  narrowly kinematic or dynamic features of action,
it would be advantageous for some  motor representations to represent action outcomes . 
Given that, as we saw, markers of motor processes carry information about action outcomes, it is now reasonable to conclude that some motor representations are representations of action outcomes.

So far we have argued that motor processes involve representations of action outcomes. It remains to show that such representations ground purposive actions. But this is hardly a further step. How do intentions ground the purposiveness of actions?  On any standard view, an intention represents an outcome, causes an action, and does so in such a way as to facilitate the outcome's occurrence. Similarly, some motor representations  represent action outcomes, play a role in generating actions, and do this in a way that normally facilitates the occurrence of the outcomes represented. To say that motor representations do all this is one way of making precise the metaphor involved in saying that purposive actions are directed to outcomes.  

To sum up so far, the first temptation was to suppose that motor representation has no bearing at all on our question about the relation between actions and the outcomes to which they are directed. 
This temptation must be resisted because
motor representations represent action outcomes and ground the directedness of actions to outcomes.  
This fact about motor representation is why a second, complementary temptation is appealing ...

\section{A motor format for representation}

The second temptation is to suppose that motor representation, or some species of it, is a variety of intention, where intention is understood in the standard way as a propositional attitude with a characteristic role in practical reasoning \citep{Bratman:1987xw}. 
Because our claim is that intention and motor representation have distinctive roles in explaining action (§1 on your handout),
we need to show that this temptation should also be resisted. 

As background we first need a generic distinction between content and format. Imagine you are in an unfamiliar city and are trying to get to the central station. A stranger offers you two routes. Each route could be represented by a distinct line on a paper map. The difference between the two lines is a difference in content. Each of the routes could alternatively have been represented by a distinct series of instructions written on the same piece of paper; these cartographic and propositional representations differ in format. 
The distinction between content and format is necessary because, as our illustration shows, each can be varied independently of the other.

Our aim now is to show that motor representations differ from intentions with respect to their format. This is why the second temptation must be resisted. That motor representations differ in format from intentions shows that they are genuinely distinct phenomena.

How can we distinguish representational formats? Because a representational format is typically associated with characteristic processing profiles, it is sometimes possible to infer similarities and differences in representational format from similarities and differences in the processes in which representations feature. 
Cognitive neuroscience frequently depends on inferences of just this type. Contrast imagining seeing an object moving with imagining moving an object. 
Imagining seeing and imagining acting involve different processes
 \citep{sirigu:2011_motor,nico:2004_left,vargas:2004s_influence,fourkas:2006_influence}. Importantly for our purposes, imagining seeing an object move and imagining moving an object also differ in ways that indicate differences in representational format. For instance, the way imagining acting unfolds in time is quite different from the way imagining seeing unfolds: how long the former takes is closely related to the time it would take to perform the action, but this is not true of the latter \citep{decety:1989_timing,decety:1996_imagined,Jeannerod:1994oz}. In addition, for object-related actions such as grasping the handle of a cup, manipulating the object in ways that make the action harder (such as orienting the handle to make it less convenient for you to grasp) make a corresponding difference to the effort involved in imagining performing the action \citep{parsons:1994_temporal,frak:2001_orientation}, whereas such manipulations would make no systematic difference to the effort involving in imaging seeing. The explanation for these differences between imagining acting and imagining seeing plausibly involves a difference in the formats of the representations involved.  

How do these findings bear on motor representation? Contrast imagining an action with actually performing that action. There is substantial evidence that imagining an action involves many of the same processes which would be involved in performing that action up to the actual muscle contractions \citep{jeannerod:1995_mental,jeannerod:2003_mechanism}. 
There is also independent evidence that the representations underpinning imagining acting have the same format as those involved in actually acting. 
We can conclude, then, that imagining and executing actions are both underpinned by motor representations, and that motor representations differ in format from representations involved in both seeing and imagining seeing. 

So far we have been arguing that motor and visual representations differ in format.  
But what we need to show is that motor representations differ in format from intentions.
Why accept this?
We suppose that intentions are propositional. 
This is because of their role in practical reasoning and of the fact that one can have intentions involving quantification and identity; for example, one can intend that one cross seven distinct bridges in 48 hours without yet specifying which bridges or hours.\footnote{ Of course some use the term `intention' for non-propositional representations involved in the execution and control of action.  This is a narrowly terminological issue.} 
But we can also be sure that motor representations differ from propositional representations at least as much as they differ from visual representations given what we have just seen of the characteristic processing profile associated with them.
So no motor representation could be an intention.


\section{The Interface Problem}

We have just argued for three claims. First, some motor representations represent outcomes (rather than, say, only bodily movements). Second, there are actions whose directedness to an outcome is grounded in motor representation. And third, motor representation differs from intention with respect to representational format. A consequence is that a single purposive action may involve representations of the outcomes to which it is directed in at least two different representational formats, motor and propositional. This contributes to a problem we call \emph{the interface problem} ... 

Imagine that you are strapped to a spinning wheel facing near certain death as it plunges you into freezing water. To your right you can see a lever and to your left there is a button. 
You form an intention to pull the lever, hoping that this will stop the wheel. If things go well, and if intentions are not mere epiphenomena, this intention will result in your reaching for, grasping and pulling the lever. These actions---reaching, grasping and pulling---may be directed to specific outcomes in virtue of motor representations which guide their execution. It might not be an accident that you both intend to pull a lever and you end up with motor representations of reaching for, grasping and pulling that very lever, so that the outcomes specified by your intention match those specified by motor representations. If this match between outcomes variously specified by intentions and by motor representations is not  accidental, what explains it?  

The interface problem is the problem of answering this question, of explaining how intentions and motor representations, with their distinct representational formats, are related in such a way that, in at least some cases, the outcomes they specify non-accidentally match. 
(This question is more precisely stated on your handout.\footnote{Let us say that two collections of outcomes, A and B, \emph{match} in a particular context just if, in that context, either the occurrence of the A-outcomes would normally constitute or cause, at least partially, the occurrence of the B-outcomes or vice versa.
In some cases (i) a particular action is guided both by one or more intentions and by one or more motor representations,
(ii) the outcomes specified by the intentions match the outcomes specified by the motor representations, and
(iii) this match is not accidental. 
The question is, How do non-accidental matches come about?
})

Why think that this question poses a problem at all?  
To see why, consider how the question might be answered.

One might try to explain the match by supposing that intentions and motor representations have a common cause. If the mere presence (or the mere perception) of a lever invariably triggered intentions and motor representations specifying grasping (say), it might be possible to explain matching in this way. But this sort of consideration cannot provide a full explanation of non-accidental matching 
because
intentions are not always triggered in straightforward ways by agents' environments or perceptions.
To suppose otherwise is to ignore the very phenomena, decision and planning, which make intention so interesting.  

Another natural approach is to  appeal to content-respecting causal processes. Perhaps, for example, intentions with certain contents (concerning grasping, say) reliably cause motor representations with corresponding contents (also concerning grasping, say). Alternatively we might suppose, very crudely, that some comparator process checks that the contents of motor representations are appropriate given what the agent intends.  

There is a difficulty for any  explanation along these lines.
As argued,
intention and motor representation involve different representational formats. 
In general, when any two representations differ in format, postulating reliably content-respecting causal processes linking them requires us to explain how their contents are coordinated. 
To illustrate, suppose you are given some verbal instructions describing a route. You are then shown a representation of a route on a map and asked whether this is the same route that was verbally described. You are not allowed to find out by following the routes or by imagining following them. This puts you in something like the position of the comparator process envisaged above. Special cases aside, answering the question will involve a process of translation because two distinct representational formats are involved, propositional and cartographic. 
Similarly, for there to be reliably content-respecting causal processes linking intentions with motor representations there would have to be some process of translation.

But what is wrong with postulating a process of translation? The difficulty is that nothing at all is known about this hypothetical translation between intention and motor representation, nor about how it might be achieved, nor even about how it might be investigated. Of course this doesn't \emph{show} that we couldn't fully explain matching by appeal to content-respecting causal processes. But it does show that no such explanation is currently available.  

This, then, is why our question about the interface between intentions and motor representations amounts to a problem. It is a problem because of two natural routes to answering the question, the first (appealing to common causes of intentions and motor representations) is a non-starter and the second (appealing to content-respecting causal processes) amounts to no more than a stab in the dark.  Our aim for the rest of this talk is to solve the interface problem without postulating either common causes or translation processes.

\section{Demonstrative and deferential action concepts}

The interface problem  arises because intentions and motor representations have different representational formats.
There is a way to link representations with different formats that requires neither common causes nor translations. To illustrate, imagine once again that we have two representations of a route, one propositional the other cartographic. But this time suppose that the propositional representation is simply `Follow this route!' where the demonstrative phrase `this route' refers to the route marked on the map. This instruction does not describe the route but merely defers to another representation of it. Because the representation deferred to is cartographic, comparing the instruction with the map no longer requires translation between representational formats. 

We shall suggest that something analogous holds concerning the relation between intention and motor representation. To anticipate what will be explained below, some intentions involve demonstrative concepts which refer to actions by deferring to motor representations.
These demonstrative concepts would be concepts of actions not of motor representations, but they would succeed in being concepts of actions by deferring to motor representations. 

That 
there are demonstrative  thoughts and concepts is quite widely accepted
\citep[e.g.][]{McDowell:1996yi,Brewer:1999ud},
 for both individuals and types  \citep[§3.4]{levine:2010_demonstrative}.
And it seems necessary to suppose that there are demonstrative concepts of actions
given that someone might think to herself, `I wish I could do that too'.
But the further claim that some demonstrative concepts refer to actions \emph{by deferring to motor representations} immediately raises a question. 
Could a demonstrative concept really defer to a motor representation? 

It seems clear that we can't select a motor representation to defer to in the same way that we can select a map when we say `Follow this route!' 
We shall argue, nevertheless, that motor representations are available. 
To start with an analogy, consider pantomiming an action to yourself. You are rehearsing part of an operation which involves precisely grasping a delicate structure with some tweezers. Just as someone might point to a map and say `Follow that route!', so also they could point to your pantomime and say `Do that!'
 In our analogy, the pantomime stands in for a motor representation of the action.  The demonstrative in `Do that!' refers to an action by deferring to the pantomime. Of course this analogy doesn't show that demonstrative concepts can defer to motor representations. But now consider purely mental pantomime---that is, motor imagery. One might use motor imagery to explore different ways of completing a task and then, having hit on a good solution, think to oneself `Do that!' It seems possible that in some such cases the demonstrative refers by deferring to a motor representation of action involved in mental motor imagery.

Here we are assuming that 
it is possible to imagine acting purposively with imagining acting intentionally.
We lack time to defend that assumption now but 
there's plenty more to say, perhaps in the question time.

Until now we have focused on motor imagery. 
Given the parallels between motor imagery and motor preparation \citep{Jeannerod:2001yb}, it is plausible that motor representations also feature in experiences of action in such a way that motor actions are available for demonstrative reference. 

Recall that the interface problem is the problem of explaining how outcomes variously specified by intentions and motor representations, despite their different formats, sometimes non-accidentally match. As long as we think of intentions and motor representations as having logically independent contents, it seems that fully solving the problem would require appeal to 
common causes or 
processes of translatio. But intentions can involve demonstrative action concepts 
which refer to  actions by deferring to motor representations. 
Where this happens, which actions the intention specifies is partially or wholly determined by the motor representation, and so the interface problem is solved.


 
\section{Conclusion}

Philosophers typically treat motor representation either as as only an enabling condition for intentional action (the first temptation) or else as merely a variety of intention (the second temptation). 
In fact most philosophical theories of action  apply indifferently to 
(imaginary) agents lacking motor systems
who would need explicit practical reasoning for each muscle contraction. 
This is a mistake if,
as we have argued,
 motor representation has a distinctive role in explaining the purposiveness of action.  

Our argument was this.
Some motor representations, like intentions, coordinate actions in virtue of representing outcomes; but motor representations and intentions differ in representational format.  
This implies that motor representation has a distinctive role in explaining the purposiveness of action. 
It also leads to what we called \emph{the interface problem}.
A single purposive action may involve representations of the outcomes to which it is directed in at least two different representational formats, motor and propositional;
and these outcomes sometimes non-accidentally match.
The interface problem is the problem of explaining how non-accidental matching is possible, of
explaining how intentions interlock with motor representations. 
We have shown that there is a theoretically coherent and empirically plausible solution to the interface problem that requires neither common causes nor processes of translation.
Both can be avoided providing that intentions can involve demonstrative concepts which refer to actions by deferring to motor representations.
Such demonstrative concepts tie intention to motor representation.

More speculatively, we suggest that where an intention properly and reliably produces bodily movement, either acting on that intention involves a further intention or else the intention involves concepts which refer to actions by deferring to motor representations. If so, it is deferential action concepts that ultimately connect intentions to bodily movements. Only by recognising how intention is tied to motor representation can we hope to understand how our intentions ever make a difference to the world around us.

On this view experience of action plays a novel role. Action experiences in which motor representations feature, such as those associated with motor imagery and those associated with really acting, are arguably necessary for there to be concepts which are constituents of intentions and refer to actions by deferring to motor representations. But if, as we conjecture, such deference is necessary for intentions to properly and reliably result in bodily movements, it may turn out that intentionally acting in the world depends on action experiences featuring motor representation. Much as, on some views, thinking about objects depends on perceptual experience \citep[e.g.][]{Campbell:2002ge}, so also intending actions may depend ultimately on motor experience.  






\bibliography{$HOME/endnote/phd_biblio}

\end{document}