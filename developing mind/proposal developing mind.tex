%!TEX TS-program = xelatex
%!TEX encoding = UTF-8 Unicode

\def \papersize {a4paper}
\documentclass[12pt,\papersize]{extarticle}
% extarticle is like article but can handle 8pt, 9pt, 10pt, 11pt, 12pt, 14pt, 17pt, and 20pt text

\def \ititle {The Developing Mind: \\ {A philosophical introduction to %issues in 
cognitive development}
}
\def \isubtitle {-- Book Proposal --}
\def \iauthor {Stephen A. Butterfill}
\def \iemail{s.butterfill@warwick.ac.uk}
%\def \iauthor {}
%\def \iemail{}
\date{}


%\input{$HOME/Documents/submissions/preamble_steve_report}
%!TEX TS-program = xelatex
%!TEX encoding = UTF-8 Unicode

\title{\ititle\\\isubtitle}
\author{\iauthor\\<{\iemail}>}

\usepackage[\papersize]{geometry} % see geometry.pdf
\geometry{twoside=false}
\geometry{headsep=2em} %keep running header away from text
\geometry{footskip=1cm} %keep page numbers away from text
\geometry{top=3cm} %increase to 3.5 if use header
\geometry{left=4cm} %increase to 3.5 if use header
\geometry{right=4cm} %increase to 3.5 if use header
\geometry{textheight=22cm}

%non-xelatex
%\usepackage[T1]{fontenc}
%\usepackage{tgpagella}

%for underline
\usepackage[normalem]{ulem}

%get the font here:
% http://scripts.sil.org/CharisSILfont

\usepackage{fontspec,xunicode}
%nb do not explicitly use package xltxtra because this introduces bugs with footnote superscripting  -- perhaps because fontspec is supposed to include it anyway.
%UPDATE:  "You need to use the no-sscript option in xltxtra: \usepackage[no-sscript]{xltxtra}, this is explained in the documentation of xltxtra.  The issue is that Sabon does not contain true superscript glyphs for every character and the no-sscript option will instead use scaled regular glyphs, which is typographically inferior, but there is no other option available when using Sabon." --- http://groups.google.com/group/comp.text.tex/browse_thread/thread/19de95be2daacade
\defaultfontfeatures{Mapping=tex-text}
%\setromanfont[Mapping=tex-text]{Charis SIL} %i.e. palatino
%\setromanfont[Mapping=tex-text]{Sabon LT Std} 
%\setromanfont[Mapping=tex-text]{Dante MT Std} 
%\setromanfont[Mapping=tex-text,Ligatures={Common}]{Hoefler Text} %comes with osx
\setromanfont[Mapping=tex-text]{Linux Libertine O} 
\setsansfont[Mapping=tex-text]{Linux Biolinum O} 
\setmonofont[Scale=MatchLowercase]{Andale Mono}


%hyperlinks and pdf metadata
%TODO avoid duplication of title & author
\usepackage{hyperref}
\hypersetup{pdfborder={0 0 0}}
\hypersetup{pdfauthor={\iauthor}}
\hypersetup{pdftitle={\ititle\isubtitle}}


%handles references to labels (e.g. sections) nicely
\usepackage{varioref}

%line spacing
\usepackage{setspace}
%\onehalfspacing
%\doublespacing
\singlespacing

\usepackage{natbib}
%\usepackage[longnamesfirst]{natbib}
\setcitestyle{aysep={}}  %philosophy style: no comma between author & year

%enable notes in right margin, defaults to ugly orange boxes TODO fix
%\usepackage[textwidth=5cm]{todonotes}

%for comments
\usepackage{verbatim}

%footnotes
\usepackage[hang]{footmisc}
\setlength{\footnotemargin}{1em}
\setlength{\footnotesep}{1em}
\footnotesep 2em

%tables
\usepackage{booktabs}
\usepackage{ctable}

%section headings
\usepackage[sf]{titlesec}
%\titlespacing*{\section}{0pt}{*3}{*0.5} %reduce vertical space after header
%large headings:
%\titleformat{\section}{\LARGE\sffamily}{\thesection.}{1em}{} 
\titlelabel{\thetitle.\quad}

%captions
\usepackage[font={small,sf}, margin=0.75cm]{caption}

%lists
\usepackage{enumitem}
\newenvironment{idescription}
{ 	
	% begin code
	\begin{description}[
		labelindent=1.5\parindent,
		leftmargin=2.5\parindent
	]
}
{ 
	%end code
	\end{description}
}


%title
\usepackage{titling}
\pretitle{
	\begin{center}
	\sffamily
	\Huge
} 
\posttitle{
	\par
	\end{center}
	\vskip 0.5em
} 
\preauthor{
	\begin{center}
	\normalsize
	\lineskip 0.5em
	\begin{tabular}[t]{c}
} 
\postauthor{
	\end{tabular}
	\par
	\end{center}
}
\predate{
	\begin{center}
	\normalsize
} 
\postdate{
	\par
	\end{center}
}



%adjust title size & font
\pretitle{
	\begin{center}
	%\sffamily
	\LARGE
} 

%\author{}

%change fonts (not anonymous)
%\setromanfont[Mapping=tex-text]{Linux Libertine O} 
%\setsansfont[Mapping=tex-text]{Linux Biolinum O} 
%\setmonofont[Scale=MatchLowercase]{Andale Mono}


%use titlesec package to make subsections bold italics
\titleformat*{\subsection}{\bfseries\itshape}  


%\setromanfont[Mapping=tex-text]{Sabon LT Std} 

\begin{document}

\setlength\footnotesep{1em}

\bibliographystyle{newapa} %apalike


\tolerance=5000


\maketitle
%\tableofcontents

% disables chapter, section and subsection numbering
\setcounter{secnumdepth}{-1} 

%\begin{abstract}
%\noindent
%***
%\end{abstract}


\section{Brief Description}
How do humans come to know about %
objects,
causes,
words,
numbers,
colours,
actions
and
minds? 
This question goes back to Plato or earlier and remains unanswered.
%Philosophers have been pursuing this question since Plato or before. 
%More recently it has become the focus of a branch of psychology, developmental psychology.
%So far no one has convincingly answered the question, but two major breakthroughs have been made in the last three decades or so.
Two recent scientific breakthroughs appear to bring us closer to an answer, and to show that the question is even less straightforward than previously  assumed.
The first breakthrough concerns social interaction;
is the discovery that preverbal infants enjoy surprisingly rich social abilities.
These abilities enable infants to engage in some forms of social interaction. 
This may well facilitate the subsequent acquisition of linguistic abilities and enable the emergence of knowledge \citep[e.g.][]{Csibra:2009xr,Meltzoff:2007pj,Tomasello:2005wx}. 
A second breakthrough involves the use of increasingly sensitive---and sometimes controversial---methods to detect  expectations without relying on subjects' abilities to talk or act.  
These methods have revealed sophisticated expectations about  causal interactions, numerosity, actions, mental states and more besides in preverbal infants \citep[e.g.][]{Spelke:1990jn,Baillargeon:gx}.
These expectations or the representations and processes underpinning them arguably also enable the emergence of knowledge. 
Although each breakthrough has been discussed at length by leading psychologists, philosophers have yet to consider them systematically in print.
Further, the two breakthroughs are associated with different camps, one nativist, the other Vygotskian.
Perhaps for this reason the breakthrough findings have rarely been considered together as identifying twin factors enabling the emergence of knowledge.
The proposed book aims to introduce readers to  
%example of new---the question about simple joint action
new philosophical issues raised by these findings  
%example of existing question: modularity
and to explain their relevance to longstanding philosophical questions about the mind.  

%if we're talking about breakthroughs we'd better mention neuroscience
Advances in neuroscience may be transforming parts of developmental psychology much as they have already transformed the study of perception and action \citep{Johnson:2005az}. 
Although detailed discussion of neuroscientific methods is beyond its scope, the proposed book will introduce a selection of relevant findings in developmental cognitive neuroscience along the way. 

The proposed book is organised by domains of knowledge, so that one chapter concerns knowledge of objects, another focuses on  knowledge of number, and so on.  
The domains are chosen so that each set of developmental findings is linked to one or more philosophical issues. 
For instance, research on knowledge of objects gives bite to questions about modularity and the nature of tacit knowledge%\citep{Fodor:1983dg,Davies:1989gg}
; research on knowledge of number invites discussions of nativism% \citep{Fodor:1981ep,Spelke:1998im}
; and developmental findings on knowledge of colour may challenge certain assumptions philosophers have made about relations between language, thought and perception.  





\section{Audience}

The book is aimed at advanced undergraduate and graduate students in philosophy. 
It is suited to teaching a course on philosophical issues in cognitive development. 
The book could also be used in courses on the philosophy of mind and action, or the philosophy of psychology. 
To this end the book will not presuppose familiarity with the philosophy of mind and will include chapter summaries and suggestions for further reading.

Professional philosophers whose research connects with issues in cognitive development may also find the book useful.  
Judging by papers and manuscripts under review, philosophers sometimes have difficulty identifying the full range of findings in cognitive development relevant to their positions.  

Psychology students, undergraduate and graduate, are often drawn to philosophy and should find in this book an accessible introduction to key distinctions and arguments. 
It might  be used for courses in developmental psychology. (I once co-taught such a course in psychology with Jim Russell in Cambridge).

For readers captivated by Gopnik's popular \textit{The Philosophical Baby} \citep{gopnik:2009_philosophical} who can cope with academic texts, the proposed book might provide bridge to studying philosophy or psychology.



\section{Length}
70,000--100,000 words


\section{Timetable}
I plan to submit a first draft of the manuscript by 30 September 2014.  

I aim to try out material for the book while teaching courses at  Central European University (Hungary) and the University of Warwick (UK).



\section{Competition}
At the time of writing there is no book or collection devoted to philosophical issues in cognitive development other than \textit{The Philosophical Baby} \citep{gopnik:2009_philosophical}  is primarily aimed at general readers.

Philosophers have written monographs and edited collections on particular issues in cognitive development \citep[e.g.][]{Bermudez:2003dj,carruthers:1996_theories}.  
The proposed book complements these by providing integrated discussion of a wider range of topics. 
The advantage is not just convenience: understanding and properly evaluating theories about the developmental origins of knowledge requires bringing together research on different domains of knowledge.

There are also several collections bringing together research by philosophers and developmental psychologists or presenting developmental research in ways accessible to philosophers 
\citep[e.g.][]{Whiten:1991qn,hirschfeld:1994_mapping,carruthers:2005_innate_structure,carruthers:2006_innate_culture,hoerl:2011_understanding}.
In relation to the proposed book, these provide useful sources of further reading for those who want more information on particular issues.

Finally, the proposed book will complement some excellent books which introduce particular theories of the origins of knowledge; these include 
	\citet{Elman:1996zd},
	\citet{Gopnik:1997xq},
	\citet{Tomasello:1999xz},
	and
	\citet{carey:2009_origin}.
The proposed book will include critical discussion of theories presented in some of these  books.  
It's aims are clearly distinct from those of these books: the proposed book will review a range of existing findings and theories and explore philosophical questions they raise.  







\section{Contents (provisional)}



\subsection{Introduction}
How do humans come to know  about---and to knowingly manipulate---%
objects,
causes,
words,
numbers,
colours,
actions
and
minds?
In pursuing this question we have to consider minds where the knowledge is neither clearly present nor obviously absent. 
This is challenging because both commonsense and theoretical tools for describing minds are generally suitable only for fully developed adult minds. 
Davidson writes:
%
\begin{quote}
`%
%if you want to describe what is going on in the head of the child when it has a few words which it utters in appropriate situations, you will fail for lack of the right sort of words of your own. 
We have many vocabularies for describing nature when we regard it as mindless, and we have a mentalistic vocabulary for describing thought and intentional action; what we lack is a way of describing what is in between' \citep[p.\ 11]{Davidson:1999ju}.
\end{quote}
%
%Of course you might not agree with Davidson that children just starting to use words are incapable of thought or of intentional action. 
%In fact research on children's pre-verbal communicative behaviours such as pointing seems not only to manifest purposiveness but also a sensitivity to others' thoughts.
%(Davidson seems to have paid little attention to development, and he held that being able to think, to act intentionally and to communicate with language all require understanding the possibility that one's own beliefs may be false.)
%Even so, Davidson 
%
To understand the emergence of knowledge we need to find ways of describing what is in between: individuals whose movements are neither mindless nor guided by intention and knowledge.  
And the ways we find to describe these individuals 
%that are not only theoretically coherent and empirically motivated---in addition,
%we need ways of describing these individuals 
must enable us to understand what it is like to be them, to
see things from their points of view.
Some progress has already been made but many challenges remain. 
Philosophers have much to contribute in crafting distinctions and conceptual tools useful for meeting the challenges involved in describing what is in between. 
There is also much for philosophers to gain: studying what is in between mindless nature and the sorts of cognition captured by   commonsense psychological notions 
 will reveal things about what minds are and how they work.

In explaining how research on developing minds bears on some existing philosophical issues and raises some new ones,
this chapter introduces two scientific breakthroughs that have recently furthered understanding of how knowledge might emerge in development.
As already mentioned (in the Brief Description), the first breakthrough is the discovery that preverbal infants enjoy surprisingly rich social abilities, abilities which
% begin to be built from the first months of life and 
may well be foundational for later linguistic abilities and enable the emergence of knowledge \citep[e.g.][]{Csibra:2009xr,Meltzoff:2007pj,Tomasello:2005wx}. 
A second breakthrough is the use of increasingly sensitive---and sometimes controversial---methods to detect  sophisticated expectations concerning  causal interactions, numerosity, mental states and more besides in preverbal infants \citep[e.g.][]{Spelke:1990jn,Baillargeon:gx}.
These breakthroughs are driven by researchers with conflicting theoretical positions and raise quite different issues.  
But both breakthroughs are needed to identify ingredients necessary for understanding the developmental origins of human knowledge.



\subsection{Chapter 1. Social Interaction without Words}
Could social interaction enable cognitive development? 
This chapter examines arguments for the hypothesis that it could (as offered in \citealp{Tomasello:2003tq} and \citealp{Moll:2007gu}). 
It also discusses objections to this hypothesis and introduces questions it raises.
 
One obstacle to even understanding the hypothesis is that much adult social interaction depends so heavily on communication by language (not to mention social networks) that it can be hard to imagine what social interaction with preverbal infants could involve.
To overcome this obstacle we shall review evidence for a package of social abilities manifested in preverbal infants.
These include imitation, which can occur just days and even minutes after birth \citep{meltzoff:1977_imitation,field:1982_imitation,meltzoff:1983_newborn}, 
%for a review on imitation, see \citep{jones:2009_development}
imitative learning \citep{carpenter:1998_fourteen}, 
gaze following \citep{Csibra:2008be},
goal ascription \citep{Gergely:1995sq,Woodward:2000jw},
social referencing \citep{Baldwin:2000qq} 
and 
pointing \citep{Liszkowski:2006ec}.  
Taken together, the evidence reveals that preverbal infants have surprisingly rich social abilities.

The hypothesis that these social abilities enable cognitive development faces an objection. 
For it seems that many of these abilities already presuppose  knowledge of objects, actions and minds. 
For instance, twelve-month old infants  will helpfully point to inform ignorant but not knowledgable adults about the location of an object  \citep{Liszkowski:2008al}.
Doesn't this show that these infants are already capable of knowing about others' knowledge and ignorance? 
%could also ask whether imitation involves understand action --- after all, need to understand what to imitate!
If so, pointing and other social abilities could play at most a limited role in explaining the developmental origins of knowledge of mind. 
After all, we can hardly explain the emergence of something by appealing to abilities whose possession already presuppose it.
We will encounter tools for replying to this objection in Chapters 2 (on Objects), 6 (on Actions) and 7 (on Minds).

If this objection can be overcome, how could social abilities enable cognitive development? 
Some have argued that early social abilities partially explain the emergence in humans of communication by language (\citealp{tomasello:2008origins}; see Chapter 5 on Words below). 
This is a valuable contribution.  
But could social abilities also play a role in explaining the developmental origins of knowledge of things other than words---for example, in explaining the emergence of knowledge of objects, numbers, colours or minds? 
This question comes up in one way or another in each of the following chapters.


\subsection{Chapter 2. Objects and How They Interact}
\label{ch:objects}
When can infants first know things about objects they aren't perceiving?   
For instance, when a ball falling behind a chair disappears from view, when do infants first realise that the ball is somewhere behind the chair?
The ability to realise this is known as `object permanence'.  
One way to test for object permanence is to ask when infants first reach for objects they can’t see or when they first remove barriers to retrieve objects concealed behind them.  
Infants don’t do this until around eight months of age \citep[p.\ 202]{Meltzoff:1998wp} or  later \citep{moore:2008_factors}.  
Since four-month-olds already have the planning skills they would need to execute the reach \citep{Shinskey:2001fk}, 
their failure to reach is evidence that infants can first think about objects they aren’t perceiving at around eight months or later.  
But another way to test for object permanence is to ask how infants respond to apparently impossible events. 
Suppose, for example, that infants watch as a solid object is placed immediately behind a screen and then the screen falls backwards, ending up flat as though the object were not there, which is apparently impossible \citep{baillargeon:1985_object,baillargeon:1987_object}. 
If infants show heightened interest in this and similar cases, perhaps by looking at the display for longer than might otherwise be expected, this would be evidence that they can know things about objects they aren't perceiving.
%\footnote{
%These particular studies have been attacked on methodological grounds \citep[e.g.][]{sirois:2007_social_}, but there are other, related studies \citep[some are mentioned by][]{Aguiar:2002ob}.
%}
 As it turns out, infants show such heightened interest from around four months of age or earlier. 
Put together, the two sorts of findings give rise to the `paradox of early permanence' (as \citealp{Meltzoff:1998wp} call it). 
The best explanation of the first sort of findings seems to be that infants cannot think about objects they aren't perceiving until eight months or later; 
but the best explanation of the second sort of findings seems to be that infants can do this from around four months or earlier. 
Clearly these explanations cannot both be correct. 
But neither seems to be wrong. 

This conflict exemplifies a pattern that occurs again and again in investigating the developmental origins of knowledge. 
Here's the pattern.
We ask when humans can first know about X. 
One set of findings provides converging evidence that the answer is: surprisingly early.
Another set of findings, using a different set of techniques,  provides converging evidence that the answer is: much later. 
Unless they arise from methodological failings, these conflicting answers force us to recognise that the question involves a mistake. 
The mistake is to think that there is a single kind of knowledge. 
%As some psychologists reflecting on difficulties in understanding knowledge of objects suggest, 
In fact:
%
\begin{quote}
`there are many separable systems of mental representations ...\ and thus many different kinds of knowledge. ...\ 
[T]he task ...\ is to contribute to the enterprise of finding the distinct systems of mental representation and to understand their development and integration% 
%. ...\ the challenge ...\ is to characterize what it is about the mental representations of the presence or absence of the barriers that constrains representations of where the object should be in looking-time studies and how these representations differ from those that guide search or pointing in the present studies%
'
\citep[p.\ 1522]{Hood:2000bf}.
\end{quote}
%
It is one thing to propose that there are multiple kinds of knowledge, quite another to make systematic sense of this possibility.
In this chapter we will consider attempts to do this by appeal to notions of modularity \citep{Fodor:1983dg}
and core knowledge \citep{Spelke:2007hb}. 
Both attempts raise further questions.  
These questions about how to make sense of the possibility that there are multiple kinds of knowledge 
run through the following chapters. 



\subsection{Chapter 3. Numbers}

How might abilities based on core knowledge enable the emergence in development of knowledge proper? 
%(\textit{Knowledge proper} is knowledge not subject to the limits associated with  core knowledge.)
In this chapter we consider answers to this question for the special case of knowledge of number.

In Chapter 2 (on Objects) we saw that core knowledge of objects is manifested early in infancy, whereas more flexible abilities involving knowledge proper first appear months or even years later. 
A similar pattern can be observed in the case of number. 
From around five months of age or earlier, infants 
are sensitive to the number of items in a repeatedly presented stimulus \citep{starkey:1980_perception}, 
can sum over the number of items in a short sequence \citep{wynn:1996_infants}, 
can discriminate between small and large sets of objects \citep{xu:2000_large}, 
and can keep track of the precise number of items in a small set of objects following additions or subtractions \citep{wynn:1992_addition}. 

Should we conclude that infants already have adult-like number concepts?  
Limits on infants' abilities show that we should not. 
To illustrate, 
at around fourteen months of age infants who have seen two objects put into a box will search in it until they have recovered two objects, and infants who have seen three objects put into the box will search three times.
But after seeing four objects placed in the box, infants usually search just once \citep{feigenson_tracking_2003}. 
In general, infants' abilities to deal with precise numerosity are strictly limited to situations involving at most three items \citep{feigenson_tracking_2003}.
%debate about exact number?
It seems clear, then, that infants do not have adult-like number concepts. 

The three-item limit on infants' abilities to deal with precise numerosity is the key to a revealing discovery.
Many species besides humans have abilities to deal with precise numerosity.
Because these abilities are also subject to limits like infants' three-item limit, it is reasonable to hold that they are  closely related \citep{hauser:2003_spontaneous}.  
%In fact, infant-like, limited abilities can also be found in human adults who have number concepts. 
This makes it unlikely that infants' core knowledge of number is  a product of learning \citep{feigenson:2004_core}, and so supports a modest form of nativism.

A bolder form nativism would insist not only that core knowledge of number is innate but also that number concepts proper are innate \citep{Fodor:1981ep}. 
Against this bolder view, some have attempted to explain how core knowledge together with mastery of a sequence number words could explain how number concepts can be acquired \citep{carey:2009_origin}. 
If successful, this explanation shows that not all primitive concepts are innate and provides us with one model of how core knowledge together with linguistic abilities might enable the emergence of knowledge proper.



\subsection{Chapter 4. Seeing and Talking about Colours}
How do children acquire colour concepts and colour words---concepts and words for red, blue and green, say? 
Exploring this question reveals unexpected complexity in the developmental origins of knowledge, as we will see in this chapter.

%Frank Jackson is best known for the story of Mary, a brilliant scientist who spends the first part of her stuck in a monochrome environment learning all that can be known from books about everything including the colour blue \citep[p.\ 130]{jackson:1982_epiphenomenal}. 
%Then one day she is set free, goes out into the world and learns something she could not have learnt from a book, namely what it is like for others to sense blue \citep[p.\ 292]{jackson:1986_what}.
%Although the story seems fantastic, it echoes a developmental pattern. 

Categorical perception of colour emerges early in infancy.  This has been demonstrated with four-month-olds using habituation \citep{Bornstein:1976of} and visual search \citep{Franklin:2005xk}.   
%say something about ERP / oddball study \citep{Clifford:2009fo}
Slightly older infants can make use of colour properties such as red and green to recognise objects.  
For instance, nine-months-olds can determine whether an object they saw earlier is the same as a subsequently presented object on the basis of its colour \citep{Wilcox:2008jk}.  
By the time they are two years old, toddlers who do not comprehend any colour words can use colour categories implicitly in learning and using proper names; for instance, they are able to learn and use proper names for toy dinosaurs that differ only in colour \citep[][Experiment 3]{Soja:1994np}.  
So infants and toddlers enjoy categorical perception of colour and may benefit from it in recognising and learning about objects.  
However children only acquire concepts of, and words for, colours some time later; and colour concepts, like colour words, are acquired gradually \citep{Pitchford:2005hm,Kowalski:2006hk,Sandhofer:1999if,Sandhofer:2006qo}. 

A natural hypothesis, then, is that the acquisition of colour words and concepts builds on categorical perception of colour: infants are first perceptually acquainted with colours and then learn names for them.
And since infants enjoy categorical perception not only of colour but also of orientation \citep{franklin:2010_hemispheric}, speech \citep{Kuhl:1987la,Kuhl:2004nv,Jusczyk:1995it} and facial expressions of emotion \citep{Etcoff:1992zd,Kotsoni:2001ph,Campanella:2002aa}, 
we might suppose that something similar holds in these cases too.
The hypothesis, then, is that categorical perception provides `the building blocks—the elementary units—for higher-order categories' \citep[p.\ 3]{Harnad:1987ej}.

This hypothesis turns out to be entirely mistaken. 
The extensions of colour words may be subject to universal constraints but they certainly vary significantly between languages  \citep{kay:2003_resolving}.
%can use figure from Kay & Regier 2007 Berinmo on p. 291
The boundaries of adult humans' colour concepts and their perceptual colour categories  are influenced by the extensions of the colour words they use \citep{Kay:2006ly,Roberson:2007wg,Winawer:2007im}. 
It is even possible to alter the boundaries of adults' perceptual categories by teaching them new colour words
\citep{Ozgen:2002yk}. 
But preverbal infants' perceptual colour categories are not influenced by colour words. 
Indeed, in toddlers who have recently acquired colour words, the extensions of their colour concepts do not match their perceptual category boundaries \citep{Franklin:2005hp}. 
So it is a mistake to think of the development of colour words and concepts as a matter of learning to label or demonstratively refer to categories infants are already perceptually acquainted with. 
Quite the reverse: acquiring colour words and concepts is a prerequisite for being able to perceive the colour categories they refer to.

%***Something about different processes:
%Adult-mode categorical perception of colour disappears in the face of predictable verbal interference but not non-verbal interference (*Roberson, Davies and Davidoff 2000: 985; Pilling, Wiggett, et al. 2003: 549–50; Wiggett and Davies 2008).
%***Maybe also mention that language can play a role.

Can the case of colour shed light on the question of how core knowledge enables the emergence of knowledge proper?
There are some parallels between categorical perception and core knowledge: for instance, both phenomena are judgement-independent and have limited effects on action. 
Indeed, there seems no less reason to postulate core knowledge of colour or speech than there is to postulate core knowledge of objects or agents. 
So the case of colour suggests that the development of knowledge proper is not always a matter of making explicit, or building on, what was earlier implicit in core knowledge. 
In fact core knowledge might sometimes be in competition with knowledge proper. 
%It should not be taken for granted that the several kinds of knowledge involved in development are harmoniously integrated. 
%Rather, core knowledge might sometimes be in competition with knowledge proper. 



\subsection{Chapter 5. Words and Other Communicative Tools}
One problem in acquiring a language is to identify which things are words; among much else, this involves working out where, in a stream of speech or gesture, one word ends and another begins.  
In solving this problem infants may rely on statistical learning \citep{Saffran:1996aj}.
Another problem is learning which words pick out which objects and properties; this is often called the `mapping problem'. 
Philosophers have sometimes held that in solving this problem children rely on association (or perhaps other forms of statistical learning).
Compare Wittgenstein (\citeyear[p.\ 77]{Wittgenstein:1972lj}): 
%
\begin{quote}
`The child learns this language from the grown-ups by being trained to its use. I am using the word ``trained'' in a way strictly analogous to that in which we talk of an animal being trained to do certain things. It is done by means of example, reward, punishment, and suchlike.'
%careful with this quote --- from the context Wittgenstein is imagining how things might be, not how they are.
%but this quote shows that he is serious about how children are:
%`A child uses such primitive forms of language when it learns to talk. Here the teaching of language is not explanation, but training.' \citep[§5, p.\ 4]{Wittgenstein:1953mm}
%And:
%`The primitive language-game which children are taught needs no justification; attempts at justification need to be rejected.' \citep[p.\ 200]{Wittgenstein:1953mm}
%And:
%`I myself wrote in my book that children learn to understand a word in such and such a way. Do I know that, or do I believe it? Why in such a case do I write not "I believe etc." but simply the indicative sentence?' \citep[§290, p.\ 38]{Wittgenstein:1974dk}
%
%See also Russell (1921: 71): “A child learning to speak is learning habits and associations which are just as much determined by the environment as the habit of expecting dogs to bark and cocks to crow. The community that speaks a language has learnt it, and modified it by processes almost all of which are not deliberate, but the results of causes operating according to more or less ascertainable laws.”
\end{quote}
%
If this sort of view were right, abilities to communicate linguistically would not necessarily depend on knowledge proper. 
So it would be possible to conjecture that the emergence in development of knowledge proper is a consequence of 
training with language.
%Compare Wittgenstein again:
%`535. The child knows what something is called if he can reply correctly to the question "what is that called?"
%536. Naturally, the child who is just learning to speak has not yet got the concept is called at all.
%537. Can one say of someone who hasn't this concept that he knows what such-and-such is called?
%538. The child, I should like to say, learns to react in such-and-such a way; and in so reacting it doesn't so far know anything. Knowing only begins at a later level.'
%\citep[p.\ 71]{Wittgenstein:1974dk}
This does indeed seem plausible for the special case of colour words, as we saw in Chapter 4 (on Colours); 
and something like it may hold for number words too (see Chapter 3 on Numbers).  
%ref to Wittgenstein on child learning about numbers somewhere?
But putting these special cases aside, the experimental findings do not support Wittgenstein's claim that language is entirely acquired through training.
Instead they support the hypothesis that children solve the mapping problem by general reasoning and rely on social cognition
\citep{Baldwin:1995xl,Bloom:2000qz,Sabbagh:2001sp}. 
This hypothesis is further supported by findings that children who have learnt few or no words sometimes invent their own words \citep{Clark:1993bv}, 
and by
research on profoundly deaf children brought up in purely oral environments and therefore without experience of language who, individually or in groups, create their own signed languages \citep{Kegl:1999es,Senghas:2001zm,Goldin-Meadow:2003pj}. 

This hypothesis---children solve the mapping problem by general reasoning---conflicts with Davidson's claims that having thoughts involves grasping the concept of truth \citep[p.\ 189]{Davidson:2001sm}, 
% `What gives my belief its content ... is my knowledge of what is required for the belief ... to be true' \citep[p.\ 189]{Davidson:2001sm}, 
and that `we grasp the concept of truth only when we can communicate the contents—the propositional contents—of the shared experience, and this requires language' \citep[p.\ 27]{Davidson:1997wj}.
% also:
%`language ... is, or has become, a mode of perception.  However, speech is not just one more organ; it is essential to the other senses if they are to yield propositional knowledge' \citep[p.\ 22]{Davidson:1997wj}
% also:
%`Mastery of the word or concept [`table'] requires in addition, I believe, that the child understand that error is possible, that "That's a table' expresses a judgment that has a truth value independent of its merely being uttered.' \citep[p.\ 9]{Davidson:2001np}
%
The objection to Davidson's view is that  it has things backwards: acquiring your first words already involves thinking about what they pick out. 
If this objection is correct, can anything be salvaged from Davidson's arguments? 
While the objection rules out claiming that being able to think depends on being able to communicate propositional contents using language, it leaves open the possibility that thought is essentially intersubjective.
One challenge, then, is to explain the intersubjectivity of thought without appeal to communication by language.

So far our focus has been the mapping problem, the problem of determining which words pick what out. 
A distinct and no less difficult problem is how infants come to understand words as communicative tools at all.
Merely associating a word or gesture with what it picks out would not enable a child to respond appropriately to uses of that word or gesture, nor to use it intelligently (contra \citealp[p.\ 75]{Russell:1921ww}).
%`four ways of understanding words:
%[...]
%(4) When the word is being first learnt, you may associate it with an object, which is what it "means," or a representative of various objects that it "means."
%In the fourth case, the word acquires, through association, some of the same causal efficacy as the object. The word "motor" can make you leap aside, just as the motor can, but it cannot break your bones. The effects which a word can share with its object are those which proceed according to laws other than the general laws of physics, i.e. those which, according to our terminology, involve vital movements as opposed to merely mechanical movements.'
%\citep[p.\ 75]{Russell:1921ww} 
After all, chimpanzees associate pointing gestures with their referents \citep[][p.\ 6]{Moll:2007gu} but fail to respond appropriately to helpful pointing gestures \citep{hare_chimpanzees_2004}. 
Unlike chimpanzees, children not only map words to what they pick out but also use and understand them as tools for communication. 
What does this understanding amount to---in particular, do children learning their first words already have an adult-like understanding of communicative intention?  And how do children acquire this understanding?
%On one view, it involves a special `action interpretation system' distinct from that involved in understanding ordinary, non-communicative action \citep[p.\ 456]{Csibra:2003kp}. 
%***This question is pressing because we want to appeal to language to explain the emergence of full-blown mindreading
Answering these questions may require revisions or extensions to existing philosophical accounts of communication which often presuppose conceptual sophistication \citep[e.g.][]{Grice:1957kg}.




\subsection{Chapter 6. Actions: Teleology and Motor Awareness}
%Astington's paradox:
%\begin{quote}
%`There is something paradoxical about intention attribution.
%Though it is true that motivational states are more obvious and more frequently inferred than beliefs, the attribution of intention---in a precise and complete sense---is not simpler, and may indeed be harder, than the attribution of belief'
%\citep[p.\ 85]{astington:2001_paradox}


Tracking a goal-directed action involves at least two things: distinguishing the action from other events 
and
identifying to which goal or goals it is directed.
Infants' expectations indicate that they can do both of these things from some time around the end of their first year of life \citep[e.g.][]{Csibra:2003kp}. 
But tracking is not understanding.
To say that infants can track goal-directed action is not yet to say anything about what they understand of action.
Consider two questions about the nature of action.
First, which events are actions?
Second, what is the relation between an action and the goal or goals to which the action is directed?
For each of these we can ask corresponding questions about infants' understanding of action.
Which events do infants take to be actions?
And how do they understand the relation between actions and the goals to which they are directed?
We need to answer these questions in order to understand how early abilities to track action are developmentally related to knowledge of action, and also how they relate to developments in other domains including language.
In this chapter we explore three hypotheses about the understanding which underpins infants' abilities to track action.


The question of what actions are is usually answered by appeal to intention, a propositional attitude which plays a characteristic role in planning and coordinating action, is linked to practical reasoning and is subject to characteristic norms (\citealp{Bratman:1987xw}).  
On the standard view, an action is directed to an goal in virtue of the action's being appropriately related to an intention which represents this goal or some related goal.
The corresponding view about understanding has it that 
understanding action involves understanding intentions and so having mastered propositional attitude psychology.
Could this be how infants understand actions?
This view would be consistent with the views of those who hold that one-year-old infants understand psychological notions like belief and desire \citep{Baillargeon:gx}.
But this view is incompatible with  views on which early understanding of action plays a role in explaining how children later acquire knowledge of belief, desire and other mental states.
We should therefore consider alternative, less conceptually demanding accounts of action understanding.

Are there alternative views of what understanding actions involves which do not demand knowledge of intentions?
As before, it is helpful to start with a view about what actions are and then consider corresponding views about understanding actions.
Instead of thinking of the relation between actions and the goals to which they are directed as determined by intentions representing those goals,
we might instead attempt to explain the relation in terms of justification.
This is the idea behind Gergely and Csibra's `principle of rational action' which states that:
%
\begin{quote}
`%the principle of rational action ... states that 
an action can be explained by a goal state if, and only if, it is seen as the most justifiable action towards that goal state that is available within the constraints of reality' 
(Csibra and Gergely 1998: 255; cf. Csibra, Bíró, et al. 2003)
\end{quote}
%
%*Don't we have to separate out the question of which events are actions?
As part of an account of what actions are, this is unlikely to be correct.
After all, agents sometimes fail to select actions that could be justified (or seen as justifiable) given the goal to which they are directed.
\citet[p.\ 123]{Csibra:2003jv} themselves note this point.
But the principle of rationality might serve as a useful approximation.
Perhaps someone who applied the principle of rationality to work out which goals actions are directed to would be right much of the time. 
So perhaps infants' abilities to track actions involve understanding action in accordance with the principle of rationality.
This view of action understanding requires less conceptual sophistication thanks to not requiring knowledge of intentions or other propositional attitudes as such. 
But it remains quite demanding, for it entails that one-year-olds  are able to justify actions and engage in means end reasoning.
Gergely and Csibra stress this aspect of their view:
%
\begin{quote}
`%when taking the teleological stance 
one-year-olds apply the same inferential principle of rational action that drives everyday mentalistic reasoning about intentional actions in adults' (\citealp[p.\ 259]{Csibra:1998cx}; \citealp[compare][p.\ 290]{Gergely:2003gb})
\end{quote}
%
This is incompatible with the notion that infants' abilities to track action depend on core knowledge or modular representations.
For, as we saw in Chapter 2 (on Objects),  processes involving core knowledge  are supposed to be informationally encapsulated;
and informationally encapsulated processes cannot represent relations involving rationality or justification as such \citep{Fodor:2000cj}.
To see whether it is even theoretically coherent to suppose that understanding action could involve core knowledge only, we must find alternative accounts of what it is to understand actions, accounts that involve neither intention nor rationality.


%Some philosophers oppose the standard view of goal-directed action.
%Instead of explaining the directedness of actions to goals by appeal to intentions which represent those goals, 
%it is at least theoretically coherent to suppose that an action's being directed to an outcome may consist in its having the function of bringing about that outcome. 
%Here function should be construed teleologically. 
%One perhaps inadequate but illustrative construal says that an action has the function of bringing about an outcome just if (i) actions of this type have brought about outcomes of this type in the past, and (ii) this action occurs now in part because of that fact.
%Teleological accounts of function, and of the application of this notion to understanding goal-directed action, have been extensively developed \citep[see further][]{Godfrey-Smith:1996ln,Millikan:1984ib,Neander:1991to,Price:2001hs,Wright:1976ls}.

So far we have ignored the possibility that motor representations and processes play a role in understanding action.
It is now well established that motor processes and representations are involved in observing actions---not only in producing them \citep{rizzolatti_functional_2010}.
Furthermore, some motor representations represent outcomes to which actions are directed \citep{jeannerod_motor_2006}. 
For instance, observing another grasping a cup can involve a motor representation of the outcome, of the cup's being grasped, and not only motor representations of merely kinematic or dynamic features.
Apparently, then, motor planning can be used in reverse: it can be used for inferring a goal from an observed action and not only for  determining what to do in order to achieve a particular goal.
We also know that, in adults and infants alike, motor representation can facilitate judgements about the goals of actions 
(on adults: \citealp{serino:2009_lesions_,pazzaglia:2008_sound_}; on infants: \citealp{bekkering:2000_imitation}).
It is possible, therefore, that infants can track actions not because they know about agent's intentions or consider the rationality of actions,
but because their motor expertise enables  them to  determine to which goals others' actions could be directed. 
(This is surely not the whole story, but it can be extended to involve statistical learning as well as motor representation; see \citet{paulus:2011_role}.)
%don't cite here: these are to do with segmentation
%, \citet{Baldwin:2001rn}, and \citet{Saylor:2007pj}.)
On this hypothesis, infants' abilities to track action do not involve understanding actions in the way that human adults do (or are standardly held to). 
For this reason the hypothesis might  appear  to make infants' abilities concerning action philosophically uninteresting.
We can see that this appearance is mistaken by reflecting on how intentions (and other propositional attitudes) might refer to actions involving bodily movements.
The considerations for the view that 
thought about objects depends on perceptual experience of them  \citep[see][]{Campbell:2002ge} 
also support the view that 
all thought about actions ultimately depends on motor experience of them.



%\begin{quote}
%\citep[p.\ 154]{bekkering:2000_imitation}: `behaviours are not simply replicated as unified, non-decomposed motor patterns. Rather, imitation involves first a decomposition of the motor patterns into their constituent components and later a reconstruction of the action pattern from these components. Second, the decomposition--reconstruction process is guided by an interpretation of the motor pattern as a goal- directed behaviour. Thus, the constituent elements in the mediating representation involve goals rather than motor segments. Third, we assume that these goals are represented in a hierarchical pattern with some of the encoded goals being dominant over others.' 
%\end{quote}
%
%*parallels between how visual representations and processes enable experiences of objects and how motor representations and processes enable experiences of actions. Of course, just as not every visual representation leaves its mark on experience, there are surely many motor representations that are only distantly if at all related to experience. But, equally, there are experiences which depend in part on motor representation in much the way that some visual experiences depend in part on visual representations.
%
%


\subsection{Chapter 7. Beliefs}
What is involved in representing belief?
The following three claims are separately defensible but appear to be jointly inconsistent: 
%
\begin{enumerate}
\item Infants can represent false beliefs from around their first birthday or earlier. \label{infant_fb}
%
\item  \label{fb_is_perspectives}  Being able to represent false beliefs involves being able to  (i) process perspective differences or (ii) reason counterfactually (or both).
%
\item Infants cannot (i) process perspective differences nor (ii) engage in counterfactual reasoning until they are at least one year old.  \label{infant_perspectives}
%
\end{enumerate}
%
This chapter will first consider evidence and theoretical arguments in support of each claim in turn.  

There is a growing body of evidence for (\ref{infant_fb}).
From around their first birthday infants  predict actions of agents with false beliefs about the locations of objects \citep[]{Onishi:2005hm, Southgate:2007js}
and choose different ways of interacting with others depending on whether their beliefs are true or false \citep[]{Buttelmann:2009gy,Knudsen:2011fk,southgate:2010fb}.  
And in much the way that irrelevant facts about the contents of others’ beliefs modulate adult subjects’ response times, such facts also affect how long 7-month-old infants look at some stimuli \citep[]{kovacs_social_2010}.
The variety of  measures---looking time, anticipatory looking, pointing and helping---makes it hard to dismiss these findings on methodological grounds.
So we cannot straightforwardly reject (\ref{infant_fb}).

Why accept (\ref{fb_is_perspectives})?  
Part of the answer is theoretical. 
On any standard view, beliefs are propositional attitudes and are individuated (at least in part) by their causal and normative roles in explaining thoughts and actions \citep[]{Davidson:1980xp, Davidson:1990du}.
Representing states with these features arguably involves being able to process perspective differences and a core component of genuine counterfactual reasoning \citep{perner:2007_objects}. 
Another part of the argument for accepting (\ref{fb_is_perspectives}) is empirical.
Until they are around four years of age,
children systematically fail a wide range of false belief tasks.
These tasks are very varied: most are verbal but some are nonverbal (\citealp{Call:1999co}; \citealp{low:2010_preschoolers} Study 2),
some involve prediction whereas others involve retrodiction or justification \citep[e.g.][]{Wimmer:1998kx},
some concern the first-person perspective, whereas others involve a second- or third-person perspective \citep[e.g.][]{Gopnik:1991db},
some involve interaction whereas in others the subject is a mere observer \citep[e.g.][]{Chandler:1989qa},
and some involve predicting actions whereas others involve predicting desires \citep{Astington:1991kk} or selecting an argument appropriate for someone with a false belief \citep{Bartsch:2000es}. 
Despite all this variation and more, these false belief tasks all appear to measure a single developmental transition \citep{Wellman:2001lz},
and that developmental transition is linked to developments in abilities to process perspective difficulties \citep{Perner:2002jj} and to reason counterfactually \citep{rafetseder:2012_submitted}.
While some researchers have claimed that the developmental transition these false belief tasks measure is not a transition in understanding belief 
	(e.g.\ 
	\citealp[][p.\ 417]{Carpenter:2002gc},
	\citealp{Bloom:2000bt}, and
	\citealp{Leslie:1998nq}),
the wide variety of tasks used is evidence against such claims.
So we cannot straightforwardly reject  (\ref{fb_is_perspectives}).

Can we reject (\ref{infant_perspectives}), the claim that infants neither process perspective differences nor engage in counterfactual reasoning until they are at least one year old?
One difficulty is that current developmental evidence almost uniformly supports this claim  \citep{rafetseder:2010_counterfactual,beck:2011_almost}.
A second problem is that the development of abilities to reason counterfactually,
%\footnote{Here and throughout I follow Eva Rafetseder in using the term `counterfactual reasoning'  to refer to reasoning which essentially involves comparing two possible situations; the term is sometimes used more broadly to include the sorts of representations involved in pretence.} 
like the development of abilities to represent false belief,
appears to involve working memory and inhibitory control 
(on counterfactuals: 
	\citealp{drayton:2011_counterfactual,beck:2011_supporting};
on false beliefs: \citealp{apperly:2008_back, Apperly:2009cc}; \citealp{lin:2010_reflexively, McKinnon:2007rr} experiments 4-5; \citealp{saxe_reading_2006}).
As even committed nativists are likely to agree, capacities for inhibitory control and working memory develop over several years and are limited in infants \citep[e.g.][]{carlson:2005_developmentally}.
So we cannot straightforwardly reject (\ref{infant_perspectives}).



If each of the three claims (1)-(3) is true, they cannot be inconsistent after all. 
Can we avoid the apparent inconsistency by appealing to the distinction between core knowledge (or modular representation) and knowledge proper introduced in Chapter 2 (on Objects)?  
One objection to such an appeal concerns the flexibility of false belief understanding in infants.
One-year-old infants succeed on false belief tasks which involve actively helping others, interpreting their utterances and pointing to provide information \citep{Buttelmann:2009gy,Knudsen:2011fk,southgate:2010fb}. 
By contrast, core knowledge is supposed to have only limited consequences for purposive action.

To the apparent inconsistency of (1)-(3) there are three main responses.
One is to reject (1) \citep[e.g.][]{Perner:2005hq}, 
another is to reject (2),
 and a third is to argue that the inconsistency is only apparent by appeal to a distinction such as that between core knowledge and knowledge proper \citep[e.g.][]{Clements:1994cw,low:2010_preschoolers}.
We have seen objections to each of these responses.
This indicates that we do not adequately understand what is involved in representing beliefs and, more generally, what it is to have %a theory of mind 
knowledge of minds.

%This is intended as leading in to measurement and minimal theory of mind

\subsection{Conclusion}
Are we closer to understanding 
how humans come to know about %
objects,
causes,
words,
numbers,
colours,
actions
and
minds? 
Maybe.
%In some special cases---number, colour, words and minds---there are theories which make testable predictions.
%These theories conflict with some philosophers' views about development.
%Most obviously, these theories provide objections to any form of nativism on which number, colour or psychological concepts are innate.
%The findings that inform these theories also lead to objections to ***Wittgenstein ***Davidson
We have seen that apparent inconsistencies in developmental findings concerning knowledge of objects (Chapter 2) and minds (Chapter 7) can be resolved by distinguishing kinds of knowledge.
One challenge is to make such distinctions in ways that are both  theoretically coherent and empirically motivated.
In addressing this challenge we appealed to the notions of modularity and core knowledge and considered how, in particular cases (including objects, numbers, colours and actions), perceptual and motor representations and processes can realise core knowledge.

A second, related challenge arises from the hypothesis that social interaction enables the emergence in development of knowledge.  
For this hypothesis to be coherent, there must be forms of social interaction which do not already presuppose the knowledge whose emergence is to be explained.
Existing accounts of social interaction apply only to those who already have knowledge of several domains, including objects, actions and minds.
The challenge, then, is to characterise these simple forms of social interaction.

Distinguishing core knowledge from knowledge proper 
and
characterising simple forms of social interaction
yielded 
two key ingredients of
developmental theories for some special cases, including numbers, colours, words and minds.
We know that these  theories advance our understanding of development because they lead to new objections to various philosophical views about nativism, language acquisition and the dependence of thought on language.
Of course these theories will probably turn out to be wrong.
What seems secure, though, is the idea that explaining development requires 
%1
distinguishing multiple kinds of knowledge, 
%2
identifying simple forms of social interaction 
and 
%3
understanding how 
	these jointly enable development.
	%social interaction can transform knowledge.





\bibliography{$HOME/endnote/phd_biblio}

\end{document}