%!TEX TS-program = xelatex
%!TEX encoding = UTF-8 Unicode

\def \papersize {a4paper}
\documentclass[12pt,\papersize]{extarticle}
% extarticle is like article but can handle 8pt, 9pt, 10pt, 11pt, 12pt, 14pt, 17pt, and 20pt text

\def \ititle {The Developing Mind: A philosophical introduction to issues in cognitive development}
\def \isubtitle {-- Book Proposal --}
\def \iauthor {Stephen A. Butterfill}
\def \iemail{s.butterfill@warwick.ac.uk}
%\def \iauthor {}
%\def \iemail{}
\date{}


%!TEX TS-program = xelatex
%!TEX encoding = UTF-8 Unicode

\title{\ititle\\\isubtitle}
\author{\iauthor\\<{\iemail}>}

\usepackage[\papersize]{geometry} % see geometry.pdf
\geometry{twoside=false}
\geometry{headsep=2em} %keep running header away from text
\geometry{footskip=1cm} %keep page numbers away from text
\geometry{top=3cm} %increase to 3.5 if use header
\geometry{left=4cm} %increase to 3.5 if use header
\geometry{right=4cm} %increase to 3.5 if use header
\geometry{textheight=22cm}

%non-xelatex
%\usepackage[T1]{fontenc}
%\usepackage{tgpagella}

%for underline
\usepackage[normalem]{ulem}

%get the font here:
% http://scripts.sil.org/CharisSILfont

\usepackage{fontspec,xunicode}
%nb do not explicitly use package xltxtra because this introduces bugs with footnote superscripting  -- perhaps because fontspec is supposed to include it anyway.
%UPDATE:  "You need to use the no-sscript option in xltxtra: \usepackage[no-sscript]{xltxtra}, this is explained in the documentation of xltxtra.  The issue is that Sabon does not contain true superscript glyphs for every character and the no-sscript option will instead use scaled regular glyphs, which is typographically inferior, but there is no other option available when using Sabon." --- http://groups.google.com/group/comp.text.tex/browse_thread/thread/19de95be2daacade
\defaultfontfeatures{Mapping=tex-text}
%\setromanfont[Mapping=tex-text]{Charis SIL} %i.e. palatino
%\setromanfont[Mapping=tex-text]{Sabon LT Std} 
%\setromanfont[Mapping=tex-text]{Dante MT Std} 
%\setromanfont[Mapping=tex-text,Ligatures={Common}]{Hoefler Text} %comes with osx
\setromanfont[Mapping=tex-text]{Linux Libertine O} 
\setsansfont[Mapping=tex-text]{Linux Biolinum O} 
\setmonofont[Scale=MatchLowercase]{Andale Mono}


%hyperlinks and pdf metadata
%TODO avoid duplication of title & author
\usepackage{hyperref}
\hypersetup{pdfborder={0 0 0}}
\hypersetup{pdfauthor={\iauthor}}
\hypersetup{pdftitle={\ititle\isubtitle}}


%handles references to labels (e.g. sections) nicely
\usepackage{varioref}

%line spacing
\usepackage{setspace}
%\onehalfspacing
%\doublespacing
\singlespacing

\usepackage{natbib}
%\usepackage[longnamesfirst]{natbib}
\setcitestyle{aysep={}}  %philosophy style: no comma between author & year

%enable notes in right margin, defaults to ugly orange boxes TODO fix
%\usepackage[textwidth=5cm]{todonotes}

%for comments
\usepackage{verbatim}

%footnotes
\usepackage[hang]{footmisc}
\setlength{\footnotemargin}{1em}
\setlength{\footnotesep}{1em}
\footnotesep 2em

%tables
\usepackage{booktabs}
\usepackage{ctable}

%section headings
\usepackage[sf]{titlesec}
%\titlespacing*{\section}{0pt}{*3}{*0.5} %reduce vertical space after header
%large headings:
%\titleformat{\section}{\LARGE\sffamily}{\thesection.}{1em}{} 
\titlelabel{\thetitle.\quad}

%captions
\usepackage[font={small,sf}, margin=0.75cm]{caption}

%lists
\usepackage{enumitem}
\newenvironment{idescription}
{ 	
	% begin code
	\begin{description}[
		labelindent=1.5\parindent,
		leftmargin=2.5\parindent
	]
}
{ 
	%end code
	\end{description}
}


%title
\usepackage{titling}
\pretitle{
	\begin{center}
	\sffamily
	\Huge
} 
\posttitle{
	\par
	\end{center}
	\vskip 0.5em
} 
\preauthor{
	\begin{center}
	\normalsize
	\lineskip 0.5em
	\begin{tabular}[t]{c}
} 
\postauthor{
	\end{tabular}
	\par
	\end{center}
}
\predate{
	\begin{center}
	\normalsize
} 
\postdate{
	\par
	\end{center}
}


%\author{}

%\setromanfont[Mapping=tex-text]{Sabon LT Std} 

\begin{document}

\setlength\footnotesep{1em}

\bibliographystyle{newapa} %apalike


\tolerance=5000

\maketitle
%\tableofcontents

%\begin{abstract}
%\noindent
%***
%\end{abstract}


\section{Brief Description}
How do humans come to know about %
objects,
causes,
words,
numbers,
colours,
actions
and
minds? 
The question goes back to Plato or earlier and remains unanswered.
%Philosophers have been pursuing this question since Plato or before. 
%More recently it has become the focus of a branch of psychology, developmental psychology.
%So far no one has convincingly answered the question, but two major breakthroughs have been made in the last three decades or so.
Two recent scientific breakthroughs appear to bring us closer to an answer, and to show that the question is even less straightforward than philosophers have  assumed.
The first breakthrough is the discovery that preverbal infants enjoy surprisingly rich social skills, skills which
% begin to be built from the first months of life and 
may well be foundational for later linguistic abilities and enable the emergence of knowledge \citep[e.g.][]{Csibra:2009xr,Meltzoff:2007pj,Tomasello:2005wx}. 
A second breakthrough concerns the use of increasingly sensitive---and sometimes controversial---methods to detect  sophisticated expectations concerning  causal interactions, numerosity, mental states and more besides in preverbal infants \citep[e.g.][]{Spelke:1990jn,Baillargeon:gx}.
These expectations or the representations and processes underpinning them arguably also enable the emergence of knowledge. 
Although each breakthrough has been discussed at length by leading psychologists, philosophers have yet to consider either systematically in print.
Further, the two breakthroughs are associated with different camps, one nativist and the other Vygotskian, and have rarely been considered together as twin factors enabling the emergence of knowledge.
The proposed book will familiarise readers with findings related to both breakthroughs and explain their relevance to philosophical accounts of knowledge, action and experience.  
Most importantly, the book aims to introduce readers to  philosophical issues raised by these findings in cognitive development.


Advances in neuroscience may be transforming parts of developmental psychology much as they have already transformed the study of perception and action \citep{Johnson:2005az}. 
The proposed book focusses on discoveries rather than methods but will introduce readers to some relevant findings in developmental cognitive neuroscience along the way. 

The book is organised by domains of knowledge, so that one chapter concerns knowledge of objects, another focusses on  knowledge of number, and so on.  
The domains are chosen so that each set of developmental findings is linked to one or more philosophical issues. 
For instance, research on knowledge of objects gives bite to questions about modularity and the nature of tacit knowledge \citep{Davies:1989gg,Fodor:1983dg}; research on knowledge of number invites discussions of nativism \citep{Fodor:1981ep,Spelke:1998im}; and developmental findings on knowledge of colour may challenge some assumptions philosophers have made about relations between language, thought and perception \citep{Gilbert:2006yb,Regier:2009ve}.  





\section{Audience}

The book is aimed at advanced undergraduate and graduate students in philosophy. 
It is suited to teaching a course on philosophical issues in cognitive development. 
The book could also be used in courses on the philosophy of mind and action, or the philosophy of psychology. 
To this end the book will not presuppose familiarity with the philosophy of mind and will include chapter summaries and suggestions for further reading.

Professional philosophers whose research connects with issues in cognitive development may also find the book useful.  
Judging by papers and manuscripts under review for journals and university presses, even those writing about cognitive development can have difficulty identifying the full range of findings relevant to their positions.

Students in psychology are often drawn to philosophy and should find in this book an accessible introduction to key distinctions and arguments. 
It might even be used for courses in developmental psychology. (I once co-taught such a course in psychology with Jim Russell in Cambridge).





\section{Length}
70,000--100,000 words


\section{Timetable}
I aim to submit a first draft of the manuscript by 30 September 2014.  

I plan to try out material for the book while teaching courses at the University of Warwick (UK) and Central European University (Hungary).



\section{Competition}
At the time of writing there is no book or collection devoted to philosophical issues in cognitive development.

Philosophers have written monographs and edited collections on particular issues in cognitive development \citep[e.g.][]{Bermudez:2003dj,carruthers:1996_theories}.  
The proposed book complements these by providing a unified discussion of a wider range of topics. 
The advantage is not just convenience: understanding and properly evaluating theories about the developmental origins of knowledge requires bringing together research on different domains of knowledge.

There are also several collections bringing together research by philosophers and developmental psychologists or presenting developmental research in ways accessible to philosophers 
\citep[e.g.][]{hirschfeld:1994_mapping,carruthers:2005_innate_structure,carruthers:2006_innate_culture,mccormack:2011_tool}.
From the point of view of the proposed book, these provide useful sources of further reading for those who want more information on particular issues.

Finally, the proposed book will complement the many excellent books which expound particular theories about the origins of knowledge; these include 
	\citet{carey:2009_origin},
	\citet{Tomasello:1999xz},
	\citet{Gopnik:1997xq},
	and
	\citet{Elman:1996zd}.
The proposed book will include critical discussion of theories presented in these volumes.  
It's aims are clearly distinct from any of these volume's aims: the proposed book will review a range of existing theories and explore philosophical questions they raise.  







\section{Contents}
(Details are likely to change.)


\subsection{Introduction}
How do humans come to know  about---and to knowingly manipulate---%
objects,
causes,
words,
numbers,
colours,
actions
and
minds?
In pursuing this question we have to consider minds where the knowledge is neither clearly present nor obviously absent. 
This is challenging because both commonsense and theoretical tools for describing minds are generally designed for characterising fully developed adults. 
Davidson writes:
%
\begin{quote}
`%
%if you want to describe what is going on in the head of the child when it has a few words which it utters in appropriate situations, you will fail for lack of the right sort of words of your own. 
We have many vocabularies for describing nature when we regard it as mindless, and we have a mentalistic vocabulary for describing thought and intentional action; what we lack is a way of describing what is in between' \citep[p.\ 11]{Davidson:1999ju}.
\end{quote}
%
%Of course you might not agree with Davidson that children just starting to use words are incapable of thought or of intentional action. 
%In fact research on children's pre-verbal communicative behaviours such as pointing seems not only to manifest purposiveness but also a sensitivity to others' thoughts.
%(Davidson seems to have paid little attention to development, and he held that being able to think, to act intentionally and to communicate with language all require understanding the possibility that one's own beliefs may be false.)
%Even so, Davidson 
%
To understand the emergence of knowledge we need to find ways of describing what is in between: individuals whose movements are neither mindless nor guided by intention and knowledge.  

***


\subsection{Social Interaction before Knowledge}
Consider the hypothesis that social interaction is an enabler of cognitive development. 
This hypothesis immediately raises a question. 
***


\subsection{Objects and How They Interact}

\subsection{Number: From Perceptual Experience to Knowledge}

\subsection{Seeing and Talking about Colours}

\subsection{Words Are Tools for Communication}

\subsection{Actions: Teleology and Mirroring Motor Awareness}

\subsection{Mindreading}

\subsection{Conclusion}


\bibliography{$HOME/endnote/phd_biblio}

\end{document}