%!TEX TS-program = xelatex
%!TEX encoding = UTF-8 Unicode

\def \papersize {a4paper}
\documentclass[12pt,\papersize]{extarticle}
% extarticle is like article but can handle 8pt, 9pt, 10pt, 11pt, 12pt, 14pt, 17pt, and 20pt text

\def \ititle {A}
\def \isubtitle {}
\def \iauthor {Stephen A. Butterfill}
\def \iemail{s.butterfill@warwick.ac.uk}
%\def \iauthor {}
%\def \iemail{}
\date{}


\input{$HOME/Documents/submissions/preamble_steve_paper2}
%\author{}

%\setromanfont[Mapping=tex-text]{Sabon LT Std} 

\begin{document}

\setlength\footnotesep{1em}

\bibliographystyle{newapa} %apalike


\tolerance=5000

\maketitle
%\tableofcontents

\begin{abstract}
\noindent
***
\end{abstract}


\section{The puzzle}
\label{sec:puzzle}

Plan:

Simple theory of the mind

When can infants first represent objects they can't see

Generalisation

Preview of cases

Preview of remedies


\section{Why}
Why study philosophy of mind and developmental psychology together?

\begin{quote}
`Naturalism in philosophy …\ has the goal of articulating the application conditions of puzzling concepts (like knowledge and perception) so that empirical (scientific) methods can be used to answer questions  …\ Naturalism in epistemology is merely the attempt to get clear enough about what we mean when we talk about knowledge and perception to be able to tell—in ways a biologist or an experimental psychologist would recognise as scientifically respectable—whether what we are saying is true or not' \citep[p.\ x]{Dretske:2000ky}
\end{quote}

\bibliography{$HOME/endnote/phd_biblio}

\end{document}