%!TEX TS-program = xelatex
%!TEX encoding = UTF-8 Unicode

%a4paper or letterpaper (also used in preamble_steve_paper2
\def \papersize {a4paper}

\documentclass[12pt,\papersize]{extarticle}
% extarticle is like article but can handle 8pt, 9pt, 10pt, 11pt, 12pt, 14pt, 17pt, and 20pt text

\def \ititle {False Beliefs, Perspectives and Counterfactuals}
\def \isubtitle {}
\def \iauthor {Stephen A. Butterfill}
\def \iemail{s.butterfill@warwick.ac.uk}

%!TEX TS-program = xelatex
%!TEX encoding = UTF-8 Unicode

\title{\ititle\\\isubtitle}
\author{\iauthor\\<{\iemail}>}

\usepackage[\papersize]{geometry} % see geometry.pdf
\geometry{twoside=false}
\geometry{headsep=2em} %keep running header away from text
\geometry{footskip=1cm} %keep page numbers away from text
\geometry{top=3cm} %increase to 3.5 if use header
\geometry{left=4cm} %increase to 3.5 if use header
\geometry{right=4cm} %increase to 3.5 if use header
\geometry{textheight=22cm}

%non-xelatex
%\usepackage[T1]{fontenc}
%\usepackage{tgpagella}

%for underline
\usepackage[normalem]{ulem}

%get the font here:
% http://scripts.sil.org/CharisSILfont

\usepackage{fontspec,xunicode}
%nb do not explicitly use package xltxtra because this introduces bugs with footnote superscripting  -- perhaps because fontspec is supposed to include it anyway.
%UPDATE:  "You need to use the no-sscript option in xltxtra: \usepackage[no-sscript]{xltxtra}, this is explained in the documentation of xltxtra.  The issue is that Sabon does not contain true superscript glyphs for every character and the no-sscript option will instead use scaled regular glyphs, which is typographically inferior, but there is no other option available when using Sabon." --- http://groups.google.com/group/comp.text.tex/browse_thread/thread/19de95be2daacade
\defaultfontfeatures{Mapping=tex-text}
%\setromanfont[Mapping=tex-text]{Charis SIL} %i.e. palatino
%\setromanfont[Mapping=tex-text]{Sabon LT Std} 
%\setromanfont[Mapping=tex-text]{Dante MT Std} 
%\setromanfont[Mapping=tex-text,Ligatures={Common}]{Hoefler Text} %comes with osx
\setromanfont[Mapping=tex-text]{Linux Libertine O} 
\setsansfont[Mapping=tex-text]{Linux Biolinum O} 
\setmonofont[Scale=MatchLowercase]{Andale Mono}


%hyperlinks and pdf metadata
%TODO avoid duplication of title & author
\usepackage{hyperref}
\hypersetup{pdfborder={0 0 0}}
\hypersetup{pdfauthor={\iauthor}}
\hypersetup{pdftitle={\ititle\isubtitle}}


%handles references to labels (e.g. sections) nicely
\usepackage{varioref}

%line spacing
\usepackage{setspace}
%\onehalfspacing
%\doublespacing
\singlespacing

\usepackage{natbib}
%\usepackage[longnamesfirst]{natbib}
\setcitestyle{aysep={}}  %philosophy style: no comma between author & year

%enable notes in right margin, defaults to ugly orange boxes TODO fix
%\usepackage[textwidth=5cm]{todonotes}

%for comments
\usepackage{verbatim}

%footnotes
\usepackage[hang]{footmisc}
\setlength{\footnotemargin}{1em}
\setlength{\footnotesep}{1em}
\footnotesep 2em

%tables
\usepackage{booktabs}
\usepackage{ctable}

%section headings
\usepackage[sf]{titlesec}
%\titlespacing*{\section}{0pt}{*3}{*0.5} %reduce vertical space after header
%large headings:
%\titleformat{\section}{\LARGE\sffamily}{\thesection.}{1em}{} 
\titlelabel{\thetitle.\quad}

%captions
\usepackage[font={small,sf}, margin=0.75cm]{caption}

%lists
\usepackage{enumitem}
\newenvironment{idescription}
{ 	
	% begin code
	\begin{description}[
		labelindent=1.5\parindent,
		leftmargin=2.5\parindent
	]
}
{ 
	%end code
	\end{description}
}


%title
\usepackage{titling}
\pretitle{
	\begin{center}
	\sffamily
	\Huge
} 
\posttitle{
	\par
	\end{center}
	\vskip 0.5em
} 
\preauthor{
	\begin{center}
	\normalsize
	\lineskip 0.5em
	\begin{tabular}[t]{c}
} 
\postauthor{
	\end{tabular}
	\par
	\end{center}
}
\predate{
	\begin{center}
	\normalsize
} 
\postdate{
	\par
	\end{center}
}


%\author{}
%\date{}

%\setromanfont[Mapping=tex-text]{Sabon LT Std} 

\begin{document}

\setlength\footnotesep{1em}

\bibliographystyle{newapa} %apalike

\maketitle
%\tableofcontents
\title{}

\begin{abstract}
\noindent
***


\end{abstract}


\section{Introduction}
The challenge is to understand what mindreading is, how it develops and which cognitive processes underpin it.
One approach is to identify components of mindreading.
As I understand it, Eva's and Matthias' research does just this.
Eva argues that mindreading, at least insofar as it involves representing beliefs, may depend on counterfactual reasoning too.
And Mattias supports the claim that mindreading involves a separable ability to process perspective differences by showing that the neural correlates of this ability are found in both mindreading and non-mindreading tasks.

There appears to be a conflict between Eva's and Matthias' interpretations of their findings and a widely held view about infant mindreading (as Eva notes in the paper on which her talk is based).
Here are three claims which are collectively inconsistent:
%
\begin{enumerate}
\item Infants can represent false beliefs from around their first birthday or earlier. \label{infant_fb}

\item Being able to represent false beliefs involves being able to  (i) process perspective differences or (ii) reason counterfactually (or both). \label{fb_is_perspectives}

\item Infants cannot (i) process perspective differences nor (ii) engage in counterfactual reasoning until they are at least one year old.  \label{infant_perspectives}

\end{enumerate}
%
Since these claims are collectively inconsistent, we must reject at least one.  But which?


\section{The puzzle}
Can we reject (\ref{infant_fb}), the claim that infants can represent false beliefs?  
This is what  Eva suggests in her paper.
But there is a growing body of evidence which appears to support  (\ref{infant_fb}).
From around their first birthday infants  predict actions of agents with false beliefs about the locations of objects \citep[]{en_1092, en_1208} and choose different ways of interacting with others depending on whether their beliefs are true or false \citep[]{en_1783,Knudsen:2011fk,southgate:2010fb}.  
And in much the way that irrelevant facts about the contents of others’ beliefs modulate adult subjects’ response times, such facts also affect how long 7-month-old infants look at some stimuli \citep[]{en_1821}.
The variety of paradigms measures---looking time, anticipatory looking, pointing and helping---makes it hard to dismiss these findings on methodological grounds.
And while some have proposed that infants might be tracking behaviour only \citet{en_1168, en_1169},
this proposal faces several objections \citep[e.g.][]{Song:2008qo}.
The core issue is not whether hypothetical behavioural strategies might in principle explain theory of mind abilities; it is whether all of these subjects’ actual behaviour reading capacities are flexible enough to explain the full range of their theory of mind abilities \citep[]{en_1686}.
In my view this is unlikely.
So straightforward rejection of  (\ref{infant_fb}) is not an option.

Can we reject (\ref{fb_is_perspectives})?
Since Matthias' and Eva's talks are (in part) arguments for this claim, let me come back to it after first considering whether we can reject 
claim (\ref{infant_perspectives}).

Can we reject (\ref{infant_perspectives}), the claim that infants neither process perspective differences nor engage in counterfactual reasoning until they are at least one year old?
One difficulty is that current developmental evidence almost uniformly supports this claim  \citep{rafetseder:2010_counterfactual,beck:2011_almost}.
A second problem is that  abilities to reason counterfactually,%
\footnote{Here and throughout I follow Eva Rafetseder in using the term `counterfactual reasoning'  to refer to reasoning which essentially involves comparing two possible situations; the term is sometimes used more broadly to include the sorts of representations involved in pretence.} 
like abilities to represent false belief,
appear to involve working memory and inhibitory control 
(on counterfactuals: 
	\citealp{drayton:2011_counterfactual,beck:2011_supporting};
on false beliefs: \citealp{en_1412, en_1698, lin:2010_reflexively, en_1547} experiments 4-5).
As even committed nativists are likely to agree, capacities for inhibitory control and working memory develop over several years and are limited in infants \citep[e.g.][]{carlson:2005_developmentally}.
So we cannot straightforwardly reject (\ref{infant_perspectives}).

Since we can't straightforwardly reject (\ref{infant_fb}) or (\ref{infant_perspectives}),
let's consider whether (\ref{fb_is_perspectives}) can be rejected.
This is the claim that being able to represent false beliefs involves being able to process perspective differences or to reason counterfactually.
One might attempt to reject this claim on the grounds that the link between false belief, perspective differences and counterfactual reasoning 
is spurious and a consequence of \emph{extraneous demands} accompanying standard false belief tasks.
Here the idea is that, although certain false belief \emph{tasks} impose demand processing perspective differences or core components of counterfactual reasoning,
these demands are extraneous in the sense that they are not a consequence merely of being required to represent false beliefs
(proponents of this view include 
	\citealp[p.\ 417]{Carpenter:2002gc},
	\citealp{Bloom:2000bt}, and
	\citealp{Leslie:1998nq}).
I want to go slowly here and consider two versions of this idea.

Sometimes it is taken for granted that representing false beliefs does not necessarily and intrinsically involve inhibition;
and so the fact that success on many false belief tasks depends on inhibitory control is interpreted as revealing a defect of the task.
Someone who holds this view will interpret the evidence Eva and Matthias present as showing that the false belief tasks they consider are defective measures of the ability to represent false beliefs.
But I think it is a mistake to assume that representing false belief---or, indeed, representing any propositional attitude---does not intrinsically demand cognitive resources such as inhibitory control.
On any standard view, propositional attitudes such as beliefs form complex causal structures, have arbitrarily nest-able contents, interact with each other in uncodifiably complex ways and are individuated by their causal and normative roles in explaining thoughts and actions \citep[]{en_809, en_249}.  
If there is anything representing which should consume scarce cognitive resources it is surely states with this combination of properties.
So there are sound theoretical reasons to suppose that representing false beliefs could intrinsically require inhibitory control and working memory \citep[see also][]{Russell:1999vr}.
And of course Eva and Matthias can provide theoretical reasons for supposing that representing false beliefs involves processing perspective differences and a core component of genuine counterfactual reasoning \citep{perner:2007_objects}. 
This is why we should not take for granted that representing belief can be separated from processing perspective differences and counterfactual reasoning; there is no good theoretical reason to assume that the connections must reflect extraneous task demands. 

A less obviously flawed approach to rejecting (\ref{fb_is_perspectives}) involves analysing particular false belief tasks.
Two explore this idea it will be useful to distinguish between two types of false belief task, call them Category A and Category B.
By stipulation Category A tasks have these features:
\begin{itemize}
\item Children tend to pass them some time after their third birthday.
\item Abilities to pass these tasks has a protracted developmental course stretching over months if not years.
\item Success on these tasks is correlated with developments in executive function \citep[]{en_410, en_1130} and language \citep[]{en_1209}.
\item Success on these tasks is facilitated by explicit training \citep[]{en_85} and environmental factors such as siblings \citep[]{en_507, en_1299}.  
\item Abilities to succeed on these tasks typically emerge from extensive participation in social interactions (as \citealp{en_1300} suggest).
\end{itemize}
All the other false belief tasks are in Category B.
It is these Category B tasks on which infants succeed.
The proposal we are considering (and which I shall reject) is this.  
\begin{quote}
	All Category A tasks impose a requirement (or set of requirements) other than the requirement to represent a false belief;
call this the \emph{Extraneous Requirement}.
	The connections between success on Category A tasks and processing perspective differences and counterfactual reasoning are a consequence just of the extraneous requirement.
\end{quote}
I am going to argue that this proposal should be rejected.
First note that the proposal is harder to defend than sometimes assumed.
For instance, philosophers sometimes mistakenly assume that all Category A tasks involve language or communication, whereas Category B tasks do not.
But some Category B tasks involve language and communication 
\citep{Knudsen:2011fk,Song:2008qo}
and there are non-verbal tasks in Category A
(\citealp{Call:1999co}; \citealp{low:2010_preschoolers} Study 2).
We should also note that children who fail Category A tasks can answer questions about perception or pretence that are word-for-word identical with the questions about false belief that they cannot answer correctly (\citealp{Gopnik:1994cb}; see also\citealp{Cluster:1996ht}).
So the Extraneous Requirement cannot be straightforwardly linked to language or communication.
There is a more general problem facing attempts to identify the Extraneous Requirement.
There is much variation among the tasks in Category A.
It is not just that some are verbal whereas others are nonverbal.
It is also that some involve prediction whereas others involve retrodiction or justification \citep[e.g.]{Wimmer:1998kx},
some concern the first-person perspective, whereas others involve a second- or third-person perspective \citep[e.g.]{Gopnik:1991db},
some involve interaction whereas in others the subject is a mere observer \citep[e.g.]{Chandler:1989qa},
and some involve prediction actions whereas others involve predicting desires \citep{Astington:1991kk} or selecting an argument appropriate for someone with a false belief \citep{Bartsch:2000es}.
Despite all this variation and more, the Category A tasks all appear to measure a developmental transition \citep{Wellman:2001lz}.
This is a significant obstacle to identifying the Extraneous Requirement.
In fact, my view the currently available evidence supports the conclusion that there is no Extraneous Requirement.
And as far as I know the only plausible candidate for the transition  Category A tasks measure is a transition in the ability to represent beliefs.
This is why I don't think that we can straightforwardly reject  (\ref{fb_is_perspectives}).

To sum up so far, 
I have suggested that there are good reasons for each of (1)-(3) and 
argued that none of these claims can be straightforwardly rejected.
This leads to a puzzle.
Because (1)-(3) are collectively inconsistent, one of them must be rejected.
The puzzle is: Which should we reject? 
None of the arguments I have offered are decisive; what have aimed to establish is just that the question of which claim to reject is to a significant degree puzzling.
And although various researchers have taken one side or another,
I think it's fair to say that no one has yet succeeded in resolving the puzzle.


\section{How to resolve the puzzle}



\bibliography{$HOME/endnote/phd_biblio_en_record_num_keys}

\end{document}