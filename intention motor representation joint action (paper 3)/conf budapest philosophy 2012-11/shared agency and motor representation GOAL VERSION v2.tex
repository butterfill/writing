%!TEX TS-program = xelatex
%!TEX encoding = UTF-8 Unicode

\def \papersize {a4paper}

\documentclass[12pt,\papersize]{extarticle}
% extarticle is like article but can handle 8pt, 9pt, 10pt, 11pt, 12pt, 14pt, 17pt, and 20pt text

\def \ititle {Shared Agency and Motor Representation}
\def \isubtitle {}
\def \iauthor {CEU Philosophy, Budapest, 20 November 2012}
\def \iemail{s.butterfill@warwick.ac.uk}
%\date{}

\input{$HOME/Documents/submissions/preamble_steve_paper3}
%\author{}
%\date{}

%\setromanfont[Mapping=tex-text]{Sabon LT Std} 


\begin{document}	

\setlength\footnotesep{1em}

\bibliographystyle{$HOME/Documents/submissions/mynewapa} %apalike

\maketitle
%\tableofcontents
\title{}

\begin{abstract}
\noindent
Shared agency is paradigmatically involved when two or more people paint a house together, tidy the toys away together, or lift a two-handled basket together.  
To characterise shared agency, some philosophers have appealed to a special kind of intention or structure of intention, knowledge or commitment often called `shared intention'.  
In this paper we argue that there are forms of shared agency characterising which requires appeal to  motor representation.  
Shared agency is not only a matter of what we intend: sometimes it  constitutively involves interlocking structures of motor representation.  This may have consequences for some metaphysical, normative and phenomenological questions about shared agency.

\end{abstract}

%
%
%\section{***Discussion with Cordula Vesper}
%\begin{enumerate}
%\item Say more about what shared agency at the start: use the contrast cases instead of examples, and say that it involves some kind of fusing of agency.
%\item Note that in the jumping experiments, only the subject with the shorter jump changes what she does depending on the difference in jump-length between the two subjects.  Vesper et al interpret this as showing that the subject with the shorter jump plans the whole activity.  But it isn't obviously reciprocal: there is no evidence that the subject with the longer jump is similarly planning.  And this makes sense: here they achieve coordination by first distributing roles (a strategy) and then having at least one subject plan both actions (so there is slightly less reliance on the two plans coinciding).  
%Of course the appearance that they are distributing roles might just be an artefact of reciprocal planning where the upshot of the planning is that the person with the longer jump just does what she would normally do.
%
%\item {[done]}  Don't say `neither shared nor an intention' unless you explain this.
%\item {[done]} Can cite \citet{konvalinka:2010_follow} as example of what happens when two people each try to predict the other.
%\end{enumerate}
%
%
%
%\section{***Discussion with Hong Yu}
%`Once we move beyond the single agent there's no a priori reason to suppose that you have a single kind of agentive structure' (Hong Yu).
%This is slightly tricky for me because I don't think there's a single kind of agentive structure in individual action either.
%
%Dissimilarity between this paper and the paper on intention and motor representation: whereas it was maybe plausible to conjecture that all intentions bottom out in motor representation, it is not similarly plausible to claim that shared intentions bottom out in social motor representations (there are surely cases of shared agency where the sharing is entirely a matter of intention and does not involve motor representations at all).
%
%Hong Yu suggested that we need to include the argument about format in order to argue that motor representations have a role distinct from that of shared intention.
%
%In discussion: I need to shift this paper from saying:
%\begin{quote}
%There is a form of shared agency and that form cannot be characterised without reference to motor representation.
%\end{quote}
%to saying:
%\begin{quote}
%There are at least two forms of shared agency (at least two agentive structures) and some of these are such that characterising them involves appeal to motor representation.
%\end{quote}
%This shift reveals a new problem: we become more open to the objection that philosophy should concern forms of shared agency involving intention only.
%
%I need to stress the \emph{parity argument}.
%The argument is this:
%If 
%\begin{quote}
%
%	two or more agents' actions being appropriately related to a shared intention (as Bratman characterises it) is sufficient for shared agency, 
%\end{quote}
%then
%\begin{quote}
%	two or more agents' actions being appropriately related to a structure of social motor representation is also sufficient for shared 
%	agency.
%\end{quote}




\section{Introduction}
Shared agency is a familiar feature of everyday life, but it's hard to get a good fix on it without assuming a big chunk of theory.
I think the best available way to zoom in on shared agency is to think about contrast cases, that is pairs of cases which are as similar as possible except one involves shared agency and the other involves parallel but merely individual agency.

When members of a flash mob in the Central Cafe respond to a pre-arranged cue by noisily opening their newspapers, they exercise shared agency. 
But when someone not part of the mob just happens to noisily open her newspaper in response to the same cue, her action does not involve shared agency.%
\footnote{
See \citet{Searle:1990em}; in his example park visitors simultaneously run to a shelter, in once case as part of dancing together and in another case because of a storm. 
Compare \citet{Pears:1971fk} who uses contrast cases to argue that whether something is an ordinary, individual action depends on its antecedents. 
}
To give another example, 
 two former members of the mob exercise shared agency when they 
   later walk to the metro station together. 
But two people who merely happened to be walking to the metro station side by side would not be exercising shared agency \citep{gilbert_walking_1990}. 

I'll use the term \emph{joint action} for an exercise of shared agency, as contrasted with an \emph{individual action} which is an exercise of individual agency.



A basic question about joint action is,
What is the relation between a joint action and the outcome or outcomes to which it is directed?
Ayesha and Beatrice lift the table. ***

The standard answer to this question involves shared intention.
It's hard to say exactly what a shared intention is because each philosopher seems to take a different view.
But for my purposes it's enough to say that shared intention, whatever exactly it is, stands to joint action as ordinary, individual intention stands to ordinary, individual action.
Building on this parallel,
we can say that when two or more agents act on a shared intention that they free they cat,
\begin{enumerate}
\item the shard intention involves a representation on the part of each agent of this outcome, the freeing of the cat
\item the shared intention coordinates the several agents' actions
\item and the shared intention coordinates their actions in such a way that, normally, the coordination would facilitate the occurrence of the represented outcome. 
\end{enumerate}
In this way we can explain the directedness of a joint action to an outcome by appeal to shared intention.

There is another feature of shared intention that should be uncontroversial on any account of it.  
Shared intentions or their components feature in practical reasoning alongside ordinary, individual intentions,
 and there are normative requirements which apply to combinations of individual and shared intentions.
To illustrate, tonight there is a party and a ceremony.
It is impossible for anyone to attend both, and this is common knowledge among us.
We have a shared intention that we attend the ceremony together.
While having this shared intention, I also intend to go to the party. 
Give our common knowledge,
this combination of shared and individual intentions is irrational. 
Its irrationality is related to that which would be involved in my individually intending to attend the ceremony while also intending to go to the party.
In short, shared intentions or their components are inferentially and normatively integrated with ordinary, individual intentions.

In this talk I'm going to argue that in some cases of joint action,
the directedness of the joint action to an outcome cannot be explained in terms of shared intention.
In some cases, the directedness of a joint action to a goal needs to be explained in terms of a special structure of motor representation.
I'll call this \emph{shared motor representation}.
As we'll see,
shared motor representation \textbf{resembles} shared intention in that if our actions are controlled by a shared motor representation, then:
%
\begin{enumerate}
\item there is a single outcome which we each represent 
\item and these representations coordinate our actions
\item and coordinate them in such a way that, normally,  the coordination would facilitate the occurrence of the represented outcome. 
\end{enumerate}
%
But shared motor representation \textbf{differs} from shared intention in not integrating in practical reasoning with ordinary individual intention.
There are normative constraints linking ordinary intentions to shared intentions,
but no  normative constraints link intention to shared motor representation.

I should say, right at the start, that this speculative philosophy.
The view inspired by some findings which provide some support for it, but the findings don't establish that the view is correct.

This then is my thesis.
To understand shared agency we need to understand the relation between joint actions and the  outcomes to which they are directed.
And understanding this relation requires appeal, not only to structures of intention, commitment and knowledge, 
but also to structures of motor representation.
Shared agency is not always only a matter of our intentions or commitments: in some cases it also constitutively involves certain structures of motor representation.
Or so I aim to show.


This conflicts with the view that understanding shared agency requires understanding only shared intention.
So I disagree with Facundo Alonso's statement that:
%
\begin{quote}
`the key property of joint action lies in its internal component [...] in the participants' having a ``collective'' or ``shared'' intention' \citep[pp.\ 444-5]{alonso_shared_2009}.
\end{quote}
%
But my disagreement is not terribly radical.
I'm not denying that appeal shared intention is also needed in giving an account of shared agency,  
%and with the claim that \emph{some forms} of shared agency can be fully characterised by appeal to shared intention alone.
and I'm not even disagreeing here with those who claim that all exercises of shared agency involve shared intentions in some way.
The point is just that we can't fully understand shared agency in terms of intention, commitment and knowledge:
instead, it's also necessary to invoke motor representations.



Before I start on this,
why should anyone care?
Shared agency raises a tangle of scientific and philosophical questions.  
%
\begin{enumerate}
\item Psychologically we want to know which mechanisms make it possible \citep{Sebanz:2006yq,vesper_minimal_2010}.  
\item Developmentally we want to know when shared agency emerges, what it presupposes and whether it might somehow facilitate socio-cognitive, pragmatic or symbolic development \citep{Moll:2007gu,Hughes:2004zj,Brownell:2006gu}.  
\item Phenomenologically we want to characterise what (if anything) is special about experiences of action and agency when shared agency is involved \citep{Pacherie:2010fk}.  
\item Metaphysically we want to know what kinds of entities and structures are implied by the existence of shared agency \citep{Gilbert:1992rs,Searle:1994lb}.  
\item And normatively we want to know what kinds of commitments (if any) are entailed by shared agency and how these commitments arise \citep{Roth:2004ki}.
%, plus a formal account of how practical reasoning for joint action differs (if at all) from individual practical reasoning \citep{Sugden:2000mw,Gold:2007zd}
\end{enumerate}
%
Maybe identifying a constitutive role for motor representation in shared agency will, in a small way,
 support investigation of these questions.  
%Or, if there is more than one form of shared agency, the account should provide a principled way of distinguishing among forms of individual and shared agency.   



\section{The theory (non-motor)}
I want to start by sketching the theory in a way that abstracts from motor representation,
  so that you can see the general outlines.
  This will involve introducing a series of ideas.  
We'll come to the evidence only right at the end.



\subsection{*distributive goal}
Two or more agents' actions have a distributive goal just if ...

Contrast cases: don't have distributive goal.

But some actions involving parallel but merely individual agency do involve distributive goals.  
One dark night two people each independently intend to paint a large bridge red.   
More exactly, each intends that her painting grounds or partially grounds the bridge's being painted red.\footnote{
Event $D$ \emph{partially grounds} event $E$ if there are events including $D$ which ground $E$.
(So any event which grounds $E$ thereby also partially grounds $E$; 
we nevertheless describe actions as `grounding or partially grounding' events for emphasis.)
See the definition of \emph{plural grounding} *.
}  
(These intentions ensure that it is possible for both people to succeed in painting the bridge, as well as for either of them to succeed alone.)
Because the bridge is large and they start from different ends, the two people have no idea of each other's involvement until they meet in the middle.
Nor did they expect that anyone else would be involved in painting the bridge red.  
On almost any account, this implies that they were not acting on a shared intention.
Despite this, 
they both succeed in painting the bridge red. 
As this illustration suggests, 
it is possible for two or more agents' actions to have a distributive goal without them thereby exercising shared agency.
So distributive goals are not shared intentions.




\subsection{*collective goal}
Let an outcome, possible or actual, be a \emph{collective goal \label{df_collective_goal}} of a joint action, or of any collection of goal-directed actions, where three conditions are met: 
	(a) this outcome is a distributive goal of the actions; 
	(b) the actions are coordinated; and 
	(c)  coordination of this type would normally  facilitate occurrences of outcomes of this type.  
Examples of actions  that typically have collective goals include two people jointly sawing a log with a two-handled saw and  
three people jointly lifting a heavy table.
The bridge painters (from section *) are different: their actions do not have a collective goal because they are not coordinated.


Where two or more agents' actions have a collective goal there is a sense in which, taken together, their actions are directed to the collective goal.  
It is not just that each agent individually pursues the collective goal; in addition, there is coordination among their actions which plays a role in bringing about the collective goal.  
We can put this in terms of the direction metaphor.  
Any structure or mechanism providing this coordination is directing the agents' actions to the collective goal.  
So the notion of a collective goal provides a schematic answer to the question about the relation between joint actions and the goals to which they are directed.

The notion of a collective goal assumes that of coordination.  This should be understood in a broad sense.  
When two agents between them lift a heavy block by means of each agent pulling on either end of a rope connected to the block via a system of pulleys, their pullings count as coordinated in this broad sense.  
In this case, the agents' actions are coordinated by a mechanism in their environment, the rope, and not necessarily by any psychological mechanism.  
By invoking a broad notion of coordination 
and invoking coordination of actions rather than of agents,
the definition of collective goal avoids direct appeal to psychological states.

%as well as non-psychological factors such as the dynamical properties of agents' bodies \citep[e.g.][]{schmidt_richardons:_2008}.
To make a conjecture based on work with bees and ants, in some cases
the coordination needed for a collective goal may even be supplied by 
	 behavioural patterns \citep{seeley2010honeybee}  
	 and 
	 pheromonal signals \citep[pp.\ 178-83, 206-21]{hoelldobler2009superorganism}.


This is not to say that collective goals never involve psychological states.
%The notion of a collective goal is more abstract than shared intention and other notions that have been used in characterising joint action.
%On many or all accounts, shared intentions function in part to coordinate actions in such a way as to facilitate realisation of the shared intention.
In fact,
one way for several actions to have a collective goal is for their agents to be acting on a shared intention; 
a shared intention supplies the required coordination.

But the possibility we are interested in is that the coordination required for a collective goal involves structures of motor representation rather than shared intention.
How could this work in principle?


*Aim: Describe reciprocal agent-neutral planning for outcomes whose obtaining would normally involve action on the part of each agent.

\subsection{Agent-neutral}

First we need to introduce the notion of an agent-neutral plan.
By saying that something is \emph{agent-neutral} I mean just that it does not involve identifying any particular agents.  

%Having an agent-neutral plan makes perfect sense if you're stranded on a desert island because, after all, there it's obvious that you're going to do everything so there's no need to specify who will bring in the washing.

Agent-neutral planning for outcomes whose realisation would normally involve action on the part of several agents is also quite common.
For example, some housemates who have decided to take on an allotment to grow vegetables might sit down together to plan what needs doing without yet assigning roles to particular individuals. 
In so planning, each housemate is thinking about what is to be done and not what she herself will do.  
%(This is an idealisation, of course.)
At some point the housemates stop planning.
(This does not necessarily mean that they have a fully worked out plan; like any other plans, agent-neutral plans can have gaps that may need filling in later.)
They now divide up the roles in a way that everyone is prepared to go along with,
	and that each implements her part in the plan.



In the above example of agent-neutral planning,
	the housemates \textbf{plan together} and \textbf{agree} on a common plan.
But an individual can construct an agent-neutral plan by herself, even if its eventual execution will involve others.
	In fact, two or more individuals who are assigned a task might each individually engage in agent-neutral planning in parallel.
	
Of course, for us to engage in reciprocal, parallel agent-neutral planning concerning a particular outcome only makes sense if the  plans we construct will be identical or similar enough that differences don't matter.
And, of course, we must also all assign the same or similar roles to ourselves and each other.
So reciprocal, parallel agent-neutral planning only makes sense where: 
\begin{enumerate}
\item 	we have relevantly similar planning strategies and expertise
\item there is one salient way for us to achieve the outcome; and
\item there is one salient way for us to assign roles to each other.
\end{enumerate}

  
	The task demands and their planning strategies may conspire to ensure that they each come up with the same agent-neutral plan.
	The task demands and manifest properties of the agents, such as their distribution in space, may also ensure that each agent also assigns the same roles to the same individuals.
(Strictly speaking it is not necessary for the plans and role assignments to be identical; it is enough if the resulting agent-specifying plans are, in a special sense, compatible.%
\footnote{
\label{fn:df_compatible}
Suppose that, for some outcome, two or more agents each have a plan for the realisation of that outcome. 
(These plans may, but need not, specify roles for all of the agents; but the plans must be agent-specific, not agent-neutral.)
By saying that these plans are \emph{compatible} we mean that:
(i) 
no agent would normally be prevented from performing the role she is assigned in her own plan by other agents performing the roles they are assigned in their plans;
and
(ii)
if all facts about which agents have which roles in which plans were common knowledge to the agents,
this would not affect the rationality of their each acting on the intention that they realise the outcome by performing the role she is assigned in her own plan.
%	each agent could rationally act on the intention that they realise the outcome by performing the role she is assigned in her own plan,
%	and she could do so even if all facts about which agents are performing which roles in which plans were common knowledge to the agents.
To illustrate, suppose that our task is to press a button simultaneously. 
If your plan is that each of us will to start to move in exactly 60 seconds and press the button 5 seconds later whereas my plan specifies that you are the leader and we are each to press the button 5 seconds after you start moving, then our plans are compatible.
})
%
Finally, each agent may know enough about herself and the others to be able to determine, without communicating, whether the plan and role assignments will be acceptable to everyone. 
	And all of this---that they engage in parallel, agent-neutral planning resulting in identical (or compatible) plans and role assignments, which are acceptable to all---may be common knowledge to the agents.
	So it is possible, in principle at least, 
	that several agents might each individually engage in agent-neutral planning and rationally perform their part in the resulting plan, knowing that the others will do likewise.
	Parallel, agent-neutral planning can rationally result in coordinated action without presupposing shared agency.%
\footnote{
Note that we are claiming only that shared agency is not presupposed.
Our view is consistent with (but does not depend on) the claim
that
	if some agents each engage in parallel, agent-neutral planning and then rationally perform their part in the resulting plan,
	the upshot would be an exercise of shared agency.
}





\subsection{*}
Having each agent plan the whole joint action means that (i) each agent plans the other agent's action,
(ii) each agent's plan for the other agent's action is approximately the same as that agent's plan for her own action [***AMBIGUOUS: `that agent's plan' must refer to the other agent, not `each agent',
and 
(iii) each  agent's plans for their own action are constrained by the plans for the other agent's action.

\textbf{So what enables the two agents' plans to mesh is not that they represent each other's plans but more simply that they plan each other's actions as well as their own actions as if they were each about to do the whole thing themselves.}

Each agent is planning (and monitoring) both their actions almost as if a single agent were going to execute the whole action.
And of course this is exactly what we want for small-scale joint action---we want two or more agents to act as one.

So what is the difference between the individual and the joint case?  From the present point of view, the primary difference may be that in joint action there is a need to prevent execution of the parts of the action which are not one’s own.

So far: shown theoretically how there could be directedness of joint action to outcome by appeal to reciprocal representations of outcomes which trigger agent-neutral planning.

But unclear that any such representations exist, or at least that they plan any significant role in ordinary cases of joint action.
What makes this theoretical possibility interesting is evidence that these representations occur in motor control of action.

***section: say what motor processes are; examine what would be evidence for different points.


\section{Shared motor representation}
Our argument starts with an empirical premise about enabling conditions for joint action: there are reciprocal, agent-neutral motor representations of outcomes whose obtaining would normally involve action on the part of each agent;
moreover, these structures of motor representation sometimes facilitate coordination when agents exercise shared agency.
This needs unpacking. 
So before considering evidence for this premise (in the next section), let us first explain  it.

A \textit{motor} representation is the sort of representation that enables agents to reach for, grasp and transfer objects in a coordinated and fluid way. 
Motor representations play a key role in monitoring and planning actions  \citep[e.g.][]{wolpert:1995internal, miall:1996_forward,Wilson:2005qu}.
Unlike intentions and related mental states, motor representations are usually involved in planning actions over milliseconds rather than minutes or even years.

\textbf{*** explain motor representation in terms of passing object from one hand to the other}

Some motor representations resemble intentions  in 
	representing outcomes (rather than merely kinematic or dynamic features of action), 
	coordinating multiple  component activities by virtue of their role as elements in hierarchically structured plans,
	and coordinating these activities in a way that would normally facilitate the represented outcome's occurrence (\citealp{hamilton_action_2008}; \citealp[pp.\ 189-90]{pacherie:2008_action}; \citealp{butterfill:2012_intention}).
Despite these points of similarity, 
motor representations can be distinguished from intentions and other action-related intentions,
	 as well as from perceptual representations,
	 by their representational format \citep{butterfill:2012_intention}. 

Motor representations lead a kind of double life.
For motor representations are involved not only in producing actions but also in observing actions. 
Indeed there are some striking similarities between the sorts of processes and representations usually involved in performing a particular action and those which typically occur when observing someone else perform that action.
In some cases it is almost as if the observer were planning the observed action, only to stop just short of performing it herself.%
\footnote{ 
For reviews, see \citet{jeannerod_motor_2006,rizzolatti_mirrors_2008,rizzolatti_functional_2010}.
If motor representations occur in action observation, then observing actions might sometimes facilitate performing compatible actions and interfere with performing incompatible actions.  Both effects do indeed occur, as several studies have shown \citep{brass:2000_compatibility, craighero:2002_hand, kilner:2003_interference, costantini:2012_does}. 
}
When motor representations of outcomes trigger a planning-like process in action observation, this may allow the observer to predict others' actions \citep{Flanagan:2003lm,ambrosini:2011_grasping,ambrosini:2012_tie,Costantini:2012fk}.




A representation (motor or not) is \emph{agent-neutral} if its content does not specify any particular agent or agents.%
\footnote{
Our use of the term `agent-netural' to describe motor representations bears no relation to the use of the same term  to describe reasons \citep[on the latter, see][]{Parfit:1984fk}.
}
To illustrate, agent neutral representations are sometimes found at the early stages of planning.
Imagine that you and some friends are tasked with preparing a holiday.  
You might first write down a plan of action without specifying who will act; the plan simply describes what is to be done.
The plan  will eventually be implemented by you and your friends
 but this is not written in plan itself  and so it is agent-neutral.
Of course the fact that this plan is your collective plan may be represented elsewhere; this fact may also be implicit in  the plan's being stapled to the door of your communal kitchen.
The agent-neutrality of a representation does not require that the agents are nowhere specified, only that they are not specified in the content of the representation.%
\footnote{
Strictly the following argument does not hinge on the agent-neutrality of representations.
It is sufficient for our purposes that there are reciprocal motor representations  of outcomes whose obtaining would normally involve action on the part of each of the reciprocating agents.
In principle such representations could have contents which specify particular agents. 
However, we focus on agent-neutral representations to show that our view is consistent with the possibility that reciprocal motor representations are agent-neutral. 
}

Two or more agents have \emph{reciprocal} representations (motor or not) just if there is a single outcome and each agent has a motor representation of that outcome. 
Reciprocal representations can occur both when one agent observes another acting alone and also when two or more agents act together.
Where one agent observes another acting alone,
it is possible that both have motor representations of an outcome to which the agent's action is directed.
For the agent, this representation plays a role in planning and monitoring; for the observer, this representation may facilitate prediction and recognition.
This is one case of reciprocal motor representation---or, as it is more usually called, mirroring.
But our main interest is in a different case,
one where two or more agents are acting together,
 there is a single outcome to which their actions are directed,
 and the agents each have a motor representation of this outcome.
In this case of reciprocal motor representation, the agents each represent an outcome whose obtaining would normally involve not only their own actions but also those of other agents.%
\footnote{ 
	Note that our having reciprocal motor representations would not by itself imply that we know that the represented outcome's obtaining would normally involve actions other than our own.
}
So when we act together, some of my motor representations may concern outcomes that are partly but not entirely to be realised by my actions; and likewise for you.
To illustrate, suppose our task is to move an object from A to B, where you pick it up and pass it to me so that I can then place it.
In this case I may represent the movement of the object from A to B and not only the component movements.
I represent a collective outcome of our actions and not just outcomes to which each of our actions are individually directed.
%
%So it is not just that, when we act together, I represent the outcomes to which your actions are directed and the outcomes to which my actions are directed; I may also represent an outcome to which all of our actions are directed.

We shall use the term \emph{shared} motor representation as an abbreviation for the reciprocal, agent-neutral motor representation of outcomes whose obtaining would normally require action on the part of each reciprocating agent. 
Note that shared motor representations are shared only in the sense in which two people can share a name.  
(We do not mean to suggest that they are shared in the sense in which two people can share a parent: there need be no representation with two or more subjects.)
Note also that a shared motor representation is not a special kind of motor representation: it is merely a structure of ordinary motor representations. 
At this stage we are not strictly entitled use the term `shared motor representation' as an abbreviation.
After all, we have yet to show that these structures of motor representation play a constitutive role in explaining shared agency, or even an enabling role; 
we have yet to review evidence that they exist.
So the choice of term merely reflects our aim. 




\section{Interpersonal coordination and motor representation}

These bits of evidence are relevant (***):
%
\begin{enumerate}

\item \citet{kourtis:2012_predictive} shows that motor planning can occur for others' actions when we are engaged in joint action with them.

\item Vesper's ESPP paper (on jumping together and imagining jumping together --- jumping is published \citep{vesper:2012_jumping}, imagining jumping is about to be submitted (as of August 2012).  This shows that individuals are capable of running motor simulations of multiple roughly simultaneous actions. 
(The important point for me is that one can simulate roughly simultaneous actions, not that the simulations are simultaneous.)

\item The GROOP effect shows that there are representations which specify each agent's task in relation to the other (so they are not simple representing the outcomes to which each of their actions are directed; they are representing an outcome to which their actions taken together are directed.)

\item Vesper says forthcoming EEG paper using piano playing paradigm on agent-neutral identification of error: one brain wave signals whether there is an error, and a different brain wave signals whose error it is (also tells you whether the overall harmonics are affected)
\end{enumerate}


It is hardly controversial that reciprocal motor representations exist, for their existence is suggested by by a large body of research on motor cognition in action observation.
[***Rest of this paragraph belongs in the evidence section?
Should be clearer about transition:
(a) I am observing and we reciprocally represent an outcome of your action (mirroring).
(b) We are interacting and reciprocally represent an outcome of your action (mirroring in joint action).
(c) We are interacting and reciprocally represent an outcome to which our actions are distributively directed.%
] 
It is more controversial that reciprocal motor representation occurs in joint action, but there is some evidence for this claim too.%
\footnote{
See \citet{kourtis:2012_predictive}: `the partner’s expected action is simulated at the motor level, which probably facilitates effective performance of the joint action.'
\citet{kourtis:2010_favoritism} show that 
reciprocal motor representation is more likely to occur in joint action than is mere observation. 
See also \citet{Knoblich:2003nf}.
 }









What follows is speculative philosophy: we now take for granted that  social motor representation sometimes facilitates the exercise of shared agency and ask whether this matters for a philosophical account of shared agency. 
As a first step, we shall ask how social motor representation might facilitate agents in coordinating their actions.
Note that answering this question does not directly commit us to any view about shared agency.
After all, social motor representation might sometimes facilitate coordination while being extraneous to a philosophical account of shared agency.


\section{How could social motor representation coordinate actions?}
Suppose that social motor representation is sometimes present when two or more agents act together.
It doesn't follow, of course, that social motor representation will play any role in coordinating the agents' actions.
But let us consider just the possibility that it might.
How could this happen?
How could social motor representation play a role in coordinating action when two or more agents act together?

As a first step towards answering  this question let us illustrate a general principle about planning by scaling up from motor action all the way to corporate action.

In small city state, Aravinda organises the trams and Gerhard the busses.
Each is responsible for devising and updating the timetable for the service.
Let us stipulate that the extent to which the tram and bus services are coordinated is to be measured by the sum of all passengers' total journey times.
So a change to the timetables would result in better coordination  just if it would decrease the sum of passengers' total journey times.
To illustrate, improving coordination might involve having trams arrive at a stop shortly before, rather than shortly after, busses depart from there.
Now Aravinda and Gerhard each intend they, Aravinda and Gerhard, coordinate the trams with the busses  to the greatest extent possible given other constraints; and they intend to do this by way of these intentions and meshing subplans of them, and this is common knowledge between them. 
Aravinda and Gerhard thus meet Bratman's sufficient conditions for them to have a shared intention.
Acting on this intention, 
every January Aravinda updates the tram timetables and sends Gerhard the changes.
Likewise, Gerhard  updates the bus timetables every June and sends Aravinda the changes.
In this way  each is responsive to the other's intentions and subplans of these. 
Indeed, each may even try to predict changes in the other's subplans and modify their own accordingly.  
But from each individual's point of view, the other's plans are  merely a constraint.

This approach to planning suffers from two defects.
First, it is unlikely to be optimal in the sense of resulting, eventually, in a combination of plans such that no other combination of plans would have been better for at least one service and no worse for either service.
Second, it is unlikely to be efficient in the sense of allowing Gerhard and Aravinda to arrive at an optimal combination of plans for the two services, trams and buses, with the fewest iterations.%
% --- I removed this footnote because the case is a bit different
% --- In Konvalinka, the subjects have to act simultaneously so they can't wait to see what the other does
%\footnote{
%The limits of this type of coordination, where each agent tries to predict the actions of the other, is nicely illustrated by the `two-way interaction' condition of an experiment by \citet{konvalinka:2010_follow}.
%When subjects were required to synchronise the timing of finger taps with each other in this condition, 
%%each had to predict the others' action and modify their own accordingly. 
%they ended up with a pattern of errors `oscillating in opposite directions' as each attempted to move towards the other (p.\ 2228).
%}

How could they do better?
One possibility may be to have a single person planning the busses and trams---perhaps, for example, Aravinda could buy Gerhard's franchise. 
But suppose that this is not possible, and that there are limits on how much information Aravinda and Gerhard can share. 
Then another way that might improve things would be to have Aravinda and Gerhard each plan everything.
So Aravinda would plan the best timetable for all the trams and busses, and Gerhard would do the same.
Since Aravinda only controls the trams, she can only implement the tram-related part of this master timetable.
Likewise, Gerhard can only implement the bus part of his master timetable.
But if Aravinda's and Gerhard's separate planning processes are sufficiently similar, this process may result in coordinated services.
In this new case, 
Aravinda and Gerhard coordinate their actions 
not by knowing or predicting each others' plans 
but by thinking about the best overall plan.
Instead of each viewing the others' plans and actions as constraint on their own,
in planning actions they absorb the others' actions.

The question was how social motor representation might play a role in coordinating action.
We propose that it might do so in the way illustrated by Aravinda and Gerhard's final attempt at coordination,
where each plans everything despite only being in a position to execute half of the plan.




%Let us go very slowly here.
%Suppose that, in moving the mug, you try to locate two separate plans:
%a plan for your left hand and a plan for your right hand.
%In the Left Plan (as we might call it), the movements of the right hand feature as a constraint.
%And conversely for the Right Plan.
%It is almost as if you two hands are under the control of two separate agents who periodically share information about plans.
%So if you update the Right Plan, you will need to make compensating changes to the Left Plan.
%And of course those compensating changes to the left plan may in turn trigger further changes to the Right Plan.
%And so on.
%This approach might work in some cases. 
%But it imposes two costs.
%First, it is unlikely to be efficient in the sense of minimising the range of possible combinations that are considered. 
%Second, it is unlikely to be optimal in the sense of finding a combination of plans such that no other combination of plans would have been better for at least one hand and no worse for any hand.
%How could you have done beter?
%The mistake was to have two separate plans, Left and Right, in each of which the movements of the other hand are a constraint.
%The better 


To see how this might work,
let us step down to motor action but first consider only  an individual action.
Suppose an agent moves a mug from one place to another, passing it from her left hand to her right hand half way.
It is a familiar idea that motor planning, like planning generally, involves starting with relatively abstract representations of outcomes and gradually filling in details.
We can capture this by supposing that 
motor representations for planning and monitoring action involve a hierarchical structure of representations.
At the top we might find a relatively abstract representation of an outcome, in this case of the movement of the object from one location to another.
Action-relevant details are progressively filled in by representations at lower stages of the hierarchy. 
Now in the action we are considering there is a need, even for the single agent, to coordinate the exchange between her two hands.
How is this achieved? 
We suppose that part of the answer involves the fact that planning for the movements of each hand is not done entirely independently.
Rather there is a plan for the whole action
and plans for the movements of each hand are components of this larger plan.
It is in part because they are parts of a larger plan that the plan for one hand constrains and is constrained by the plan for the other hand.





How is this relevant to the case of joint action?
In joint action the agents have the same goal, to move the object from one place to another.
They also face a similar coordination problem, requiring a precisely timed swap from one hand to another.
Now suppose, that the same planning is involved in the individual case (where one agent performs the whole action) and in the joint action case (where the action is distributed between two agents).
The planning is the same almost up to the actual muscle contractions.

How could this be helpful?
Suppose the agents' planning processes are similar enough that, for a given context and problem, they will produce approximately the same plans.
Then having each agent plan the whole joint action means that (i) each agent plans the other agent's action,
(ii) each agent's plan for the other agent's action is approximately the same as that agent's plan for her own action [***AMBIGUOUS: `that agent's plan' must refer to the other agent, not `each agent',
and 
(iii) each  agent's plans for their own action are constrained by the plans for the other agent's action.

[***TODO: (a) contrast this case with a team of experts, each with different motor expertise (e.g.\ musicians playing together).  
They can't plan each others' actions.
(b) Discuss in how much detail each others' actions should be planned so as to enable coordination.
]


So what enables the two agents' plans to mesh is not that they represent each other's plans but more simply that they plan each other's actions as well as their own actions as if they were each about to do the whole thing themselves.

Each agent is planning (and monitoring) both their actions almost as if a single agent were going to execute the whole action.
And of course this is exactly what we want for small-scale joint action---we want two or more agents to act as one.
This may be why the performance of dyads in joint actions often resembles the performance of individuals tasked with performing the whole action alone \citep{Knoblich:2003nf}.

So what is the difference between the individual and the joint case?  From the point of view of motor representation, the primary difference may be that in joint action there is a need to prevent execution of the parts of the action which are not one’s own.

%Here then is the basic idea I take to be guiding Kourtis and others.
%Coordination is sometimes achieved by having each agent’s motor system plan all of their actions; 
%given some assumptions, this could be a way of making it likely that each will execute their part in the joint action in a way that meshes with the way the other agents execute their parts.


\section{Grounding the purposiveness of joint action}
So far we have only been considering a possible role for social motor representation in enabling joint action.  
How does any of this bear on our main question about shared agency?
The details of how 
social motor representation enables joint action
 already give us grounds for holding that motor representation has a role to play in explaining shared agency.

Here are two basic question about  joint action are.
%What is the relation between a purposive joint action and the outcome or outcomes to which it is directed?
What singles out the outcome or outcomes to which a purposive joint action is directed?
And what binds together the various activities (of several agents) that make up the joint action?
%A difference in the case of joint action is, of course, that the component activities are not activities of a single agent.

If we appeal to a notion of shared intention,
we can answer these questions about joint action.
A shared intention is what relates purposive joint actions to the outcomes to which they are directed.
For the shared intention 
involves a representation, on the part of each agent, of an outcome,
coordinates the several agents’ activities
and 
coordinates the several agents’ activities in such a way that would normally facilitate the occurrence of the represented outcome.
This is how a shared intention can bind together the activities comprising a joint action and link them to an outcome.
%Shared intention does for joint action roughly what ordinary intention does for ordinary, individual action.

% [individual]	intention : motor representation
% 					::
% [joint]			shared intention : social motor representation

Our earlier discussion of how social motor representation might enable joint action already shows that social motor representation resembles shared intention in this respect.
Return to the example of two agents moving an object in a way that involves passing it between them.
Suppose that their passing involves a social motor representation of the outcome,
%reciprocal agent-neutral motor representations of the outcome, 
which is the movement of the object. 
Then there are motor representations, one for each agent, 
of an outcome to which the joint action is directed.
And these representations coordinate the several agents' activities,
and 
do so in ways that would normally facilitate the occurrence of the  outcome represented.%
\footnote{
This implies that social motor representation and the associated processes underwrite what \citet{Butterfill:2011_wija} calls \textit{collective goals}.
}
So social motor representation can bind together the activities comprising a joint action and link them to an outcome in much the way that shared intention can.

%[*use above?] Because each agent represents the whole movement and plans all of its implementation irrespective of which parts she will actually perform, each agent plans the action in a way that should coordinate with the other agent's plans providing they use similar planning procedures

%[*What I’m saying here, in effect, is that both shared intention and social motor representation can yield a COLLECTIVE GOAL]

What we are suggesting is very simple.
Given the correctness of a standard view about shared intention in joint action, 
and 
given that in ordinary, individual action, motor representations  bind together activities and link them to outcomes,
it is plausible that 
in joint action, several agents' activities can be bound together and linked to an outcome by social motor representation.
That is,
the purposiveness of a joint action can be grounded not only in shared intention, but also in social motor representation.
This is why we suppose that an account of shared agency must appeal not only to shared intention but also to social motor representation.


\section{How social motor representation resembles shared intention}
It may be helpful to compare and contrast the notion of social motor representation with a notion of shared intention. 
We shall use Bratman's account of shared intention as it is the best developed. 
Here are Bratman’s collectively sufficient\footnotemark \ conditions for you and I to have a shared intention that we J:
%
\footnotetext{
In \citet{Bratman:1992mi}, the following were offered as jointly sufficient \textit{and individually necessary} conditions; the retreat to sufficient conditions occurs in \citet[][pp.\ 143-4]{Bratman:1999fr} where he notes that `for all that I have said, shared intention might be multiply realizable.'
} 
%
\begin{quote}
\label{quote:bratman_account}
`1. (a) I intend that we J and (b) you intend that we J
 
`2. I intend that we J in accordance with and because of la, lb, and meshing subplans of la and lb; you intend that we J in accordance with and because of la, lb, and meshing subplans of la and lb
 
`3. 1 and 2 are common knowledge between us' \citep[][p.\ View 4]{Bratman:1993je}
\end{quote}
%
Let us take each of these three conditions in turn.

To see a parallel with the first condition, (1), recall two (empirical) claims on which the notion of social motor representation is based.
First, some motor representations represent outcomes.
Second, some motor representations represent the outcomes of actions not all of whose components will be executed by the agent whose motor representation it is.
Given these claims, there is a direct parallel with Bratman's first condition, (1).
Where some agents have either a shared intention or a social motor representation, there is an outcome to which their actions are directed and each agent represents this outcome.
Of course there is also a difference: In the case of social motor representation, the outcome is represented motorically and need not feature in the content of any intention.%
\footnote{
Here and below were are assuming that no motor representations are intentions. 
If this assumption is wrong (as \citealp{pacherie:2008_action} suggests), social motor representation may be even more closely related to shared intention that we suggest here.
}


Concerning the second condition, (2), there is clearly no {direct} parallel. 
Whereas one intention can be about another intention, 
we assume that one motor representation cannot be about another motor representation.
But there is a parallel of sorts. 
A function of the second condition, (2), is to ensure meshing of subplans. 
Each agent's having a motor representation of the outcome to which all their actions are together directed does ensure meshing of subplans.
What ensures this meshing is not the fact that each agent represents the other's plans {as the other's plans}.
Rather what ensures meshing of subplans is this:
Each agent plans all of the agents' actions, and the agents rely on planning strategies that are sufficiently similar to ensure meshing subplans.


The third condition, (3), concerns common knowledge.
Why is this condition needed?
Bratman himself says little.%
\footnote{
See \citet[p.\ 117]{Bratman:1993je}:
`it seems reasonable to suppose that in shared intention the fact that each has the relevant attitudes is itself out in the open, is public.' 
In other words, common knowledge is needed because it is.
}
One possible justification for supposing that shared intention involves common knowledge concerns a normative link between intention and reasons.
In acting on an intention, there should be reasons for which the  agent acts.
And, arguably, a consideration can only be among the reasons for which an agent acts if she knows that consideration (or at least is in a position to know it).
So the need for common knowledge may arise from the need to explain how reasons for which an agent acts could include facts about others' intentions.
This need does not arise in the case of social motor representation (at least not in the same way).
For, arguably, where actions involve motor representations, it is not true that there should be reasons for which the agent acts.
(Of course there are reasons which explain why motor actions happen; but these need not be reasons for which agents act.)
So motor joint action does not require that one agent's motor representations provide reasons for which another agent acts.
Instead, what is required is this.
There should be a good chance---good relative to the potential costs and benefits of attempting this particular joint action now---that social motor representation will provide the necessary coordination.
Of course this could be guaranteed by common knowledge. 
But common knowledge is not required.
Alternatively it can be ensured by common planning processes and a common {background} of dispositions, habits and expectations.%
\footnote{
Another possible line of justification for the claim that common knowledge is involved in shared intention might start from a natural generalisation of Davidson's claim that
`[a]ction does require %that what the agent does is intentional under some description, and this in turn requires 
...\ that what the agent does is known to him under some description' \citep[p.\ 50]{Davidson:1971fz}. 
%If we accepted this claim, then it seems we might have to conclude that not all joint actions are actions. 
%While this may appear to be an obstacle to accepting our view, there is an independent reason for thinking that not all joint actions are actions.
%For if we also follow Davidson in accepting that all actions are bodily movements (or Hornsby in accepting that all actions are tryings), it turns out that few paradigm cases of joint action are actions.
%For instance, we might paint a house together or make a hollandaise sauce together without there being any bodily movements of which we are both agents.
} 
%common background is for reciprocity of motor representation and similar willingness to engage in joint action.

If, as we have just argued, social motor representations play a role analogous to the structure of intentions and knowledge which Bratman identifies as sufficient for shared intention, then this is a  (non-decisive) reason to think that motor representation is also needed in characterising shared agency.

%
%\section{The Contrast Cases}
%We have been arguing that social motor representation can bind multiple agents' activities together and link them to an outcome, and this in much the way that shared intention does  so.
%This is a reason, 
%not decisive but perhaps sufficient in the absence of strong contrary reasons, 
%for supposing that social motor representation plays a role in explaining which events are joint actions.
%Perhaps some events are joint actions in virtue of being appropriately related not to shared intention but to social motor representation.
%
%As mentioned at the start, our aim is to extend and generalise existing theories, not to replace them. 
%But at this point our answer to the question of which events are joint actions seems to involve a disjunction.
%This is both puzzling and inelegant. 
%(Unless, of course, social motor representation can be regarded as a special case of shared intention.  
%We explain why we reject this possibility below.)
%In this section we explain how an account of joint action in terms of shared intention can be generalised to accommodate the possibility that social motor representation plays a role in explaining which events are joint actions.
%
%What guides and constrains theorising about shared intention? 
%While there is little agreement on what shared intention is or what it is for, intuitions about it are sometimes grounded in contrast cases.
%Contrast cases are pairs of events which are similar in terms of the behaviour and coordination they involve but where one is a joint action while the other is not.  
%Thus \citet{gilbert_walking_1990} contrasts two people walking together with two people individually walking side by side.  
%The two pairs' movements may be the same and similarly coordinated (to avoid collision), but walking together is a joint action whereas merely walking side by side is not. 
%Relatedly,  \citet{Searle:1990em}  contrasts a case in which several park visitors simultaneously run to a central shelter in order to perform a dance with another case in which the park visitors run to the central shelter in order to escape a storm.  The first is a case of joint action, the second is not; but the same movements occur in both.  
%These sorts of contrast case invite the question, 
%How do joint actions differ from individual but parallel actions? 
%Gilbert’s example shows that the difference can’t just be a matter of coordination, because people who are merely walking alongside each other also need to coordinate their actions in order to avoid colliding.  
%And Searle’s example shows that the difference between joint action and parallel individual action can’t just be that the actions have a common effect because merely parallel actions can have common effects too. 
%
%How might the contrast cases be used to guide and constrain theorising about shared intention?
%The idea that they serve this function seems to rest on a premise:
%It is possible to distinguish systematically between the contrast cases by appeal to shared intention and only by appeal to shared intention.
%We shall show that there is another, more general way to distinguish contrast cases.
%
%Take Searle's example of several people running to the shelter.  
%In the joint case, there is a single outcome to which each person's actions is individually directed, namely their collective arrival at the shelter.  
%In the contrasting individual case, where park visitors individually run to the shelter to escape a storm, there is no single outcome to which each of their actions is individually directed.  
%Instead each visitor's actions are directed to that visitor's own arrival at the shelter.  
%Similarly, turning to Gilbert's example, when two people walk together, there is a single outcome to which both of their actions are individually directed (their collective arrival at a corner, say); whereas when two people merely walk side by side there is no single outcome to which each agent's actions are directed.  
%In general, where two or more agent's actions constitute a joint action, there is an outcome to which each agent's actions are individually directed such that it is possible for them all to succeed relative to this outcome.%
%\footnote{
%Some readers may be sceptical of this claim.
%But note that it is a consequence of the view we are opposing, the view that all joint actions involve shared intention.
%Or, rather, it is a consequence of this view given a further premise which is a consequence of any standard account of shared intention: Where several agents act on a shared intention that they J, 
%each will perform an action directed to J and it is possible for them all to succeed in J-ing.
%} 
%Since this is not true in the contrasting cases of non-joint action, it allows us to distinguish systematically between joint actions and their non-joint but behaviourally and coordinatively indistinguishable counterparts.  
%
%For concision let us stipulate that an outcome is a \emph{distributive goal} of two or more agents' actions just if two conditions are met.
%First, this outcome is a goal to which each agent's actions are individually directed.
%Second, each agent's actions are related to the goal in such a way that it is possible for all the agents (not just any agent, all of them together) to succeed relative to this goal.
%In these terms, our claim is that joint actions involve distributive goals and that this distinguishes joint actions from non-joint actions in the standard contrast cases.  
%Since social motor representation and shared intention are each sufficient for the existence of a distributive goal, we can conclude that systematically distinguishing between standardly considered contrast cases does not require shared intention.
% 
%A natural response to this argument would be to suggest that the standard contrast cases are insufficient. 
%For in some cases two agents' actions can have a distributive goal although arguably there is no joint action. 
%Nora and Olive killed Fred.  
%Each fired a shot.
%Each intended that her shooting ground or partially ground Fred's death.%
%\footnote{
%Events $D_1$, ...\ $D_n$ \emph{ground} $E$, if: $D_1$, ...\ $D_n$ and $E$ occur; 
%$D_1$, ...\ $D_n$ are each part of $E$; and 
%every event that is 
%	a part of $E$
%	but does not overlap $D_1$, ...\ $D_n$ 
%is caused by some or all of $D_1$, ...\ $D_n$.
%(This is a refinement and generalisation of a notion due to \citet{pietroski_actions_1998}.)
%Event $D$ \emph{partially grounds} event $E$ if $D$ alone does not ground $E$ but there are events including $D$ which do ground $E$.
%} 
%%
%As it turned out, both intentions were fulfilled.
%Neither shot was individually fatal but together they were deadly.
%An ambulance arrived on the scene almost at once but Fred didn't make it to the hospital.
%Now Nora and Olive's actions have a distributive goal.
%After all, 
%	each agent's actions are individually directed to Fred's death
%	and  
%	it is consistent with the stipulations made about this scenario that these goal relations are compatible in the sense that both agents could succeed together. 
%But is their killing of Fred a joint action?
%Imagine that Nora and Olive had no knowledge of each other, nor of each other's actions, and that their efforts were entirely uncoordinated.
%We might even suppose that Nora and Olive are so antagonistic to each other that they would, if either knew the other's location, turn their guns on each other.
%Their actions nevertheless have a distributive goal.
%But given these further suppositions it is likely to seem counterintuitive to suppose that their killing Fred was a joint action. 
%So although we can distinguish the standardly considered contrast cases just by appeal to the notion of a distributive goal, we should not assume that this notion captures the intended contrast between joint and merely parallel actions.
%
%***HERE refine: collective goals
%
%
%
%\section{Social motor representation: unlike shared intention}
%
%*function (not to coordinate planning in Bratman's sense)
%
%*format (based on earlier paper ...)
%


\section{Are social motor representations shared intentions?}

We have been arguing that an account of shared agency cannot appeal to shared intention only  but must also appeal to  social motor representation (and perhaps to other ingredients besides).
Our argument rests on the premise that social motor representations are not shared intentions.
But since we have just bee pointing to broad similarities between shared intention and social motor representation in that both play a role in coordinating agents’ actions by virtue of representing outcomes,
it may be tempting to suppose that some social motor representations are  shared intentions.

This issue might easily seem narrowly conceptual or terminological.  
At the end of the day it doesn’t much matter if we want to call some motor representations ‘shared intentions’.  
After all, as already noted, on some accounts shared intentions are neither shared nor intentions so we would hardly be doing more violence to the term than is already being done.  

However exactly one decides to use the term `shared intention',  at least three substantive issues remain. 
The first concerns conceptual demands.
Whereas having a shared intention arguably demands an ability to represent others' intentions \citep[pp.\ *][]{Butterfill:2011fk},
social motor representation imposes no such demands.
The second concerns planning.
Whereas shared intentions are elements in long-term plans and function in part to enable agents to coordinate their plans, social motor representation is incapable of playing this role.
A third, and related substantive issue is that social motor representations are structures of representations with a non-propositional format and so cannot be inferentially integrated with ordinary intentions and knowledge \citep[pp.\ *][]{butterfill:2012_intention}, whereas shared intentions can.

While prefer (for narrowly terminological reasons) to state our claim by saying that explaining shared agency requires ingredients other than shared intention,
the claim could alternatively be formulated by saying that  explaining shared agency requires importantly different (in the ways described) varieties of shared intention.




\section{Conclusion}
We have been considering how to provide an account of shared agency that might contribute to investigating a tangle philosophical and scientific questions.

Whereas some have claimed that shared agency can be fully explained in terms of a notion of shared intention, we have argued that some events are joint actions by virtue of being appropriately related to a structure of motor representations we call social motor representation.
We don’t mean to suggest that all joint actions involve social motor representation.
The view we are aiming to establish is rather this: Some events are joint actions in virtue of being appropriately related to social motor representations which bind their components together and ensure that there is a single outcome to which these components are collectively directed.
This is why 
%Recap: the question was: Does social motor representation  play a role in explaining what joint is?
%I have just been arguing for a positive answer.
%My thesis is this:
%\textbf{Reciprocal agent-neutral motor representations coordinate multiple agents’ actions around an outcome in part by virtue of representing that outcome.}
%That is, reciprocal social motor representations can ground the purposiveness of joint action.
%This is why I think that 
understanding shared agency requires understanding not only shared intention but also
 the coordinating role of social motor representation.

None of this is to deny that shared intention is among the ingredients needed to characterise shared agency.
Indeed, it may be that the notion of social motor representation has a role to play in explaining what shared intention is.
In constructing realisers of shared intention from ordinary individual intention, we need intentions \emph{that we $J$}.
As has been much discussed \citep[e.g.][]{petersson_collectivity_2007}, the contents of these intentions cannot all refer to actions involving shared intentions.
For this reason Michael Bratman suggests that things we intend are  cooperatively neutral activities. %*ref
It is then necessary to add further intentions in order to transform cooperatively neutral activities into joint actions.
But it also seems possible that in some cases, what we intend when we intend that we $J$ is not a cooperatively neutral activity but instead a joint action of the sort which involves social motor representation.

So perhaps harmony between shared intention and social motor representation is sometimes achieved in this way: what we intend when we share an intention is the sort of joint action that involves social motor representation.

In conclusion, two things.
First, some events are joint actions in virtue of being appropriately related to social motor representations which bind their components together and ensure that there is a single outcome to which these components are collectively directed.
Second, and much more tentatively, in some cases social motor representation may be among the ingredients that realise a shared intention.


\bibliography{$HOME/endnote/phd_biblio}

\end{document}