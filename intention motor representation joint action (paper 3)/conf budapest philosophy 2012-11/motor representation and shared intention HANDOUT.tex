%!TEX TS-program = xelatex
%!TEX encoding = UTF-8 Unicode

\documentclass[11pt]{extarticle}
% extarticle is like article but can handle 8pt, 9pt, 10pt, 11pt, 12pt, 14pt, 17pt, and 20pt text

\def \ititle {Shared Agency and Motor Representation}
\def \isubtitle {}
\def \iauthor {Stephen A. Butterfill}
\def \iemail{s.butterfill@warwick.ac.uk}
\date{}

%for strikethrough
\usepackage[normalem]{ulem}

\input{$HOME/Documents/submissions/preamble_steve_handout}

\bibpunct{}{}{,}{s}{}{,}  %use superscript TICS style bib
%remove hanging indent for TICS style bib
%TODO doesnt work
\setlength{\bibhang}{0em}
%\setlength{\bibsep}{0.5em}


%itemize bullet should be dash
\renewcommand{\labelitemi}{$-$}

\begin{document}

\begin{multicols}{3}

\setlength\footnotesep{1em}


\bibliographystyle{newapa} %apalike

%\maketitle
%\tableofcontents



\

\begin{center}
{\Large
\textbf{Shared Agency and Motor Representation}
}


%<s.butterfill@warwick.ac.uk>
butterfillS@ceu.hu / s.butterfill@warwick.ac.uk

\end{center}
%
%
%\textbf{Abstract}
%On the assumption that social motor representation plays a role in explaining how effective joint action is possible, do we also need motor representation to explain what joint action is?  Philosophers tend to assume that motor representation is only an enabling condition for joint action and of no direct interest to narrowly philosophical theories of joint action and shared intention.  In this talk I shall argue that social motor representation and shared intention  have distinctive roles in explaining the purposiveness of joint action.  This gives rise to a challenge.  On the one hand, effective joint action---imagine two people erecting a tent in a gale together---sometimes requires both shared intentions and social motor representations plus a certain kind of harmony between the two.  On the other hand, recognizing their distinctive roles precludes the existence of direct inferential links between shared intentions and social motor representations.  The challenge is to explain how these two kinds of representation could sometimes harmoniously contribute to effective joint action despite the lack of inferential integration.
%

\section{Introduction}
A representation (motor or not) is \emph{agent-neutral} if its content does not specify any particular agent or agents.

%***CHECK DEFINITIONS***

Two or more motor representations are \emph{reciprocal} just if there is a single outcome which each motor representation represents.

\textbf{Premise} 
Reciprocal agent-neutral motor representation enables joint action


\begin{quote}
`the social relation between individuals modulates action simulation ...\ motor activation during action anticipation depends on the ...\ relation between the actor and the observer ...\ Simulation of another person’s action, as reflected in the activation of motor cortices, gets stronger the more the other is perceived as an interaction partner.’  \citep{kourtis:2010_favoritism}
\end{quote}

\textbf{Question} Does reciprocal agent-neutral motor representation also play a role in explaining what joint action is?  

\textbf{Challenge}  How could social motor representation and shared intention harmoniously contribute to joint action?



\section{The possibility of purposive joint action}
What is the relation between a purposive joint action and the outcome or outcomes to which it is directed?

Reciprocal agent-neutral motor representations can
(1) involve a representation, on the part of each agent, of an outcome; 
(2) coordinate the several agents’ activities;
and 
(3) coordinate the several agents’ activities in such a that would normally facilitate the occurrence of the represented outcome.

For you and I to have a shared intention that we J it is sufficient that: `(1)(a) I intend that we J and (b) you intend that we J; (2) I intend that we J in accordance with and because of la, lb, and meshing subplans of la and lb; you intend that we J in accordance with and because of la, lb, and meshing subplans of la and lb; (3) 1 and 2 are common knowledge between us.'\citep%[View 4]
{Bratman:1993je}




\section{An objection}
‘the key property of joint action lies in its internal component [...] in the participants’ having a “collective” or “shared” intention.’ \citep[pp.\ 444-5]{alonso_shared_2009}


But could some reciprocal agent-neutral motor representations be shared intentions?  No ...
\begin{enumerate}
\item Only representations with a common format can be inferentially integrated.

\item Any two intentions can be inferentially integrated in practical reasoning.

\item My intention that I visit Paris on Friday is a propositional attitude.

\item All intentions are propositional attitudes.

\item No motor representations are propositional attitudes.

\item No motor representations are intentions.
\end{enumerate}



\section{The Interface Problem}
Two  outcomes, A and B, \emph{match} in a particular context just if, in that context, either the occurrence of A would normally constitute or cause, at least partially, the occurrence of B or vice versa. 

Some joint actions involve both shared  intention and reciprocal agent-neutral motor representation.

How are non-accidental matches between the outcomes specified by shared intentions and by reciprocal agent-neutral motor representations possible?

`motor imagery could play a crucial role in bridging the gap'\citep{pacherie:2000_content}


\footnotesize 
\bibliography{$HOME/endnote/phd_biblio}

\end{multicols}

\end{document}