%!TEX TS-program = xelatex
%!TEX encoding = UTF-8 Unicode

\documentclass[12pt]{extarticle}
% extarticle is like article but can handle 8pt, 9pt, 10pt, 11pt, 12pt, 14pt, 17pt, and 20pt text

\def \ititle {Shared Agency and Motor Representation}
\def \isubtitle {}
\def \iauthor {Stephen A. Butterfill}
\def \iemail{s.butterfill@warwick.ac.uk}
\date{}

%for strikethrough
\usepackage[normalem]{ulem}

\input{$HOME/Documents/submissions/preamble_steve_handout}

\bibpunct{}{}{,}{s}{}{,}  %use superscript TICS style bib
%remove hanging indent for TICS style bib
%TODO doesnt work
\setlength{\bibhang}{0em}
%\setlength{\bibsep}{0.5em}


%itemize bullet should be dash
\renewcommand{\labelitemi}{$-$}

\begin{document}

\begin{multicols}{3}

\setlength\footnotesep{1em}


\bibliographystyle{newapa} %apalike

%\maketitle
%\tableofcontents



\

\begin{center}
{\Large
\textbf{Shared Agency \\and Motor Representation}
}


%<s.butterfill@warwick.ac.uk>
butterfillS@ceu.hu / s.butterfill@warwick.ac.uk

\end{center}


\section{Question}
\emph{Shared Agency}.  Sisters exercise shared agency when they cycle to school together; in contrast,  strangers who happen to be cycling the same route are exercising parallel but merely individual agency.\citep{gilbert_walking_1990}

When members of a flash mob in the Central Cafe respond to a pre-arranged cue by noisily opening their newspapers, they exercise shared agency. 
But when others happen to noisily open their newspapers in response to the same cue, they do not.\citep{Searle:1990em}


A \emph{joint action} is an exercise of shared agency (in contrast to an \emph{individual action}).

What is the relation between a purposive joint action and the goal or goals to which it is directed?



\section{The Standard View}
‘the key property of joint action lies in its internal component [...] in the participants’ having a “collective” or “shared” intention.’ \citep{alonso_shared_2009} %[pp.\ 444-5]

For you and I to have a shared intention that we J it is sufficient that: `(1)(a) I intend that we J and (b) you intend that we J; (2) I intend that we J in accordance with and because of la, lb, and meshing subplans of la and lb; you intend that we J in accordance with and because of la, lb, and meshing subplans of la and lb; (3) 1 and 2 are common knowledge between us.'\citep%[View 4]
{Bratman:1993je}

`each agent does not just intend that the group perform the […] joint action. Rather, each agent intends as well that the group perform this joint action in accordance with subplans (of the intentions in favor of the joint action) that mesh'\citep%[p.\ 332]
{Bratman:1992mi}


\section{Thesis}
In some cases 
	it is not a shared intention 
	but a special structure of motor representation,
	a `shared motor representation',
	in virtue of which a joint action is related to its goal.


\section{Shared Motor Representation}

A \emph{goal} is an outcome to which actions are, or might be, directed.  (Contrast a \emph{goal-state}, an intention or other state of an agent linking an action to a goal to which it is directed.)

An outcome is a \emph{distributive goal} of two or more actions  just if 
(a) each action is individually directed to this outcome; and 
(b) it is possible that: all actions succeed relative to this outcome.

An outcome is a \emph{collective goal} of two or more actions just if
(a) this outcome is a distributive goal of the actions;
(b) the actions are coordinated; and 
(c) coordination of this type would normally  facilitate occurrences of outcomes of this type



A representation or plan is \emph{agent-neutral} if its content does not specify any particular agent or agents; a planning process is agent-neutral if it involves only agent-neutral representations.

Events $D_1$, ...\ $D_n$ \emph{ground} $E$, if: $D_1$, ...\ $D_n$ and $E$ occur; 
$D_1$, ...\ $D_n$ are each (perhaps improper) parts of $E$; and 
every event that is a proper part of $E$ but does not overlap  $D_1$, ...\ $D_n$ is caused by some or all of $D_1$, ...\ $D_n$.

For an individual to be \emph{among the agents of an event} is for there to be actions $a_1$, ...\ $a_n$ which ground this event where the individual is an agent of some (one or more) of these actions.

We have a \emph{shared motor representation} of an outcome just if 
\begin{enumerate}[label=\emph{\alph*}),itemsep=0pt,topsep=0pt]
\item we each have a motor representation of this outcome; 
\item we are each disposed to inhibit some but not all of the planning or actions resulting from (a);
\item  we each expect that if the outcome occurs, we will all be among the agents of its occurrence; and
\item the truth of (a) and (b) depends on the truth of (c).
\end{enumerate}

%Two or more representations (motor or not) are \emph{reciprocal} just if there is a single outcome which each represents.
%



\section{Evidence that Shared Motor %Rep\textsuperscript{\underline{n}} 
Representation Exists}

In joint action, motor planning can occur for another's actions,\citep{kourtis:2012_predictive} and can inform planning for one's own actions.\citep{vesper:2012_jumping}  %*submitted imagining acting paper

In joint action, it is sometimes necessary to inhibit planning or performing another's action.\citep{sebanz:2006_twin_peaks} 
Whether this is necessary depends on one's beliefs about co-actors' agency.\citep{tsai:2008_action}

In some joint actions, the agents have a single representation of the whole action (not only separate representations of each agent's part).\citep{tsai:2011_groop_effect}


% Vesper says forthcoming EEG paper using piano playing paradigm on agent-neutral identification of error: one brain wave signals whether there is an error, and a different brain wave signals whose error it is (also tells you whether the overall harmonics are affected)





\section{The Interface Problem}
Two  outcomes, A and B, \emph{match} in a particular context just if, in that context, either the occurrence of A would normally constitute or cause, at least partially, the occurrence of B or vice versa. 

A shared motor representation is in \emph{harmony} with a shared intention if they concern matching outcomes.

Some joint actions involve both shared intention and shared motor representation.

How is non-accidental harmony between shared intentions and shared motor representations?

Proposal: `motor imagery could play a crucial role in bridging the gap'\citep{pacherie:2000_content}


\footnotesize 
\bibliography{$HOME/endnote/phd_biblio}

\end{multicols}

\end{document}