 %!TEX TS-program = xelatex
%!TEX encoding = UTF-8 Unicode

\def \papersize {a4paper}

\documentclass[12pt,\papersize]{extarticle}
% extarticle is like article but can handle 8pt, 9pt, 10pt, 11pt, 12pt, 14pt, 17pt, and 20pt text

\def \ititle {Motor Representation \\ and Shared Intention}
\def \isubtitle {}
\def \iauthor {
\textit{Collective Intentionality}, Manchester, August 2012}
\def \iemail{s.butterfill@warwick.ac.uk}
%\date{}

%!TEX TS-program = xelatex
%!TEX encoding = UTF-8 Unicode

\title{\ititle\\\isubtitle}
\author{\iauthor\\<{\iemail}>}

\usepackage[\papersize]{geometry} % see geometry.pdf
\geometry{twoside=false}
\geometry{headsep=2em} %keep running header away from text
\geometry{footskip=1cm} %keep page numbers away from text
\geometry{top=3cm} %increase to 3.5 if use header
\geometry{left=4cm} %increase to 3.5 if use header
\geometry{right=4cm} %increase to 3.5 if use header
\geometry{textheight=22cm}

%non-xelatex
%\usepackage[T1]{fontenc}
%\usepackage{tgpagella}

%for underline
\usepackage[normalem]{ulem}

%get the font here:
% http://scripts.sil.org/CharisSILfont

\usepackage{fontspec,xunicode}
%nb do not explicitly use package xltxtra because this introduces bugs with footnote superscripting  -- perhaps because fontspec is supposed to include it anyway.
%UPDATE:  "You need to use the no-sscript option in xltxtra: \usepackage[no-sscript]{xltxtra}, this is explained in the documentation of xltxtra.  The issue is that Sabon does not contain true superscript glyphs for every character and the no-sscript option will instead use scaled regular glyphs, which is typographically inferior, but there is no other option available when using Sabon." --- http://groups.google.com/group/comp.text.tex/browse_thread/thread/19de95be2daacade
\defaultfontfeatures{Mapping=tex-text}
%\setromanfont[Mapping=tex-text]{Charis SIL} %i.e. palatino
%\setromanfont[Mapping=tex-text]{Sabon LT Std} 
%\setromanfont[Mapping=tex-text]{Dante MT Std} 
%\setromanfont[Mapping=tex-text,Ligatures={Common}]{Hoefler Text} %comes with osx
\setromanfont[Mapping=tex-text]{Linux Libertine O} 
\setsansfont[Mapping=tex-text]{Linux Biolinum O} 
\setmonofont[Scale=MatchLowercase]{Andale Mono}


%hyperlinks and pdf metadata
%TODO avoid duplication of title & author
\usepackage{hyperref}
\hypersetup{pdfborder={0 0 0}}
\hypersetup{pdfauthor={\iauthor}}
\hypersetup{pdftitle={\ititle\isubtitle}}


%handles references to labels (e.g. sections) nicely
\usepackage{varioref}

%line spacing
\usepackage{setspace}
%\onehalfspacing
%\doublespacing
\singlespacing

\usepackage{natbib}
%\usepackage[longnamesfirst]{natbib}
\setcitestyle{aysep={}}  %philosophy style: no comma between author & year

%enable notes in right margin, defaults to ugly orange boxes TODO fix
%\usepackage[textwidth=5cm]{todonotes}

%for comments
\usepackage{verbatim}

%footnotes
\usepackage[hang]{footmisc}
\setlength{\footnotemargin}{1em}
\setlength{\footnotesep}{1em}
\footnotesep 2em

%tables
\usepackage{booktabs}
\usepackage{ctable}

%section headings
\usepackage[sf]{titlesec}
%\titlespacing*{\section}{0pt}{*3}{*0.5} %reduce vertical space after header
%large headings:
%\titleformat{\section}{\LARGE\sffamily}{\thesection.}{1em}{} 
\titlelabel{\thetitle.\quad}

%captions
\usepackage[font={small,sf}, margin=0.75cm]{caption}

%lists
\usepackage{enumitem}
\newenvironment{idescription}
{ 	
	% begin code
	\begin{description}[
		labelindent=1.5\parindent,
		leftmargin=2.5\parindent
	]
}
{ 
	%end code
	\end{description}
}


%title
\usepackage{titling}
\pretitle{
	\begin{center}
	\sffamily
	\Huge
} 
\posttitle{
	\par
	\end{center}
	\vskip 0.5em
} 
\preauthor{
	\begin{center}
	\normalsize
	\lineskip 0.5em
	\begin{tabular}[t]{c}
} 
\postauthor{
	\end{tabular}
	\par
	\end{center}
}
\predate{
	\begin{center}
	\normalsize
} 
\postdate{
	\par
	\end{center}
}


%\author{}
%\date{}

%\setromanfont[Mapping=tex-text]{Sabon LT Std} 

\begin{document}

\setlength\footnotesep{1em}

\bibliographystyle{$HOME/Documents/submissions/mynewapa} %apalike

\maketitle
%\tableofcontents
\title{}

\begin{abstract}
\noindent
On the assumption that motor representation plays a role in explaining how effective joint action is possible, do we also need motor representation to explain what joint action is? Philosophers tend to assume that motor representation is only an enabling condition for joint action and of no direct interest to narrowly philosophical theories of joint action and shared intention. In this talk I shall argue that social motor representation and shared intention have distinctive roles in explaining the purposiveness of joint action. This gives rise to a challenge. On the one hand, effective joint action—imagine two people erecting a tent in a gale together—sometimes requires both shared intentions and social motor representations plus a certain kind of harmony between the two. On the other hand, recognizing their distinctive roles precludes the existence of direct inferential links between shared intentions and social motor representations. The challenge is to explain how these two kinds of representation could sometimes harmoniously contribute to effective joint action despite the lack of inferential integration. 
\end{abstract}

\section{Introduction}
Which events are joint actions?
Most philosophers and many (but not all) psychologists appear to assume that this question can be fully answered just in terms of a special kind of intention or structure of intention, knowledge and commitment often called a shared (or `collective') intention.
On this widely held view, for an event to be a joint action is for it to be appropriately related to a shared intention.

[\textit{Short version: One consideration in favour of this view is ... and jump to the second consideration towards the end of  section \ref{text:second_consideration} on page \vpageref{text:second_consideration}.}]

\subsection{What is shared intention?}
Of course this answer to the question about which events are joint actions assumes that we know what shared or collective intention is.
But there are many accounts of it.
Still, I think most people would agree that shared intention stands to joint action approximately as plain vanilla intention stands to ordinary, individual action
Further,
on all or most leading accounts of shared intention, each of the following is a necessary condition:

\begin{description}

\item[awareness of joint-ness] Agents acting on a shared intention know that they are not acting individually; they have `a conception of themselves as contributors to a collective end.'\footnote{
	\citet[p.\ 10]{Kutz:2000si}.  Compare \citet[p.\ 361]{Roth:2004ki}: `each participant ... can answer the question of what he is doing or will be doing by saying for example ``We are walking together'' or ``We will/intend to walk together.''' 
Relatedly, \citet[p. 56]{miller_social_2001} requires that each agent believes her actions are interdependent with the other agent's.
}

\item[awareness of others' agency]  When agents act on a shared intention, each is aware of at least one of the others as an intentional agent.\footnote{
	Compare \citet[p.\ 333]{Bratman:1992mi}: `Cooperation ... is cooperation between intentional agents each of whom sees and treats the other as such'.  See also \citet[p.\ 105]{Searle:1990em}: `The biologically primitive sense of the other person as a candidate for shared intentionality is a necessary condition of all collective behavior' 
}

\item[awareness of others' states or commitments] When two agents share an intention that they F, each is aware of, or has individuating beliefs about, some of the other's intentions, beliefs or commitments concerning F.\footnote{
This condition is necessary for shared intention even on what \citet[p.\ 40]{tuomela_collective_2000} calls `the weakest kind of collective intention'.  But it may not be necessary if, as \citet{Gold:2007zd} suggest, shared intentions are constitutively intentions formed by a certain kind of reasoning.
% "if the distinctive feature of collective intentions is to be found in the reasoning by which they were formed, then an analysis that focuses on the intentions themselves will miss the feature that makes collective intentions collective. " 
}

\end{description}
%
I think we can go even further and say that, 
on most accounts,
where joint action involves shared intention, the agents act in part \emph{because} of their awareness of joint-ness, of others' agency and of others' states or commitments.  


What follows assumes that where one or more of these three conditions is not met, there is no shared intention. 


\subsection{Why shared intention?}
Why might anyone hold that for an event to be a joint action is for it to be appropriately related to a shared intention?
Two considerations favour this view.

First, the view might be motivated by appeal to some well-known contrast cases.
Contrast cases are pairs of events which are similar in terms of the behaviour and coordination they involve but where one is a joint action while the other is not.  
Thus \citet{gilbert_walking_1990} contrasts two people walking together with two people individually walking side by side.  
The two pairs' movements may be the same and similarly coordinated (to avoid collision), but walking together is a joint action whereas merely walking side by side is not. 
Relatedly,  \citet{Searle:1990em}  contrasts a case in which several park visitors simultaneously run to a central shelter in order to perform a dance with another case in which the park visitors run to the central shelter in order to escape a storm.  The first is a case of joint action, the second is not; but the same movements occur in both.  
These sorts of contrast case invite the question, 
How do joint actions differ from individual but parallel actions? 
Gilbert’s example shows that the difference can’t just be a matter of coordination, because people who are merely walking alongside each other also need to coordinate their actions in order to avoid colliding.  
And Searle’s example shows that the difference between joint action and parallel individual action can’t just be that the actions have a common effect because merely parallel actions can have common effects too. 
If someone thought that it is possible to distinguish systematically between the contrast cases by appeal to shared intention (and perhaps only by appeal to shared intention), this could motivate holding that for an event to be a joint action is for it to be appropriately related to a shared intention.

{\label{text:second_consideration} A second consideration} in favour of this view is a possible parallel between joint action and ordinary, individual action.
If you think that events are actions in virtue of being appropriately related to intentions, 
and if you think that shared intention stands to joint action as plain vanilla intention stands to ordinary, individual action,
then it seems reasonable to suppose that for an event to be a joint action is for it to be appropriately related to a shared intention (as \citealp{Pacherie:2012fk} suggests).



\subsection{My view}
On the second consideration, 
we take for granted that there is a parallel between joint action and ordinary, individual action.
But is the standard view about the relation between action and intention correct? 

Not everyone accepts the standard view.
Some philosophers allow that there could be agents whose actions are purposive---and, in some cases, even intentional---although  the agents have no intentions at all.
For instance, Michael Bratman describes a creature who `acts on the basis of its beliefs and considered desires' only (not intentions) as doing things `intentionally' \citet[p.\ 251]{bratman:2000_valuing}.%
%\footnote{
%For instance, \citet[p.\ 251]{bratman:2000_valuing} % = \citep[p.\ 51]{Bratman:2007bd}
%describes an agent called `Creature 2' who `acts on the basis of its beliefs and considered desires' only (not intentions) as doing things `intentionally'.
%}

Others reject the standard view without asserting there can be action without intention.
Instead they claim that ingredients other than intention are needed to say which events are actions.
In particular, some claim that motor representation and intention are both needed in explaining which events are ordinary, individual actions \citep{pacherie:2000_content,butterfill:2012_intention}.
If any such view is correct, it is impossible to fully explain what it is for an event to be an action by appeal to intention alone.

Our aim in this talk is to defend a parallel view about joint action. 
%(Parallel to the latter view, that is: not saying that creatures without any shared intentions are capable of joint actions.)
We shall consider the possibility that fully explaining which events are joint actions requires appeal to structures of motor representation and not only to shared intention.
%***what's the argument for this second claim!?

Please bear in mind that we are not trying to get rid of, or replace, the notion of shared intention.
%I take for granted that a notion of shared intention is needed for explaining which events are joint actions. 
%My concern is only with the possibility that additional ingredients are needed as well.
In fact recognising that motor representation is needed to say which events are joint actions may make it easier to understand what shared intention is.


\section{Intention and Motor Representation}
[\emph{*Probably skip all of this section in the talk}]

So that you can see where this is going, let me briefly explain what stands behind the view that motor representation is needed in addition to intention in saying which events are actions.

A basic question about ordinary, individual action is:
What is the relation between a purposive action and the outcome or outcomes to which it is directed?
Purposive actions typically have many actual outcomes.
Grabbing little Isabel by the hands I swing her around, causing her to laugh and, simultaneously, breaking a vase.
Either or both of these might be outcomes to which my action is directed.
Note also that some or all of the outcomes to which my action was directed might not be among its actual outcomes; after all, actions can fail.
So among all the actual and possible outcomes of my action, one or some are singled out as specially related to this action.
The question is what singles out the outcome or outcomes, actual or merely possible, to which a particular purposive is directed.

This question is closely related to a second.
Ordinary purposive actions are sometimes composed of more than one motor action.  My swinging Isabel around includes my reaching for her wrists, grasping them and then spinning us around.  
But  my action doesn’t include other things which I might be doing simultaneously, like refusing a cup of tea with my eyes or  trying to determine whether that smell is coming from Isabel’s sister Hannah’s nappy.
The second question, then, is this: For a particular action directed to a specified goal, what determines which activities%
\footnote{
As we use the term `activities',
all actions are activities 
and so are  things like reaching and grasping.
This should make it clear that our position does not depend on whether or not reachings and the like are  actions.
} 
comprise the purposive action and which do not?

The standard answer to both questions involves intention.
An intention represents an outcome, coordinates the one or several activities which comprise the action, and coordinates these activities in a way that would normally facilitate the outcome's occurrence.
What binds component activities together into larger purposive actions?  
It is the fact that these actions are all consequences of plans involving a single intention (and are all appropriately related to those plans).
What singles out an actual or possible outcome as one to which the component activities are collectively directed?  
It is the fact that this outcome is represented by the intention.
So the intention is what binds component actions together into purposive actions and links the action taken as a whole to the outcomes to which they are directed.

Motor representations are relevantly similar to intentions,
as some have recently argued (\citealp[pp.\ 189-90]{pacherie:2008_action}; \citealp{butterfill:2012_intention}).
Of course motor representations differ from intentions in some important ways (as these authors note).
But they are similar in the respects that matter for explaining the purposiveness of action.
For, like intentions, some motor representations represent outcomes and not merely kinematic features of action.
Like intentions, some motor representations play a role in coordinating multiple  component activities by virtue of their role as elements in hierarchically structured plans.
And, like intentions, some motor representations coordinate these activities in a way that would normally facilitate the outcome’s occurrence.
So anyone who accepts the standard story about purposive action and intention
should also accept that a similar story about purposive action and motor representation.
Given that the two basic questions about ordinary, individual purposive action can also be answered by appeal to intention,
they can also be answering by appeal to motor representation.
This motivates the following view.
Not all purposive actions are bound together and linked to outcomes by intentions. 
In some cases what binds together purposive actions and links them to outcomes are motor representations.
%In some cases, the purposiveness of an action is grounded in a motor representation of an outcome; in other cases it is grounded in an intention.
And of course in many cases it may be that both intention and motor representation are involved.

This is one reason for supposing that motor representation is not merely an enabling condition for ordinary, individual action but also plays a role in explaining what action is (just as intention does).
Now let’s return to joint action.



\section{Social motor representation}
We start from the premise that some joint actions are facilitated by reciprocal, agent-neutral motor representations of outcomes whose obtaining would normally involve action on the part of each agent.

This needs unpacking. 
A \textit{motor} representation is the sort of representation that enables us to reach for, grasp and transfer objects in a coordinated and fluid way.
We follow several psychologists and philosophers in supposing that motor representations feature in planning and monitoring action \citep[e.g.]{wolpert:1995internal, miall:1996_forward}.
One consequence is that motor representations are not concerned with merely kinematic or dynamic features of actions only.
Rather, some motor representations represent outcomes, such as the movement of a target object from one place to another.

A representation (motor or not) is \emph{agent-neutral} if its content does not specify any agent or agents.%
\footnote{
Our use of the term `agent-netural' to describe motor representations bears no relation to the use of the same term  to describe reasons \citep[on the latter, see][]{Parfit:1984fk}.
}
To illustrate, agent neutral representations are sometimes found at the early stages of planning.
Imagine that you and some friends are tasked with preparing a holiday.  
You might first write down a plan of action without specifying who will act; the plan simply describes what is to be done.
The plan  will eventually be implemented by you and your friends
 but this is not written in plan itself  and so it is agent-neutral.
Of course the fact that this plan is your collective plan may be represented elsewhere; this fact may also be implicit in  the plan's being stapled to the door of your communal kitchen.
The agent-neutrality of a representation does not require that the agents are nowhere specified, only that they are not specified in the content of the representation.%
\footnote{
Strictly the following argument does not hinge on the agent-neutrality of representations.
It is sufficient for our purposes that there are reciprocal motor representations  of outcomes whose obtaining would normally involve action on the part of each of the reciprocating agents.
In principle such representations could have contents which specify other agents or multiple agents. 
We focus on agent-neutral representations to simplify exposition.
It may be important that our view is consistent with the possibility that reciprocal motor representations are agent-neutral. 
}

Two or more agents have \emph{reciprocal} motor representations  just if there is a single outcome and each agent has a motor representation of that outcome. 
It is hardly controversial that reciprocal motor representations exist, for their existence is suggested by by a large body of research on motor cognition in action observation.
What is much more controversial is that reciprocal motor representation occurs in joint action.
Some of the research that Gunether Knoblich presented supports the claim that there are reciprocal motor representations in joint action, and he suggests that these enable joint action.%
\footnote{
See \citet{kourtis:2012_predictive}: `the partner’s expected action is simulated at the motor level, which probably facilitates effective performance of the joint action.'
\citet{kourtis:2010_favoritism} show that 
reciprocal motor representation is more likely to occur in joint action than is mere observation. 
See also \citet{Knoblich:2003nf}.
 }

I want to go a tiny step further and suggest that in joint action there are sometimes reciprocal, agent-neutral representations of outcomes whose obtaining would normally involve action on the part of all of the agents.
So when we act together, some of my motor representations may concern outcomes that are partly but not entirely to be realised by my actions.
For example, suppose our task is to move an object from A to B, where you pick it up and pass it to me so that I can then place it.
In this case I may represent the movement of the object from A to B and not only the component movements.
I represent a collective outcome of our actions and not just outcomes to which each of our actions are individually directed.
%
%So it is not just that, when we act together, I represent the outcomes to which your actions are directed and the outcomes to which my actions are directed; I may also represent an outcome to which all of our actions are directed.

We shall use the term \emph{social} motor representation as an abbreviation for the reciprocal, agent-neutral motor representation of outcomes whose obtaining would normally require action on the part of each reciprocating agent.
(Of course we cannot assume in advance of argument that such reciprocal, agent-neutral motor representations are social in any interesting sense;
but we are using the term `social' in a non-standard way as an abbreviation, one that reflects our aim.)

What follows is speculative philosophy: we take for granted that sometimes social motor representation facilitates joint action and ask whether this conjecture bears on our question about which events are joint actions.


\section{How could social motor representation facilitate joint action?}
Suppose that social motor representation is present in some joint action contexts.
It doesn't follow, of course, that social motor representation facilitates some joint actions.
But let us consider just the possibility that it might.
How could social motor representation facilitate joint action even in principle?

To answer this question
let us take a step back and consider an individual action.
Suppose an agent moves a mug from one place to another, passing it from her left hand to her right hand half way.
It is a familiar idea that motor planning, like planning generally, involves starting with relatively abstract representations of outcomes and gradually filling in details.
We can capture this by supposing that 
motor representations for planning and monitoring action involve a hierarchical structure of representations.
At the top we might find a relatively abstract representation of an outcome, in this case of the movement of the object from one location to another.
Action-relevant details are progressively filled in by representations at lower stages of the hierarchy. 
Now in the action we are considering there is a need, even for the single agent, to coordinate the exchange between the two hands.
How is this achieved? 
We suppose that part of the answer involves the fact that planning for the movements of each hand is not done entirely independently.
Rather there is a plan for the whole action
and plans for the movements of each hand are components of this larger plan.
It is in part because they are parts of a larger plan that the plan for one hand constrains and is constrained by the plan for the other hand.


How is this relevant to the case of joint action?
In joint action the agents have the same goal, to move the object from one place to another.
They also face a similar coordination problem, requiring a precisely timed swap from one hand to another.
Now suppose, 
inspired by Koutis et al’s and others' findings,
that the same planning is involved in the individual case (where one agent performs the whole action) and in the joint action case (where the action is distributed between two agents).
The planning is the same almost up to the actual muscle contractions.

How could this be helpful?
Suppose the agents' planning processes are similar enough that, for a given context and problem, they will produce approximately the same plans.
Then having each agent plan the whole joint action means that (i) each agent plans the other agent's action,
(ii) each agent's plan for the other agent's action is approximately the same as that agent's plan for her own action,
and 
(iii) each  agent's plans for their own action are constrained by their plans for the other agent's action.

So what enables the two agents' plans to mesh is not that they represent each other's plans but more simply that they plan each other's actions as well as their own actions as if they were each about to do the whole thing themselves.

Each agent is planning (and monitoring) both their actions almost as if a single agent were going to execute the whole action.
And of course this is exactly what we want for small-scale joint action---we want two or more agents to act as one.
This may be why the performance of dyads in joint actions often resembles the performance of individuals tasked with performing the whole action alone \citep{Knoblich:2003nf}.

So what is the difference between the individual and the joint case?  From the point of view of motor representation, the primary difference may be that in joint action there is a need to prevent execution of the parts of the action which are not one’s own.

%Here then is the basic idea I take to be guiding Kourtis and others.
%Coordination is sometimes achieved by having each agent’s motor system plan all of their actions; 
%given some assumptions, this could be a way of making it likely that each will execute their part in the joint action in a way that meshes with the way the other agents execute their parts.


\section{Grounding the purposiveness of joint action}
So far we have only been considering a possible role for social motor representation in facilitating joint action.  
Our primary concern, though, is with what joint action is.
The details of how 
social motor representation enables joint action
 already give us grounds for holding that motor representation has a role to play in explaining which events are joint actions.
To see why,
let’s go back to individual action for a moment again.

[***moved to earlier]

The same two questions we asked about ordinary, individual action also arise for joint action.
%What is the relation between a purposive joint action and the outcome or outcomes to which it is directed?
What singles out the outcome or outcomes to which a purposive joint action is directed?
And what binds together the various activities (of several agents) that make up the joint action?
%A difference in the case of joint action is, of course, that the component activities are not activities of a single agent.

If we appeal to a notion of shared intention,
we can answer these questions about joint action in a way that is superficially similar to way we answered the parallel questions about ordinary, individual action.
A shared intention is what relates purposive joint actions to the outcomes to which they are directed.
For the shared intention 
involves a representation, on the part of each agent, of an outcome,
coordinates the several agents’ activities
and 
coordinates the several agents’ activities in such a way that would normally facilitate the occurrence of the represented outcome.
This is how a shared intention can bind together the activities comprising a joint action and link them to an outcome.
%Shared intention does for joint action roughly what ordinary intention does for ordinary, individual action.

% [individual]	intention : motor representation
% 					::
% [joint]			shared intention : social motor representation

Our earlier discussion of how social motor representation might enable joint action already shows that social motor representation resembles shared intention in this respect.
Return to the example of two agents moving an object in a way that involves passing it between them.
Suppose that their passing involves a social motor representation of the outcome,
%reciprocal agent-neutral motor representations of the outcome, 
which is the movement of the object. 
Then there are motor representations, one for each agent, 
of an outcome to which the joint action is directed.
And these representations coordinate the several agents' activities,
and 
do so in ways that would normally facilitate the occurrence of the  outcome represented.%
\footnote{
This implies that social motor representation and the associated processes underwrite what \citet{Butterfill:2011_wija} calls \textit{collective goals}.
}
So social motor representation can bind together the activities comprising a joint action and link them to an outcome in much the way that shared intention can.

%[*use above?] Because each agent represents the whole movement and plans all of its implementation irrespective of which parts she will actually perform, each agent plans the action in a way that should coordinate with the other agent's plans providing they use similar planning procedures

%[*What I’m saying here, in effect, is that both shared intention and social motor representation can yield a COLLECTIVE GOAL]

What we are suggesting is very simple.
Given the correctness of a standard view about shared intention in joint action, 
and 
given that in ordinary, individual action, motor representations  bind together activities and link them to outcomes,
it is plausible that 
in joint action, several agents' activities can be bound together and linked to an outcome by social motor representation.
That is,
the purposiveness of a joint action can be grounded not only in shared intention, but also in social motor representation.



\section{Social motor representation: like shared intention}
It may be helpful to compare and contrast the notion of social motor representation with a notion of shared intention. 
We shall use Bratman's account of shared intention as it is the best developed. 
Here are Bratman’s collectively sufficient\footnotemark \ conditions for you and I to have a shared intention that we J:
%
\footnotetext{
In \citet{Bratman:1992mi}, the following were offered as jointly sufficient \textit{and individually necessary} conditions; the retreat to sufficient conditions occurs in \citet[][pp.\ 143-4]{Bratman:1999fr} where he notes that `for all that I have said, shared intention might be multiply realizable.'
} 
%
\begin{quote}
\label{quote:bratman_account}
`1. (a) I intend that we J and (b) you intend that we J
 
`2. I intend that we J in accordance with and because of la, lb, and meshing subplans of la and lb; you intend that we J in accordance with and because of la, lb, and meshing subplans of la and lb
 
`3. 1 and 2 are common knowledge between us' \citep[][p.\ View 4]{Bratman:1993je}
\end{quote}
%
Let us take each of these three conditions in turn.

To see a parallel with the first condition, (1), recall two (empirical) claims on which the notion of social motor representation is based.
First, some motor representations represent outcomes.
Second, some motor representations represent the outcomes of actions not all of whose components will be executed by the agent whose motor representation it is.
Given these claims, there is a direct parallel with Bratman's first condition, (1).
Where some agents have either a shared intention or a social motor representation, there is an outcome to which their actions are directed and each agent represents this outcome.
Of course there is also a difference: In the case of social motor representation, the outcome is represented motorically and need not feature in the content of any intention.%
\footnote{
Here and below were are assuming that no motor representations are intentions. 
If this assumption is wrong (as \citealp{pacherie:2008_action} suggests), social motor representation may be even more closely related to shared intention that we suggest here.
}


Concerning the second condition, (2), there is clearly no {direct} parallel. 
Whereas one intention can be about another intention, 
we assume that one motor representation cannot be about another motor representation.
But there is a parallel of sorts. 
A function of the second condition, (2), is to ensure meshing of subplans. 
Each agent's having a motor representation of the outcome to which all their actions are together directed does ensure meshing of subplans.
What ensures this meshing is not the fact that each agent represents the other's plans {as the other's plans}.
Rather what ensures meshing of subplans is this:
Each agent plans all of the agents' actions, and the agents rely on planning strategies that are sufficiently similar to ensure meshing subplans.


The third condition, (3), concerns common knowledge.
Why is this condition needed?
Bratman himself says little.%
\footnote{
See \citet[p.\ 117]{Bratman:1993je}:
`it seems reasonable to suppose that in shared intention the fact that each has the relevant attitudes is itself out in the open, is public.' 
In other words, common knowledge is needed because it is.
}
One possible justification for supposing that shared intention involves common knowledge concerns a normative link between intention and reasons.
In acting on an intention, there should be reasons for which the  agent acts.
And, arguably, a consideration can only be among the reasons for which an agent acts if she knows that consideration (or at least is in a position to know it).
So the need for common knowledge may arise from the need to explain how reasons for which an agent acts could include facts about others' intentions.
This need does not arise in the case of social motor representation (at least not in the same way).
For, arguably, where actions involve motor representations, it is not true that there should be reasons for which the agent acts.
(Of course there are reasons which explain why motor actions happen; but these need not be reasons for which agents act.)
So motor joint action does not require that one agent's motor representations provide reasons for which another agent acts.
Instead, what is required is this.
There should be a good chance---good relative to the potential costs and benefits of attempting this particular joint action now---that social motor representation will provide the necessary coordination.
Of course this could be guaranteed by common knowledge. 
But common knowledge is not required.
Alternatively it can be ensured by common planning processes and a common {background} of dispositions, habits and expectations.%
\footnote{
Another possible line of justification the claim that common knowledge is involved in shared intention might start from a generalisation of Davidson's claim that
`[a]ction does require %that what the agent does is intentional under some description, and this in turn requires 
...\ that what the agent does is known to him under some description' \citep[p.\ 50]{Davidson:1971fz}.
%If we accepted this claim, then it seems we might have to conclude that not all joint actions are actions. 
%While this may appear to be an obstacle to accepting our view, there is an independent reason for thinking that not all joint actions are actions.
%For if we also follow Davidson in accepting that all actions are bodily movements (or Hornsby in accepting that all actions are tryings), it turns out that few paradigm cases of joint action are actions.
%For instance, we might paint a house together or make a hollandaise sauce together without there being any bodily movements of which we are both agents.
} 
%common background is for reciprocity of motor representation and similar willingness to engage in joint action.

If, as we have just argued, social motor representations play a role analogous to the structure of intentions and knowledge which Bratman identifies as sufficient for shared intention, then this is a  (non-decisive)  reason to think that motor representations can ground the purposiveness of a joint action.

%
%\section{The Contrast Cases}
%We have been arguing that social motor representation can bind multiple agents' activities together and link them to an outcome, and this in much the way that shared intention does  so.
%This is a reason, 
%not decisive but perhaps sufficient in the absence of strong contrary reasons, 
%for supposing that social motor representation plays a role in explaining which events are joint actions.
%Perhaps some events are joint actions in virtue of being appropriately related not to shared intention but to social motor representation.
%
%As mentioned at the start, our aim is to extend and generalise existing theories, not to replace them. 
%But at this point our answer to the question of which events are joint actions seems to involve a disjunction.
%This is both puzzling and inelegant. 
%(Unless, of course, social motor representation can be regarded as a special case of shared intention.  
%We explain why we reject this possibility below.)
%In this section we explain how an account of joint action in terms of shared intention can be generalised to accommodate the possibility that social motor representation plays a role in explaining which events are joint actions.
%
%What guides and constrains theorising about shared intention? 
%While there is little agreement on what shared intention is or what it is for, intuitions about it are sometimes grounded in contrast cases.
%Contrast cases are pairs of events which are similar in terms of the behaviour and coordination they involve but where one is a joint action while the other is not.  
%Thus \citet{gilbert_walking_1990} contrasts two people walking together with two people individually walking side by side.  
%The two pairs' movements may be the same and similarly coordinated (to avoid collision), but walking together is a joint action whereas merely walking side by side is not. 
%Relatedly,  \citet{Searle:1990em}  contrasts a case in which several park visitors simultaneously run to a central shelter in order to perform a dance with another case in which the park visitors run to the central shelter in order to escape a storm.  The first is a case of joint action, the second is not; but the same movements occur in both.  
%These sorts of contrast case invite the question, 
%How do joint actions differ from individual but parallel actions? 
%Gilbert’s example shows that the difference can’t just be a matter of coordination, because people who are merely walking alongside each other also need to coordinate their actions in order to avoid colliding.  
%And Searle’s example shows that the difference between joint action and parallel individual action can’t just be that the actions have a common effect because merely parallel actions can have common effects too. 
%
%How might the contrast cases be used to guide and constrain theorising about shared intention?
%The idea that they serve this function seems to rest on a premise:
%It is possible to distinguish systematically between the contrast cases by appeal to shared intention and only by appeal to shared intention.
%We shall show that there is another, more general way to distinguish contrast cases.
%
%Take Searle's example of several people running to the shelter.  
%In the joint case, there is a single outcome to which each person's actions is individually directed, namely their collective arrival at the shelter.  
%In the contrasting individual case, where park visitors individually run to the shelter to escape a storm, there is no single outcome to which each of their actions is individually directed.  
%Instead each visitor's actions are directed to that visitor's own arrival at the shelter.  
%Similarly, turning to Gilbert's example, when two people walk together, there is a single outcome to which both of their actions are individually directed (their collective arrival at a corner, say); whereas when two people merely walk side by side there is no single outcome to which each agent's actions are directed.  
%In general, where two or more agent's actions constitute a joint action, there is an outcome to which each agent's actions are individually directed such that it is possible for them all to succeed relative to this outcome.%
%\footnote{
%Some readers may be sceptical of this claim.
%But note that it is a consequence of the view we are opposing, the view that all joint actions involve shared intention.
%Or, rather, it is a consequence of this view given a further premise which is a consequence of any standard account of shared intention: Where several agents act on a shared intention that they J, 
%each will perform an action directed to J and it is possible for them all to succeed in J-ing.
%} 
%Since this is not true in the contrasting cases of non-joint action, it allows us to distinguish systematically between joint actions and their non-joint but behaviourally and coordinatively indistinguishable counterparts.  
%
%For concision let us stipulate that an outcome is a \emph{distributive goal} of two or more agents' actions just if two conditions are met.
%First, this outcome is a goal to which each agent's actions are individually directed.
%Second, each agent's actions are related to the goal in such a way that it is possible for all the agents (not just any agent, all of them together) to succeed relative to this goal.
%In these terms, our claim is that joint actions involve distributive goals and that this distinguishes joint actions from non-joint actions in the standard contrast cases.  
%Since social motor representation and shared intention are each sufficient for the existence of a distributive goal, we can conclude that systematically distinguishing between standardly considered contrast cases does not require shared intention.
% 
%A natural response to this argument would be to suggest that the standard contrast cases are insufficient. 
%For in some cases two agents' actions can have a distributive goal although arguably there is no joint action. 
%Nora and Olive killed Fred.  
%Each fired a shot.
%Each intended that her shooting ground or partially ground Fred's death.%
%\footnote{
%Events $D_1$, ...\ $D_n$ \emph{ground} $E$, if: $D_1$, ...\ $D_n$ and $E$ occur; 
%$D_1$, ...\ $D_n$ are each part of $E$; and 
%every event that is 
%	a part of $E$
%	but does not overlap $D_1$, ...\ $D_n$ 
%is caused by some or all of $D_1$, ...\ $D_n$.
%(This is a refinement and generalisation of a notion due to \citet{pietroski_actions_1998}.)
%Event $D$ \emph{partially grounds} event $E$ if $D$ alone does not ground $E$ but there are events including $D$ which do ground $E$.
%} 
%%
%As it turned out, both intentions were fulfilled.
%Neither shot was individually fatal but together they were deadly.
%An ambulance arrived on the scene almost at once but Fred didn't make it to the hospital.
%Now Nora and Olive's actions have a distributive goal.
%After all, 
%	each agent's actions are individually directed to Fred's death
%	and  
%	it is consistent with the stipulations made about this scenario that these goal relations are compatible in the sense that both agents could succeed together. 
%But is their killing of Fred a joint action?
%Imagine that Nora and Olive had no knowledge of each other, nor of each other's actions, and that their efforts were entirely uncoordinated.
%We might even suppose that Nora and Olive are so antagonistic to each other that they would, if either knew the other's location, turn their guns on each other.
%Their actions nevertheless have a distributive goal.
%But given these further suppositions it is likely to seem counterintuitive to suppose that their killing Fred was a joint action. 
%So although we can distinguish the standardly considered contrast cases just by appeal to the notion of a distributive goal, we should not assume that this notion captures the intended contrast between joint and merely parallel actions.
%
%***HERE refine: collective goals
%
%
%
%\section{Social motor representation: unlike shared intention}
%
%*function (not to coordinate planning in Bratman's sense)
%
%*format (based on earlier paper ...)
%


\section{Conclusion}
Our opening question was, Which events are joint actions?
The standard view is that for an event to be a joint action it must be appropriately related to a shared intention.
We are in the process of arguing that this is not the whole truth about joint action,
and that some events are joint actions by virtue of being appropriately related to a structure of motor representations we call social motor representation.
We don’t mean to suggest that all joint actions involve social motor representation.
The view we are aiming to establish is rather this: Some events are joint actions in virtue of being appropriately related to social motor representations which bind their components together and ensure that there is a single outcome to which these components are collectively directed.
This is why 
%Recap: the question was: Does social motor representation  play a role in explaining what joint is?
%I have just been arguing for a positive answer.
%My thesis is this:
%\textbf{Reciprocal agent-neutral motor representations coordinate multiple agents’ actions around an outcome in part by virtue of representing that outcome.}
%That is, reciprocal social motor representations can ground the purposiveness of joint action.
%This is why I think that 
understanding which events are joint actions requires understanding not only shared intention but also
 the coordinating role of social motor representation.

This gives rise to a challenge. 
Effective joint action—imagine two people erecting a tent in a gale together—sometimes requires both shared intentions and social motor representation plus a certain kind of harmony between the two. 
To characterise the harmony required, let us define a notion of matching:
%
\begin{quote}
Two  outcomes, A and B, \emph{match} in a particular context just if, in that context, either the occurrence of A would normally constitute or cause, at least partially, the occurrence of B or vice versa. 
\end{quote}
%
When joint action involves both shared intention and social motor representation, the outcomes each of these representations specify should match.
This is the required harmony.
The challenge is to explain how matching could ever non-accidentally occur.
%This will be particularly difficult if you suppose, as we do, that motor representation and intention are not inferentially integrated. 

I think we can see a possible solution to this challenge by considering another problem.
In building shared intention from ordinary individual intention, we need intentions \emph{that we J}.
As has been much discussed, the contents of these intentions cannot all refer to actions involving shared intentions.
For this reason Michael Bratman suggests that things we intend are  cooperatively neutral activities.
It is then necessary to add further intentions in order to transform cooperatively neutral activities into joint actions.
But it also seems possible that in some cases, what we intend when we intend that we J is not a cooperatively neutral activity but instead a joint action of the sort which involves social motor representation.

So perhaps harmony between shared intention and social motor representation is sometimes achieved in this way: what we intend when we share an intention is the sort of joint action that involves social motor representation.



\bibliography{$HOME/endnote/phd_biblio}

\end{document}